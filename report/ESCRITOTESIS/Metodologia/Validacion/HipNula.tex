El prop\'osito de la inferencia estad\'istica es hacer conclusiones
acerca de una poblaci\'on con base en los datos obtenidos de una
muestra de la poblaci\'on. Probar una hip\'otesis es el proceso para
evaluar la fuerza de la evidencia de la muestra y proporcionar un 
marco de trabajo para hacer determinaciones acerca de la poblaci\'on,
es decir, proporcionar un m\'etodo para entender que tan confiables 
uno puede extrapolar las observaciones encontradas en la muestra bajo
estudio a una gran poblaci\'on de la cual la muetra fue extraida.

El primer paso en la prueba de hip\'otesis es la transformaci\'on de
la pregunta investigada dentro de la hip\'otesis nula, $\mbox{H}_0$,
y una hip\'otesis alternativa, $\mbox{H_a}$. Las hip\'otesis nula y 
alternativa son enunciados concisos, usualmente en forma 
matem\'atica, de dos posibles versiones de la verdad acerca de la 
relaci\'on entre la predicci\'on de inter\'es y el resultado de la 
poblaci\'on. Las dos posibles versiones de la verdad deben de ser
exhaustivas (es decir, cubrir todas las posibles verdades) y 
mutuamente excluyentes (es decir, que no se traslapen). La 
hip\'otesis nula es convencionalmente usada para describir la falta 
de asociaci\'on entre la predicci\'on y el resultado; la hip\'otesis
alternativa describe la existencia de una asociaci\'on y es 
t\'ipicamente la que el investigador le gustar\'ia mostrar.

Cuando un investigador est\'a dise\~nando un estudio acoplado o en
pares el an\'asis apropiado toma en cuenta las parejas. El m\'etodo 
de parejas $t$ para comparar medias muestras pareadas. Para cada par
de observaciones se calcula la diferencia entre ellas, $d_i$. Si dos
grupos tienen la misma media, de esperar\'ia que la diferencia entre 
las medias se encuentre alrededor de cero. El estad\'istico a usar es
\be
t=\frac{\bar d}{s/\sqrt{n}},
\ee
donde el numerador el la media de las diferencias y el denominador es
el error est\'andar de $\bar d$.
