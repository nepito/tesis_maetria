\subsection{Pseudopotenciales}
%DolgChemRev2012.112.403
%\paragraph{Electrones de valencia y de carozo}
Los conceptos de valencia y electrones de carozo son familiares para
todo qu\'imico y fundamental, por ejemplo, para el ordenamiento de 
los elementos en la tabla peri\'odica. Para muchas consideraciones
cualitativas la qu\'imica de un elemento es solo determinada por sus
electrones de valencia, los cuales participan activamente en los 
enlaces qu\'imicos. En esta descripci\'on los electrones de carozo 
permanecen esencialmente sin participaci\'on y a lo m\'as juegan un 
rol indirecto, por ejemplo, proporcionando junto con el n\'ucleo un 
potencial efectivo modificado para los electrones de valencia dentro
de un grupo de la tabla peri\'odica. Las modificaciones en los 
potencial efectivo no solo llevan a diferencias cuantitativas para 
los \'atomos, por ejemplo, en los potenciales de ionizaci\'on, 
afinidad electr\'onica y energ\'ias de excitaci\'on, sino tambi\'en 
para mol\'eculas, por ejemplo, las fuerzas en los enlaces qu\'imicos 
formados por estos \'atomos, y as\'i afectando su comportamiento 
qu\'imico \citep{Dolg2012}.

%\paragraph{Potenciales Efectivos de Carozo}
El m\'etodo de potenciales efectivos de carozo (ECPs, por sus siglas 
en ingl\'es) hace uso de estas ideas en los c\'alculos te\'oricos de 
la estructura electr\'onica a primeros principios. Las principales 
finalidades son una considerable reducci\'on del esfuerzo 
computacional evitando el costoso tratamiento expl\'icito de los 
carozos at\'omicos en los c\'alculos y al mismo tiempo un tratamiento
impl\'icito de la mayor\'ia de los efectos relativistas para el 
sistema de electrones de valencia. La eliminaci\'on de los carozos
at\'omicos de los c\'alculos permite tratar todos los elementos de un
grupo de la tabla peri\'odica de igual manera. Ya que el mismo 
n\'umero de electrones de valencia deben de ser tratados para todos
los elementos de una columna de la tabla peri\'odica, se espera que
el mismo esfuerzo computacional y una precisi\'on similar sea lograda,
es decir, la variaci\'on en la calidad de los resultados causada por la 
gran diferencia en el tama\~no de los sistemas es evitada. 
Claramente, tal expectaci\'on descansa en la suposici\'on discutible 
de que el formalismo de los ECPs trabaja igualmente bien para todos los 
elementos de un grupo de la tabla peri\'odica. 

%\paragraph{Efectos Relativistas}
Los efectos relativistas para los elementos pesados no pueden ser 
ignorados ya que provocan cambios cuantitativos e incluso a veces 
cualitativa a la precisi\'on de la investigaci\'on. Estas
contribuciones relativistas requieren un esfuerzo computacional 
adicional a en c\'alculos de todos los electrones (AE, por sus siglas
en ingl\'es). Dentro de la descripci\'on ECP ellas son usualmente 
incluidas por medio de una simple parametrizaci\'on del hamiltoniano 
modelo solo de valencia elegido con respecto a datos de referencia 
relativistas de AE. La descripci\'on ECP escalar relativista puede 
hacer uso de toda la maquinaria sin cambio de la qu\'imica cu\'antica
no-relativista cuando los efectos esp\'in-\'orbita pueden ser 
ignorados, sin embargo logrando resultados comparables a aquellos 
c\'alculos de mayor costo, escalar relativista de AE. Los efectos SO 
pueden ser tomados en cuenta en una descripci\'on ECP usando varias 
estrategias, por ejemplo, un tratamiento perturbativo subsecuentes a
los c\'alculos escalares relativistas, esto es, como el \'ultimo paso
del c\'alculo, o para su rigurosa inclusi\'on variacional desde el 
comienzo de los c\'alculos. Que los ECPs se vuelven un caballo de 
batalla de la qu\'imica cu\'antica relativista no es sorprendente, 
esto se debe que los resultados obtenidos en la mayor\'ia de las 
investigaciones de problemas qu\'imicos por los modernos esquemas ECP
son tan precisos como aquellos obtenidos de los m\'as costosos, y
tambi\'en aproximados c\'alculos relativistas AE, por lo que producen
m\'as resultados que ning\'un otro m\'etodo relativista. 

%\paragraph{Nomenclatura de potenciales efectivos de carozo}
Los potenciales modelos intentan construir potenciales efectivos
anal\'iticos modelando directamente el potencial HF no local original
para los electrones en orbitales de valencia los cuales tienen la 
estructura nodal completa de los orbitales de valencia AE. Esto se
logra haciendo uso de la ecuaci\'on de Huzinaga-Cantu y ser\'a 
referida como una descripci\'on de potenciales modelos HF (MP). 
Hay otro m\'etodo para la descripci\'on de Pseudopotenciales que 
modelan el potencial no local HF usando orbitales de pseudovalencia, 
esto es, orbitales que despu\'es de una transformaci\'on aun den las 
energ\'ias orbitales correctas, pero tienen una estructura nodal 
radial simplificada. Este m\'etodo se basa en la ecuaci\'on de 
Phillips-Kleinman y ser\'a referido como la descripci\'on de 
pseudopotenciales modelos HF (PP).

%\paragraph{Aproximaciones Pr\'acticas}%pag 417
Algunas aproximaciones deben de ser hechas para lograr un esquema 
pr\'actico computacionalmente de los c\'alculos de Orbitales de 
Valencia. Primero se deben de separar los sistemas en electrones de 
valencia y los de carozo. Segundo, se debe suponer que los carozos 
at\'omicos est\'an inertes y se mantienen sin cambio cuando van de un
sistema a otro. Tercero, para eliminar los electrones de carozo y 
posiblemente tambi\'en los orbitales internos de los c\'alculos se 
debe reemplazar su contribuci\'on en un hamiltoniano de AE a los 
electrones de valencia por un potencial efectivo de carozo modelando 
el potencial HF no local real. Y cuarto, los m\'etodo ECP casi 
siempre asumen que los carozos at\'omicos no interaccionan con los 
carozos de otros \'atomos excepto por su repulsi\'on coulombiana.

{\normalsize
\subsection*{Base de \cite{Yang2005}}
%\paragraph{Orbital de Cristal}
%Esta informaci\'on fue tomada de old.iupac.org
Orbital de cristal (sin\'onimo banda orbital), es la funci\'on de un
electr\'on extendida a trav\'es de un cristal. T\'ipicamente una 
combinaci\'on lineal de orbitales de Bloch:
$$\Psi_j=\sum_{\mu}a_{j\mu}\phi_{\mu}.$$
Los orbitales de Bloch incorporan la simetr\'ia de translaci\'on
aplicando condiciones a la frontera
$$\phi_{\mu}(\vec r)=\sum_n\exp{ikr}\chi_{\mu}(\vec r)$$
donde $\chi_{\mu}(\vec r)$ es el n-\'esimo orbital at\'omico definido 
el vector de translaci\'on $\vec r$. Los diferentes valores adoptados
por el vector de onda $\vec k$ determinan la simetr\'ia y las 
propiedades nodales del orbital de cristal.

%\paragraph{?`Porqu\'e orbitales de cristal}
%YangTheorChemAcc2005.113.212
Gracias a los recientes logros en la descripci\'on y predicci\'on las
propiedades de bulto de los materiales, los c\'alculos de las 
estructuras electr\'onicas se han vuelto m\'as importantes tanto en
qu\'imica como en f\'isica de materia condensada para dise\~nar y 
preparar materiales con propiedades controladas. El acceso a 
c\'odigos complicados y poder de c\'omputo ha hecho posible 
experimentos computacionales {\it ab initio}. Estos c\'alculos de 
primeros principios fueron hechos com\'unmente usando bases de ondas 
planas junto con Pseudopotenciales y teor\'ia del funcional de la 
densidad en f\'isica del estado s\'olido. Los m\'etodos de Orbitales 
de Cristales gan\'o atenci\'on en el c\'alculo de s\'olidos desde la
implementaci\'on de la combinaci\'on lineal de orbitales at\'omicos 
Hartree-Fock (LCAO) en el programa CRYSTAL para el tratamiento de 
sistemas peri\'odicos. Los m\'etodos de ECP han sido aplicados a los 
lant\'anidos principalmente para eliminar la capa abierta 4f y los
principales problemas relacionados a esta. %(?`Cu\'ales son esos problemas?). 
Sin embargo, los conjuntos de valencia generados para pseudopotenciales 
consistentes con la energ\'ia de los lant\'anidos para aplicaciones
moleculares no pueden ser transferidos directamente a componentes 
peri\'odicos sin modificaci\'on. La raz\'on es que, frecuentemente,
los conjunto de bases at\'omicas con funciones difusas dan como 
resultado un traslape entre funciones de Bloch que construyen los 
orbitales de cristal y as\'i no solo dan como resultado un 
desperdicio de recursos computacionales sino que adem\'as pueden 
causar problemas con dependencia cuasilineal debido a las 
limitaciones num\'ericas. 

%\paragraph{Resumen del trabajo de \cite{Yang2005}}
\cite{Yang2005} optimizaron un conjunto de bases Gaussianas a la 
energ\'ia HF para toda la serie de los iones de lant\'anidos
trivalentes (desde lantano hasta lutecio), adem\'as las derivaron y 
adaptaron para c\'alculos de Orbitales de Cristal usando los 
pseudopotenciales de Stuttgart-Koel de consistencia energ\'etica. 

%\ocultartrue
%\begin{Vocultar}
%\paragraph{Breve Teor\'ia de los OC}
%La naturaleza de los orbitales de cristal $\psi_i(r;k)$ en sistemas 
%peri\'odicos, los cuales son una combinaci\'on lineal de funciones de
%Bloch $\phi_{\mu}(r;k)$ que son contraidas a funciones gaussianas 
%(\ref{OC}). Cada funci\'on de Bloch es definida como la suma de los 
%mismos orbitales at\'omicos (OA,$\varphi_{\mu}(r-A_{\mu}-g)$) 
%trasladados sobre un enrejado infinito con cada OA multiplicado por 
%una onda plana (\ref{FB}), las cuales son expresadas como una 
%combinaci\'on lineal funciones gaussianas generadas (\ref{FG}). 
%As\'i, el conjunto de bases real son las funciones de Bloch debido a
%la raz\'on los coeficientes de combinaci\'on $a_{\mu,i}$ est\'an 
%involucrados en el procedimiento variasional y contribuyen en los
%c\'alculos de los elementos de la matriz densidad. Ya que el esquema
%es conservado, la libertad variasional es constante, y las funciones
%de Bloch no son sensibles al aumento en el tama\~no de las funciones
%primitivas de (4s4p3d) a (6s6p5d) debido al t\'ermino de las ondas
%planas $\exp{ik\cdot g}$, aunque un aumento en el n\'umero de 
%primitivas proporciona una mejor descripci\'on de OA.
%
%\be\label{OC}
%\psi_i(r;k)=\sum_{\mu}a_{\mu,i}(k)\phi_{\mu}(r;k)
%\ee
%\be\label{FB}
%\phi_{\mu}(r;k)=\sum_{g}\varphi_{\mu}(r-A_{\mu}-g)\exp{ik\cdot g}
%\ee
%\be\label{FG}
%\varphi_{\mu}(r-A_{\mu}-g)=\sum^{n_G}_{j}d_jG(\alpha_j;r-A_{\mu}-g).
%\ee
%\end{Vocultar}
}

