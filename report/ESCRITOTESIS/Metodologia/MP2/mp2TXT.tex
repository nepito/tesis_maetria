\subsection{M\'etodos Post Hartree-Fock}
%{\it Introducci\'on a los m\'etodos post-Hartree}

%\paragraph{Breve Introducci\'on}
Los c\'alculos confiables de las energ\'ias electr\'onicas de mol\'eculas
requieren un tratamiento preciso de los efectos de muchas 
part\'iculas (correlaci\'on electr\'onica). El d\'ia de hoy, los 
qu\'imicos tiene una amplia diversidad de t\'ecnicas con diferentes 
costos y precisiones a su disposici\'on, la cual no solo es usada 
para los c\'alculos de las energ\'ias absolutas sino tambi\'en
relativas y otras propiedades de las mol\'eculas \citep{Grim2003}. 

%\paragraph{M\'etodos del presente trabajo}
En el presente trabajo, la energ\'ia de correlaci\'on fue tratada con
tres diferentes m\'etodos: teor\'ia de perturbaci\'on de varios 
cuerpos de M\o ller-Plesset (MP2), el m\'etodo emp\'irico original de 
los componentes de esp\'in escalados (SCS) y la teor\'ia del 
funcional de la densidad; los cuales son brevemente explicados a 
continuaci\'on.

\subsubsection{MP2}
%\paragraph{Introducci\'on a MP2}
Antes del uso masivo de la teor\'ia del funcional de la densidad, 
en los a\~nos ochentas y noventas del siglo pasado, la teor\'ia de 
perturbaci\'on de segundo orden de M\o ller-Plesset (MP2) era el
\'unico m\'etodo de estructura electr\'onica practicable para 
sistemas grandes, de muchas part\'iculas, que incluyera los efectos
de correlaci\'on electr\'onica en una manera razonable 
\citep{Grim2012}.

%\paragraph{Caricatura del M\'etodo}
MP2 es un m\'etodo basado en la teor\'ia de perturbaci\'on, su 
finalidad es dar un tratamiento a sistemas de $n$ electrones en el
cual la soluci\'on Hartree-Fock aparece como la aproximaci\'on de 
orden cero. Esto se muestra desarrollando la correcci\'on a primer
orden y observando que para la energ\'ia y la densidad de carga es 
cero. La expresi\'on para la correcci\'on a segundo orden para la 
energ\'ia se simplifica de gran manera gracias a las propiedades 
especiales de la soluci\'on de orden cero \citep{Moll1934}. La 
correcci\'on a segundo orden de la energ\'ia es
\be\label{ce2omp2}
E^{(2)}_0=\sum^{\infty}_{b=a+1}\sum^{\infty}_{a=j+1}
          \sum^{n}_{j=i+1}\sum^{n-1}_{i=1}
					\frac{|\langle ab|r^{-1}_{12}|ij\rangle-
					       \langle ab|r^{-1}_{12}|ji\rangle|^2}
					{\epsilon_i+\epsilon_j-\epsilon_a-\epsilon_b}.
\ee
La energ\'ia de correlaci\'on puede ser rigurosamente separada en dos
contribuciones  de pares de electrones con el mismo esp\'in (ss) y 
con esp\'in opuesto (op), los cuales son tratados de la misma manera
en el m\'etodo tradicional MP y agregan una energ\'ia de 
correlaci\'on total
\be\label{ecs}
E_c=E^{ss}_{c}+E^{op}_c=E^{(2)}_0,
\ee
donde
\be
E^{ss}_c=\frac{1}{2}\sum_{ij}e_{ij}+\frac{1}{2}\sum_{\bar i\bar j}
e_{\bar i\bar j},
\ee
\be
E^{os}_{c}=\sum_{i\bar j}e_{i\bar j}.
\ee

Aqu\'i los electrones con esp\'in $\beta$ son identificados con una 
barra sobre el \'indice. Los valores $e_{ij}$ son definidos como
\be\label{defeij}
e_{ij}=\sum_{ab}(T^{ab}_{ij}-T^{ba}_{ij})(ia|jb)
\ee
y de una manera equivalente se definen 
$e_{\bar i\bar j}$ y 
$e_{i\bar j}$. 
En el sistema de trabajo de la teor\'ia de perturbaciones 
M\o ller-Plesset  $T$, en la ecuaci\'on (\ref{defeij}), es la 
amplitud de la funci\'on de onda corregida a primer orden, donde 
$\epsilon$ representa la energ\'ia del orbital can\'onico $HF$
\be\label{defT}
T^{ab}_{ij}=\frac{ (ia|jb)}{\epsilon_i+\epsilon_j-
\epsilon_a-\epsilon_b},
\ee
donde tanto en (\ref{defeij}) como en (\ref{defT}) se us\'o la 
notaci\'on de Mulliken
\be\label{defnM}
(ia|jb)=\langle ab|r^{-1}_{12}|ij\rangle.
\ee
Entonces, utilizando (\ref{defeij}) y sus equivalentes para 
$e_{\bar i\bar j}$ y 
$e_{i\bar j}$ en (\ref{ecs}) y se sustituye (\ref{defeij}) y 
(\ref{defT}) se recupera la ecuaci\'on (\ref{ce2omp2}). 
%\paragraph{Algunos Problemas con MP2}
%{\it Informaci\'on del art\'iculo de} \cite{Olse1996,Grim2012,Geer2003}
%{\it Buscar informaci\'on en los libros que ya revis\'e...y del 
%art\'iculo original de} \cite{Moll1934}.



El m\'etodo MP2 es considerado ser menos preciso comparado con los
mejores m\'etodos de funcionales de la densidad, por ejemplo B3LYP 
\citep{Grim2003} y adem\'as no ser tan robusto cuando es aplicado a
problemas complicados de correlaci\'on que ocurren en estados de 
transici\'on o en componentes que contengan metales. Adem\'as, se ha
encontrado que la convergencia de las series M\o ller-Plesset depende
en gran medida en el conjunto bases de un electr\'on y, en algunos 
casos, al agregar funciones bases difusas, las series se comportan de
manera divergente, aun para sistemas dominados por su configuraci\'on
{\it simple} \citep{Olse1996}, se sabe que los problemas de 
convergencia en las series M\o ller-Plesset aparecen en conjuntos de 
bases extendidas. Con la intenci\'on de incrementar la precisi\'on 
del m\'etodo MP2 en los casos problem\'aticos y mantener sus 
propiedades m\'as atractivas \cite{Grim2003} implement\'o una 
parametrizaci\'on en las contribuciones del esp\'in, el m\'etodo 
SCS-MP2.

\subsubsection{SCS-MP2}
%{\it Agregar informaci\'on de }\cite{Grim2003} {\it y buscar m\'as
%art\'iculos en SCPUs para intentar debatir su utilizaci\'on.}

%\paragraph{Introducci\'on SCS-MP2}
El m\'etodo de los componentes escalados del esp\'in (SCS, por sus 
siglas en ingl\'es) conserva las caracter\'isticas deseables del 
tradicional MP2 (consistencia de tama\~no, invariante ante rotaciones
orbitales, simplicidad computacional) para tratar la correlaci\'on 
electr\'onica, mientras mejora su precisi\'on general a un nivel en 
el que es competitivo a los DFT contempor\'aneos. Los m\'etodos SCS 
se establecieron como m\'etodos robustos de estructura electr\'onica 
en qu\'imica cu\'antica desde su desarrollo hace m\'as de diez 
a\~nos. Por ejemplo, SCS-MP2 es rutinariamente usado para problemas 
de los estados bases cuando se sabe que los m\'etodos DFT fallan.

%\paragraph{Teor\'ia de SCS-MP2}


\subsubsection{DFT}
%{\it Agrgar la informaci\'on de }\cite{Geer2003}
%\paragraph{Introducci\'on DFT}
La teor\'ia del funcional de la densidad (DFT, por sus siglas en 
ingl\'es) proporcion\'o una base para el desarrollo de estrategias
computacionales para obtener informaci\'on  acerca de las energ\'ias,
las estructuras y las propiedades de las mol\'eculas a un costo 
computacional mucho menor que las t\'ecnicas de funci\'on de onda
(WFT) \citep{Geer2003}. La forma funcional necesaria para describir 
el intercambio y correlaci\'on no puede ser derivada de primeros 
principios, por lo que DFT no es un m\'etodo {\it ab initio} puro, 
sin embargo, puede lograr la precisi\'on de estos m\'etodos y 
puede ser usado en sistemas mucho m\'as grandes \citep{Bere2007}.

%\paragraph{Sustento te\'orico}
Los m\'etodos DFT est\'an basados en los famosos teoremas de 
\cite{Hohe1964} y la idea b\'asica es que la densidad de carga
electr\'onica, $\rho(\vec r)$, en cada punto $\vec r$ determina el 
estado base exacto de la funci\'on de onda y la energ\'ia de un 
sistema de electrones.

%\paragraph{Caricatura de DFT}
Los funcionales actualmente usados no tiene una correcci\'on a  la
auto interacci\'on, debido a esto, los funcionales usualmente 
disponibles dan curvas $U(R)$ muy incorrectas a grandes distancias
internucleares para iones radicales sim\'etricos y sobrestiman la 
interacci\'on intermolecular en algunos complejos de transferencia de
carga \citep{Levi2001}. Si se conoce la densidad electr\'onica del
estado fundamentas $\rho_0(\vec r)$, el teorema de Hohenberg-Kohn dice
que es posible calcular todas las propiedades moleculares del estado 
fundamental a partir de $\rho_0$ sin necesidad de obtener la funci\'on
de onda molecular. El teorema de Hohenberg-Khon no dice c\'omo 
calcular $E_0$ a partir de $\rho_0$, ni tampoco c\'omo obtener 
$\rho_0$ sin obtener primeramente la funci\'on de onda.  Khon y Sham
idearon  un m\'etodo pr\'actico para obtener $\rho_0$ y para obtener 
$E_0$ a partir de $\rho_0$. Este m\'etodo es capaz, en principio, de
obtener  resultados exactos, pero debido a que las ecuaciones del 
m\'etodo de Khon-Sham contienen un funcional desconocido que debe 
aproximarse, la formulaci\'on da lugar a resultados aproximados.

%\paragraph{Problema con DFT}
El teorema variacional de Hohenberg-Kohn dice que se puede obtener la
energ\'ia del estado fundamental variando $\rho_0$ de forma que 
minimice el funcional $E(\rho)$. La falta de un procedimiento 
sistem\'atico para mejorar $E_{ci}$ y, por lo tanto, para mejorar las
propiedades moleculares calculadas, es el principal inconveniente del 
m\'etodo.
