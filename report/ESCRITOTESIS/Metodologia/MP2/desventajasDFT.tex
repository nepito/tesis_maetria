%Informaci\'on tomada de Levy
La teor\'ia de Hohenber-Khon-Sham es b\'asicamente una teor\'ia del
estado fundamental. Se han desarrollado m\'etodos que aplicables a 
estados excitado, pero la toer\'ia no alcanzado el punto que permite
que se efect\'uen c\'alculos precisos y pr\'acticos de estados 
moleculares individuales excitados. 

Los funcionales act

Debido a que se usan aproximaciones de funcionales, el DFT KS no es
variasional, y puede dar lugar a una energ\'ia m\'as baja que la 
energ\'ia fundamental verdadera. Con m\'etodos CC, CI y MP, la forma
de alcanzar resultados m\'as precisos es clara. Se usa una bases 
m\'as grandes y se va \'ordenes elevados de correlaci\'on, aunque el
alcance de la investigaci\'on para mol\'eculas de un tama\~no dado
est\'a limitados por la potencia del c\'alculo disponible 
actualmente. En DFT KS, no est\'a clara la v\'ia para contruir
funcionales $E_{ci}$ m\'as precisos; se deben probar uno a uno los
nuevos funcionales para ver cual de ellos dan meejor resultado.


