%\paragraph{El m\'etodo/Introducci\'on}
\section{C\'alculos cu\'anticos en fase gaseosa}
Los problemas que surgen, cuando la hidrataci\'on de Ln$^{3+}$ es 
abordada computacionalmente, son debidos al gran n\'umero de 
electrones involucrados, a la importancia de los efectos relativistas
y a la capa f abierta de los lant\'anidos. La mayor\'ia de las 
investigaciones te\'oricas a nivel cu\'antico referentes a este tema 
usan Pseudopotenciales (PP) y teor\'ia del funcional de la densidad,
de esta manera resuelven los dos primeros problemas y si la capa f es
agregada al pseudopotencial, tambi\'en el tercer problema es resuelto
\citep{Ciup2010}. Las distancias Ln$^{3+}$-O te\'oricas son mayores a
las experimentales cuando los c\'alculos son hechos utilizando PP y 
DFT como se muestra en los art\'iculos de \cite{Kuta2010} y los 
trabajos citados por \cite{Dang2012}. Como ya se dijo, una probable
explicaci\'on a las distancias te\'oricas largas, dada por 
\cite{Kuta2010}, es la ausencia de los efectos de bulto al solo 
considerar expl\'icitamente las aguas de la primera capa de 
hidrataci\'on, esta explicaci\'on es sustentada en el hecho de una 
contracci\'on en las distancias promedios al agregar las mol\'eculas 
de agua de la segunda capa. Los c\'alculos DFT son lo suficientemente
``baratos'' como para permitir los c\'alculos de la hidrataci\'on 
incluyendo mol\'eculas de la segunda esfera de hidrataci\'on. Sin 
embargo, este procedimiento tiene la desventaja de presentar 
m\'ultiples m\'inimos. En el presente trabajo se toman en cuenta 
los efectos de las esferas de hidrataci\'on mayores usando un modelo 
de medio continuo polarizable. Lo anterior se combin\'o
con pseudopotenciales escalares relativistas con la capa 4f en el 
carozo junto con DFT y m\'etodos de teor\'ia de perturbaci\'on, para
tomar en cuenta los efectos de correlaci\'on. A continuaci\'on se 
explica en que consisten los c\'alculos cu\'anticos, en particular el
m\'etodo Hartree-Fock. Despu\'es una explicaci\'on de los m\'etodos
para considerar los efectos de correlaci\'on es dada, los m\'etodos
perturbativos de muchos cuerpos son los m\'as relevantes para este 
trabajo. En seguida, la propuesta de \cite{Toma2005} para un medio
polarizable continuo (D-PCM) es brevemente revisada. Finalmente, el
cap\'itulo termina con el tema de algoritmos gen\'eticos, herramienta
usada para obtener las configuraciones optimizadas.

\subsection{M\'etodo Hartree-Fock}
%DolgChemRev2012.112.403
Es com\'unmente aceptado que la mec\'anica cu\'antica necesita ser
aplicada a la qu\'imica %para obtener una descripci\'on te\'orica no
emp\'irica precisa de los \'atomos, mol\'eculas y s\'olidos. 
Entonces, si la mec\'anica cu\'antica es utilizada, la ecuaci\'on 
central que se debe resolver es la ecuaci\'on de Schr\"odinger
dependiente del tiempo
\be\label{eSdt}
\hat H |\Psi\rangle=i\frac{\partial}{\partial t}|\Psi\rangle
\ee
donde $\hat H$ es el hamiltoniano que define al sistema y el estado
del sistema es descrito por la funci\'on de onda $|\Psi\rangle$. 
Si el hamiltoniano $\hat H$ no depende expl\'icitamente del tiempo, 
la funci\'on de onda ser\'a
\be\label{ef}
|\Psi\rangle=|\psi\rangle\cdot e^{-iEt}
\ee
y la ecuaci\'on de Schr\"odinger independiente del tiempo para 
estados estacionarios con energ\'ia total $E$ y descritos por la 
funci\'on de onda toma la siguiente forma
\be\label{eSit}
\hat H|\psi\rangle=E|\psi\rangle.
\ee

%\paragraph{Primera Aproximaci\'on}
La aproximaci\'on de Born-Oppenheimer (BO) permite una separaci\'on
aproximada  de la ecuaci\'on de Schr\"odinger independiente del 
tiempo (\ref{eSit}) en una parte nuclear y una parte electr\'onica, 
la cual trata a los electrones en el campo del n\'ucleo, y a \'este 
en una posici\'on fija, esta aproximaci\'on corresponde a la 
ecuaci\'on para la mayor\'ia de los m\'etodos de estructura 
electr\'onica en qu\'imica cu\'antica. Solo la aproximaci\'on BO 
permite a los qu\'imicos escribir las f\'ormulas estructurales con 
s\'imbolos qu\'imicos de los \'atomos denotando la posici\'on de los 
n\'ucleos como los orbitales internos o de carozo, as\'i  como los 
puntos como los electrones de valencia. Sin embargo mayores 
aproximaciones son necesarias para llegar a tan  simple 
representaci\'on esquem\'atica de \'atomos, mol\'eculas y s\'olidos 
con base en consideraciones qu\'imico cu\'anticas.

%\paragraph{El Hamiltoniano}
En c\'alculos de qu\'imica cu\'antica se considera un hamiltoniano
general $\hat H$ para los $n$ electrones y los $N$ n\'ucleos, en 
ausencia de campos externos, esto es
\be
\hat H=\sum^n_i\hat h (i)+\sum^n_{i<j}\hat g(i,j)+\sum^N_{\lambda<\mu}
\frac{Z_{\lambda}Z_{\mu}}{ r_{\lambda\mu}}.
\ee
Los \'indices $i$ y $j$ denotan electrones, mientras $\lambda$ y 
$\mu$ son de los n\'ucleos. $Z_{\lambda}$ es la carga del n\'ucleo 
$\lambda$-\'esimo. Los operadores de una y dos part\'iculas, $\hat h$ 
y $\hat g$, pueden ser de una forma relativista, cuasirelativista o 
no relativista.

%\paragraph{M\'etodos de la Mec\'anica Cu\'antica}
La soluci\'on exacta de la ecuaci\'on de Schr\"odinger independiente
del tiempo (\ref{eSit}) a\'un bajo la suposici\'on de la 
aproximaci\'on BO, es solo realizable para sistemas de un electr\'on. 
Una gran parte de la qu\'imica cu\'antica trabaja en el desarrollo de
m\'etodos para la aproximaci\'on, aunque suficientemente precisa de 
la ecuaci\'on (\ref{eSit}). Las dos principales estrategias 
desarrolladas durante al menos las \'ultimas nueve d\'ecadas son la 
teor\'ia basada en funciones de onda (WFT) y teor\'ia del funcional 
de la densidad (DFT). 

%\paragraph{Determinante de Slater}
La descripci\'on Hartree-Fock es el m\'etodo central de la teor\'ia
{\it ab initio} WFT. La funci\'on de onda de muchos electrones m\'as
simple es utilizada, en la cual los electrones obedecen el principio 
de antisimetr\'ia de Pauli y la indistinguibilidad como part\'iculas
elementales. El determinante de Slater para un sistema de $n$ 
electrones (de capa cerrada) es una combinaci\'on lineal 
antisimetrizada del producto de $n$ funciones de un electr\'on, esto
es, esp\'in-orbital ortonormal $|\varphi\rangle$
\be\label{fHF}
|\Phi_{HF}\rangle=\frac{1}{\sqrt{n!}}\det{||\varphi_1(1)\rangle
|\varphi_2(2)\rangle}\cdots|\varphi_n(n)\rangle|.
\ee

%\paragraph{M\'etodo Hartree-Fock}
%VleckRevModPhy1935.7.167
La esencia del m\'etodo Hartree es imaginar que un electr\'on dado se 
mueve en el campo producido por una mancha de electrones m\'as que en
el campo instant\'aneo de los electrones. En otras palabras, la 
posiciones moment\'anea de los electrones es remplazada por la 
posici\'on promedio, cuando la fuerza ejercida sobre el electr\'on 
dado es calculada. El campo de fuerzas obtenido es, por supuesto, el 
mismo que se obtendr\'ia de una distribuci\'on de fuerza 
electrost\'atica cuya densidad en cualquier regi\'on es proporcional 
a la fracci\'on total del tiempo de las cargas responsables en esa 
regi\'on particular. La gran ventaja del m\'etodo de Hartree es que 
reduce el problema de $n$ cuerpos a $n$ problemas de un cuerpo, ya 
que el proceso de promediar hace que la fuerza en una carga sea
independiente de la posici\'on de las otras cargas. Es claro que, en 
consecuencia, es inadecuadamente tomado el hecho que dos electrones
est\'an raramente juntos debido a la gran fuerza de repulsi\'on entre
ellos. 
%As\'i, el m\'etodo de orbitales moleculares (OM) da t\'erminos
%i\'onicos muy grandes.

%\paragraph{Ecuaci\'on de Fock}
La funci\'on de onda HF no resuelve la ecuaci\'on de Schr\"odinger
(\ref{eSit}); sin embargo, mapea aproximadamente el problema 
insoluble de muchos electrones a los problemas de un electr\'on en 
una distribuci\'on de carga efectiva, esto es, la ecuaci\'on 
(can\'onica) de Fock para esp\'in-orbital $|\varphi\rangle$ y sus 
energ\'ias orbitales $\varepsilon$
\be\label{ecF}
\hat F|\varphi_i\rangle=\varepsilon_i|\varphi_i\rangle.
\ee
Aqu\'i, $\hat F$ denota al operador de Fock, el hamiltoniano de un
electr\'on efectivo tomando en cuenta las interacciones (promedio) 
de un electr\'on esp\'in-orbital $|\varphi\rangle$ con el n\'ucleo y 
todos los otros electrones en el sistema. Ya que el operador de Fock
$\hat F$ depende de su eigenfunci\'on $|\varphi\rangle$, la
ecuaci\'on de Fock (\ref{ecF}) tiene que ser resuelta iterativamente
mediante un procedimiento de campo autoconsistente (SCF). La 
ecuaci\'on de Fock y la funci\'on de onda aproximada (normalizada), 
la ecuaci\'on (\ref{fHF}) son calculadas para el hamiltoniano exacto 
(\ref{eSit}) y aplicando el principio variacional de Ritz-Raykeigh se
obtiene
\be
E_{HF}=\langle\Phi_{HF}|\hat H|\Phi_{HF}\rangle\geq E_0,
\ee
con $E_0$ siendo la energ\'ia exacta del estado base. 

%\paragraph{M\'etodo Variacional}
La variaci\'on de la energ\'ia se hace respecto al cambio de los
orbitales $|\varphi\rangle$, bajo la restricci\'on de formar una 
base ortonormal. La mayor\'ia de los c\'alculos aplican el formalismo
de Roothaan y Hall, donde los orbitales $|\varphi\rangle$ son 
expandidos como una combinaci\'on lineal de funciones bases de una
part\'icula $|\chi\rangle$
\be
|\varphi_i\rangle=\sum_jc_{ij}|\chi_j\rangle,
\ee
por ejemplo, de funciones Gaussianas. La calidad de este conjunto 
base de funciones decide que tan cerca se puede llegar de la mejor
soluci\'on HF, esto es, el l\'imite HF puede ser alcanzado cuando se
usa una base completa  de una part\'icula.

%\paragraph{Introducci\'on a los m\'etodos post-HF} 
La diferencia entre la energ\'ia l\'imite HF $E_{HF}$ y la energ\'ia 
exacta $E$ en la ecuaci\'on (\ref{eSit}) es llamada la energ\'ia de 
correlaci\'on y es debida a la respuesta moment\'anea de los 
electrones y su interacci\'on mutua, la cual solo es tratada en 
promedio en el formalismo HF. Usualmente es impracticable el uso de 
una base de una o varias part\'iculas, as\'i que una de las 
principales tareas de la qu\'imica cu\'antica computacional es la 
selecci\'on de hamiltonianos accesibles y la combinaci\'on \'optima 
de bases de una y varias part\'iculas. As\'i, la elecci\'on de 
hamiltonianos impone algunas limitaciones pr\'acticas debido al 
esfuerzo computacional requerido. 

%\paragraph{Introducci\'on a PP}
No siempre un hamiltoniano te\'oricamente mejor fundamentado 
permitir\'a los c\'alculos pr\'acticos para la descripci\'on de 
los resultados experimentales. T\'ipicamente un apropiado 
compromiso con un hamiltoniano menos riguroso y la posibilidad de 
usar bases m\'as extendidas de uno y varios electrones da una mejor 
representaci\'on. 
% ({\bf NOTA: Esto puede justificar el uso
%de los PP que incluyen la capa 4f en lugar del uso de los ECPSC}).
