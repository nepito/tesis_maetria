\paragraph{M\'etodo Hartree-Fock}
%VleckRevModPhy1935.7.167
La esencia del m\'etodo Hartree es imaginar que un electr\'on dado se 
mueve en el campo producido por una mancha de electrones m\'as que en
el campo instantaneo de los electrones. En otras palabras, se 
remplaza las posiciones momentaneas de los electrones 
por su posici\'on promedio en el c\'alculo de la fuerza ejercida al 
electr\'on dado. El campo de fuerzas obtenido es, por supuesto, el 
mismo que se obtendr\'ia de una distribuci\'on electrost\'atica de 
fuerzas cuya densidad en cualquier regi\'on es proporcional a la 
fracci\'on total del tiempo de las cargas responsables  en esa 
regi\'on particular. La gran ventaja del m\'etodo de Hartree es que 
reduce el problema de $n$ cuerpos a $n$ problemas de un cuerpo, ya 
que el proceso de promediar ha hecho que la fuerza en una carga sea
indepediente de la posici\'on de las otras cargas. Es claro que, en 
consecuencia, es inadecuadamente tomado el hecho que dos electrones
est\'an raramente juntos debido a la gran fuerza de repulsi\'on entre
ellos. 
%As\'i, el m\'etodo de orbitales moleculares (OM) da t\'erminos
%i\'onicos muy grandes.
