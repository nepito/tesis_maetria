%Sadus1999 Molecular Simulation of Fluids. Cap. 1:Introduction
La simulaci\'on molecular es un t\'ermino gen\'erico que abarca a los
m\'etodos computacionales Monte Carlo y din\'amica molecular. El 
principal objetivo de la simulaci\'on molecular es proporcionar 
resultados exactos de problemas de mec\'anica estad\'istica sobre 
aquellos obtenidos  de soluciones aproximadas. La caracter\'istica 
que distingue la simulaci\'on molecular de otros m\'etodos de 
c\'alculo y aproximaciones es que las coordenadas moleculares del 
sistema van cambiando en acuerdo  con c\'alculo riguroso de las
energ\'ias o fuerzas intermoleculares.

Los resultados de una simulaci\'on molecular son determinados 
\'unicamente por la naturaleza del modelo te\'orico usado. Como
consecuencia, la comparaci\'on de los resultados de la simulaci\'on
con los datos experimentales es una medida de la presici\'on del
modelo. Las diferencias entre las medidas experimentales y los datos
de la simulaci\'on molecular pueden ser atribuidas a las fallas del
modelo para representar el comportamiento molecular.

El m\'etodo Monte Carlo es una estrategia que depende de 
probabilidades. Se generan intentos aleatorios de configuraciones; se
evalua un criterio de aceptaci\'on de la configuraci\'on, caluclando
el cambio de energ\'ia y otras propiedades de las configuraciones; y 
se compara el criterio de aceptaci\'on  con un n\'umero aleatorio y
se acepta o se rechaze la configuraci\'on. 

En un proceso Monte Carlo, una nueva configuraci\'on es t\'ipicamente
obtenida por desplazar, intercambiar, quitar o agragar una 
mol\'ecula. La nturaleza exacta de la probabilidad de transici\'on 
depende de la elecci\'on del ensemblepero esto simepre involucra la
evaluaci\'on de la energ\'ia en la ueva configuraci\'on y compararla
con la energ\'ia del estado existente.
