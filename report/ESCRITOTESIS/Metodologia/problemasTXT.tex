{\it La distancia media decrece monot\'onicamnete con el incremento
de la carga nuclear $Z$ y decrece con la disminuci\'on del n\'umero 
de hidrataci\'on $h$	}
Los problemas con los c\'alculos actuales son, si se trabaja con solo
la primera capa de hidrataci\'on:
si en los c\'alculos se usan pseudopotenciales que tomen en cuenta 
los efectos de correlaci\'on electr\'onica, ``en comparaci\'on con 
los resultados experimentales el c\'alculo de las distancias son 
sistem\'aticamente sobrestimadas por los m\'etodos DFT, pero la
tendencia general es de incrementar con el n\'umero Z'' 
\citep{ciup2010}. La sobrestimaci\'on de la distancia Ln$^{III}$-O
es principalmente adjudicado a ignorar los efectos de la 
solvataci\'on del bulto, que pudiera ser resuelta con la inclusi\'on
expl\'icita de la segunda esfera de hidrataci\'on  o por el uso de un
medio polarizable continuo durante el proceso de optimizaci\'on. 
Ambas posibilidades presentan problemas, por ejemplo, problemas de 
m\'inimos m\'ultiples y problemas de convergancia, respectivamente. 
Como una aproximaci\'on  a la segunda esfera de hidraci\'on se 
consideran sistemas con una mol\'ecula de agua en la segunda esfera.
{bf ?`C\'omo se modifica la distancia lant\'anido- ox\'igeno cuando
se agraga un agua en la segunda esfera?} (\cite{ciup201001} menciona
que el encontr\'o que {\it up to 0.7 pm}). Tambi\'en menciona que los
c\'alculos MP2 proporcionan distancias menores a cualquier c\'alculo
hecho utilizando DFT. {\it Obviously, the neglect of bulk solvation
effectcs at this level of theory does not lead to significantly 
different geometric parameters from those experimentally observed.}
Sin embargo ellos creen que los problemas de encontrar y la 
consideraci\'on de varias estructuras m\'inimas importantes  
{\it account} para efectos din\'amicos en la solvataci\'on hace el
modelado expl\'icito de las esferas de hidrataci\'on mayores 
{\it quite} desfavorable en comparaci\'on  con solo modelar la
primera esfera de hidrataci\'on expl\'icitamente y usar alg\'un 
modelo continuo de solvataci\'on.

\cite{ciup201001} encontraron que SCS_MP2 predice energ\'ias de 
amarre ligermante m\'as fuerte y las $\Delta$G$^{\circ}_H$ es 
tambi\'en. El n\'umero de hidrataci\'on $h$ el muy suave decremento 
en $\Delta$G$^{\circ}_H$ es atribuido a la aproximaci\'on del 
potencial y el promedio inherente  sobre varios estados 
electr\'onicos. Encontaron que que es necesario introducir el efecto
del bulto, por ejemplo COSMO PCM, para predecir los n\'umeros de
coordinaci\'on experimental. Adem\'as, si no se usa PCM los 
c\'alculos DFT predicen siempre un n\'umero de coordinaci\'on 7, y 
suponen que la repulsi\'on entre las mol\'eculas de agua es 
sobrestimada. Otra posible explicaci\'on son las no precisas en los
DFT en la descripci\'on de la interacci\'on de dispersi\'on y la 
sobrestimaci\'on de la fuerza en el enlace de hidr\'ogeno.

Finalmente, \ciet{ciup201001} concluyen	que una mayor mejoramiento de
la descripci\'on molecular requiere geometr\'ias optimizadas dentro
de un medio polarizable continuo que sea cuidadosamente 
parametrizado.  Esto deber\'ia de dar estructuras las cuales est\'en
en un mejor acuerdo con los datos experimentales. La optimizaci\'on
dentro de un COSMO-SCRF son extremadamente dif\'iciles debido a los
problemas de convergencia y a un n\'umero significante puntos sillas
de orden alto. El mejoramiento por un tratamiento de esferas de 
hidrataci\'on mayores requirir\'ia m\'etodos como din\'amica 
molecular para promediar sobre un gran n\'umero de configuraciones
estructurales la cuales son casi imposibles con una descripci\'on
qu\'imico cu\'antica est\'atica. 

Los enlaces en estos complejos pueden ser descritos principalmente
electrost\'aticos ya que  cargas de Mulliken y las cargas naturales
de +2.2 a +2.9 y de +2.65 a +2.55, respectivamente, son encontradas
en los iones lant\'anidos. La carga de Mulliken sugiere un enlace 
m\'as i\'onico a mayores n\'umeros de carga nuclear $Z$ lo cual es 
observado en una menos donaci\'on de carga del agua y as\'i una carga
efectiva mayor \citep{ciup201001}.

Hemos encontrado un decremento continuo en la carga Mulliken neta de
los cationes lant\'anidos del cerio al europio... Aunque las 
interacciones lant\'anidos agua son grandemente el\'ectricas in 
nature (carga-dipolo), la carga transferida interacciona contribuye
significativamente a la energ\'ia de enlace de los complejos de agua.
