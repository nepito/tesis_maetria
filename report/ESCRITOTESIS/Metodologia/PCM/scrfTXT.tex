% Levine Qu\'imica Cu\'antica pp 584
Quiz\'as la v\'ia m\'as com\'un para calcular los efectos del 
disolvente es usar un {\it modelo de disolvente continuo}. Aqu\'i, la
estructura del disolvente se ignora, y el disolvente se modela como 
un diel\'ectrico continuo de extensi\'on infinita, que rodea una 
cavidad que contiene la mol\'ecula de soluto M. El diel\'ectrico 
continuo se carecteriza por su constante diel\'ectrica. La mol\'ecula
de soluto puede tratarse cl\'asicamente como un conjunto de cargas
que interact\'uan con el diel\'ectrico o puede tratarse 
mecanocu\'anticamente. En el tratamiento mecanocu\'antico, la 
interacci\'on entre una mol\'ecula de soluto M y el continuo 
diel\'ectrico que la rodea, se modela mediante un t\'ermino 
$\hat V_{int}$ que se a\~nade al Hamiltoniano electr\'onico molecular
(de nucleos fijos) $\hat H^{(0)_M}$, donde $\hat H^{(0)_M}$ es para M
en el vac\'io.

En la implementaci\'on mecanocu\'antica usual del modelo de 
solvataci\'on continuo, se permite que cambien la funci\'on de onda
electr\'onica y la densidad de probabilidad electr\'onica de la 
mol\'ecula de soluto M al pasar de fase gaseosa a fase disoluci\'on,
de forma que se alcance la autoconsistencia entre la distribuci\'on
de carga M y el campo de reacci\'on del disolvente. Cualquier 
tratamiento en el que se alcance la autoconsistencia, se denomina
modelo de {\it campo de reacci\'on autoconsistente}.

\paragraph{El m\'etodo PCM} Los c\'alculos ab initio precisos de los 
efectos del disolvente requieren el uso de formas moleculares m\'as 
realistas que las formas esf\'ericas o elipsoidales. En el {\it 
modelo del continuo polarizable} (PCM) de Miertus, Scrocco y Tomasi, 
cada n\'ucleo at\'omico de la mol\'ecula de soluto, M, est\'a rodeado
por una esfera de radio 1.2 veces el radio de van der Waals del 
\'atomo. La regi\'on de la cavidad se toma como el volumen ocupado 
por esas esferas at\'omicas solapantes.

\paragraph{ASC} Ya que la cavidad PCM tiene una forma compleja, no se
pueden obtener expresiones anal\'iticas para el coeficiente de la
expansi\'on multipolar, y el m\'etodo de expansi\'on multipolar 
anal\'itico no es factible. En su lugar se usa un m\'etodo num\'erico
para obtener el t\'ermino de la energ\'ia potencial de interacci\'on
soluto-disolvente, $\hat V_{int}$. Se puede demostrar a partir de la
electrost\'atica cl\'asica, que el potencial el\'ectrico 
$\Rho_{\sigma}$ producido por el continuo diel\'ectrico polarizado es
igual al potencial de una {\it carga superficial aparente} (ASC) 
distribuida sobre la superficie de la cavidad molecular. La ASC es 
una distribuci\'on continua de carga caracterizada por la densidad 
de carga superficial (carga por unidad de \'area de la superficie) 
que var\'ia de un punto a otro de la superficie de la cavidad. En la
pr\'actica, esta continua ASC se aproxima reemplazandola por muchas
cargas puntuales, y la carga aparente $Q_k$ se situa en la regi\'on
$k-$\'esima. Si $\vec r_k$ es el punto en el que est\'a localizada
$Q_k$, entonces el potencial el\'ectrico $\phi_{\sigma}(\vec r)$ 
debido a la polarizaci\'on  del diel\'ectrico es (en unidades 
at\'omicas)
\be
\phi_{\sigma}(\vec r)=\sum_k\frac{Q_k}{|\vec r-\vec r_k|}.
\ee

% \paragraph{Ecuaciones PCM}
% La electrost\'atica cl\'asica de las siguientes expresiones para 
% las cargas aparentes:
% $$Q_k=[(\epslon_r-1)/4\pi\epsilon_r]A_k\nabla\phi_{int}(\vec r_k)
% \cdot \vec n_k$$
% donde $A_k$ es el \'area de la regi\'on $k-$\'esima, $\vec r_k$ es
% el punto en el que $Q_k$ est\'a localizada, $\nabla\phi_{int}
% (\vec r_k)$ es el gradiente del potencial el\'ectrico con la 
% cavidad evaluada en el l\'imite cuando el punto $\vec r_k$ se 
% aproxima, y $\vec n_k$ es un vector unitario perpendicular a la
% superficie de la cavidad en $\vec r_k$ y apuntando hacia afuera de
% la cavidad. [El gradiente de $\phi$ es dicintinuo en la superficie
% de la cavidad, de forma que es necesario distinguir 
% \nabla\phi_{int} de \nabla\phi_{ex}.] El potencial el\'ectrico en 
% la cavidad es la suma de la contribuci\'on $\phi_{M,int}$ de la
% distribuci\'on de carga de la mol\'ecula de soluto M y de la
% contribuci\'on $\phi_{\sigma,int}$ de diel\'ectrico polarizado:
% $\phi_{int}=\phi_{M,int}+\phi_{\sigma,int}$.

% \paragraph{Receta PCM} 
% Ya que ni $\phi_{int}$ ni $Q_k$ se conocen inicialmente, las cargas
% superficiales aparentes se obtienen mediante el siguiente proceso
% iterativo. Inicialmente, se desprecia $\phi_{\sigma,int}$, y se 
% toma la estimaci\'on inicial de $\phi_{int}$ como 
% $\phi^{(00)}_{int}$=$\phi^{(0)}_{M,int}$, donde 
% $\phi^{(0)}_{M,int}$ se calcula a partir de la densidad 
% electr\'onica $\rho^{(0)}$ de la mol\'ecula M en el vac\'io. 
% [$rho^{(0)}$ se obtiene a partir de la funci\'on de onda, o, en 
% DFT, a partir de los orbitales de Kohn-Sham de M en el vac\'io.
% N\'otese que $\phi^{(0)}_{M}$ es el potencial electrost\'atico
% molecular de la ecuaci\'on
% $$\phi(x_1,y_1,z_1)=\Sum_{\alpha}\frac{Z_{\alpha}e}{4\pi
% \epsilon_0r_{1\alpha}}-e\int\int\int\frac{\rho(x_2,y_2,z_3)}{4\pi
% \epsilon_0r_{12}}dx_2dy_2dz_2].$$ Entonces, se usa la \'ultima
% ecuaci\'on para obtener las estimaciones iniciales de 
% $Q^{(00)}_{k}$ del ASC, que se usa en (15.147) para encontrar una
% estimaci\'on  inicial $\phi^{(00)}_{\sigma,int}$ del potencial 
% el\'ectrico producido por el diel\'ectrico polarizado. El potencial
% mejorado $\phi^{(01)}_{int}=\phi^{(0)}_{M,int}+
% \phi^{(00)}_{\sigma,int}$, se usa en (15.148) para obtener cargas
% mejoradas $Q^{(01)}_k$, que se usan en (15.147) para obtener 
% $\phi^{(01)}_{\sigma,int}$, y as\'i sucesivamente. Se contin\'ua
% iterando hasta que las cargas convergen en los valores 
% $Q^{(0f)_k}$.

\paragraph{Iteraci\'on en PCM} Los valores de las cargas que han
alcanzado la convergencia se usan para obtener una estimaci\'on 
inicial del $\hat V_{int}$ como
$$\hat V^{(0)}_{int}=-\sum_i\phi^{(0f)}_{\sigma}(\vec r_i)+
\sum_{\alpha}Z_{\alpha}\phi^{(0f)}_{\sigma}(\vec r_{\alpha})$$ 
donde las sumas se extienden a todos los electrones y n\'ucleos, y
$\phi^{(0f)}_{\sigma}$ se obtiene a partir de (15.147), usando los
$Q^{(0f)}_k$. Se a\~nade $\hat V^{(0)}_{int}$ al Hamiltoniano 
electr\'onico molecular, que se usa para obtener una densidad 
electr\'onica mejorada $\rho^{(1)}$ para M, que da 
$\phi^{(1)}_{M,int}$, que a su vez da el potencial mejorado
$\phi^{(10)}_{int}=\phi^{(1)}_{M,int}+\phi^{(0f)}_{\sigma,int}$, que
se usa en (15.148) para comenzar un nuevo ciclo de iteraciones de 
carga y de $\phi_{\sigma}$. Se contin\'ua hasta que todo alcanza la
convergencia.

%AngyanJMathChem1992.10.93.pdf
Muchos modelos semiemp\'iricos y ab initio han sido propuestos en los
\'ultimos 45 a\~nos. Estos modelos a veces parecen ser diferentes 
aunque es ampliamente aceptado que el modelo f\'isico b\'asico en 
todos los casos es la idea del campo de reacci\'on, relacionada con 
el nombre de Onsager. Este modelo supone que el sistema del soluto 
polar, el cual est\'a embebido en el solvente, polariza sus 
alrededores. El ambiente polarizado crea un campo, el campo de 
reacci\'on (o de polarizaci\'on), el cual actua sobre el subsistema
soluto.
