%Informaci\'on de Levine. Qu\'imica Cuantica 2001 15.22
% A causa del momento dipolar adicional inducido por el campo de
% reacci\'on del disolvente, una mol\'ecula polar tendr\'a un momento
% dipolar mayor en un disolvente polar que en fase gaseosa ?`ocurre
% esto con las mol\'eculas de agua con las que trabajo? ?`Qu\'e 
% efecto tendr\'a el momento dipolar en las distancias y las 
% energ\'ias de hidrataci\'on?
% ?`C\'omo cambia la el momento el\'ectrico dipolar medio en una 
% simulaci\'on monte-carlo cuando se toma en cuenta los efectos del
% bulto y cuando no son tomados?

La v\'ia rigurosa para tratar los efectos del disolvente sobre las
propiedades moleculares es llevar a cabo c\'alculos mecanocu\'anticos
en sistemas que constan de una mol\'ecula de soluto redeada de muchas
mol\'eculas de disolvente, y tomar un adecuado promedio sobre las 
orientaciones para obtener propiedades promedio a una temperatura y
presi\'on particulares. Un c\'alculo de este tipo es, usualmente,
impracticable.

La mec\'anica cu\'antica del estado estacionario de una mol\'ecula
aislada es adecuada para el c\'alculo de las propiedades de la 
mol\'ecula en fase gaseosa que no se encuentre a una presi\'on 
elevada. Sin embargo, la mayor parte de la qu\'imica y la 
bioqu\'imica tiene lugar en disoluci\'on, y el disolvente puede 
tener un efecto mayor en la posici\'on de equilibrio qu\'imico y las
velocidades de reacci\'on. 

En una disoluci\'on diluida de una mol\'ecula polar de un soluto M en
un disolvente polar S, las mol\'eculas del solvente tender\'an a 
estar orientadas con sus \'atomos cargados positivamente dirigido al
los \'atomos del soluto M cargados negativamente, mientras las 
mol\'eculas del solvente tender\'an a estar orientadas con sus 
\'atomos cargados negativamente dirigidos a los \'atomos cargados 
positivamente del soluto S. Adem\'as, el momento dipolar de la 
mol\'ecula del soluto inducir\'a un momento dipolar en las 
proximidades de cada mol\'ecula del disolvente que a\~nade un momento
dipolar permanente. El resultados neto de estos efectos de 
orientaci\'on e inducci\'on es que el disolvente adquiere una 
polarizaci\'on global en la regi\'on de cada mol\'ecua del soluto. El
disolvente polarizado genera un campo el\'ectrico, llamado campo de
reacci\'on, para cada mol\'ecula de soluto. El campo de reacci\'on
distorciona la funci\'on de onda molecular del soluto con respecto a
la correspondiente en fase gaseosa, produciendo, por tanto, un 
momento dipolar inducido que se suma al momento dipolar permanente en
la fase gaseosa M. El momento dipolar aumentado de M polariza 
posteriormente al disolvente; y as\'i sucesivamente.

La funci\'on de onda molecular y todas las propiedades moleculares en
disoluci\'on diferir\'an en alguna extensi\'on de sus 
correspondientes e fase gaseosa. ({\it ?`C\'omo puedo ver esto en mis
resultados?})

La via rigurosa para tratar los efectos del disolvente sobre las 
propiedades moleculares es llevar a cabo c\'alculos mec\'anico 
cu\'anticos en sistemas que constan de una mol\'ecula de soluto 
rodeada de muchas mol\'eculas de disolvente, y tomar un adecuado 
promedio de las orientaciones para obtener las propiedades promedios
a una presi\'on y temperatura particulares. 

%TomasiChemRev1994.94.2027.pdf
Un apropiado entendimiento del fen\'omeno de reacciones qu\'imicas en 
soluci\'on es la finalidad \'ultima de los estudio te\'oricos en ese
campo; modelos y t\'ecnicas computacionales deber\'ian permitir la
predicci\'on de tasas, los mecanismos y otras caracter\'isticas de 
los procesos espec\'ificos que ocurren en soluci\'on. Estos objetivos
requieren la evaluaci\'on de muchas propiedades fisicoqu\'imicas, 
tanto de caracter local y macrosc\'opicas.

\paragraph{Clasificaci\'on ``old style''} La clasificaci\'on 
vieja de los modelos continuos es la siguiente: (a) m\'etodos basados
en la elaboraci\'on de funciones f\'isicas, (b) m\'etodos basados en 
la simulaci\'on en computadoras de l\'iquidos, (c) los m\'etodos de
modelos continuos electrost\'aticos y (d) los que hacen una 
decripci\'on de macromol\'ecula de la soluci\'on. 

\paragraph{Desarrollo hist\'orico} Los modelos continuos tienen su 
propio origen en consideraciones f\'isicas simples. La atenci\'on ha
sido puesta  desde el inicio en la descripci\'on microsc\'opica de 
una componenete del sistema (es decir, el soluto M). La expresi\'on
dada por Born \citep{Born19} y Bell \citep{Bell20}, para la energ\'ia
de interacci\'on cl\'asica de un soluto simple M con un medio 
representado como un continuo diel\'ectrico, fue formalmente 
extendido por Kirkwood \citep{Kirk21} en 1934 para descripci\'on
cu\'antica de M, sin l\'imites en la complejidad del sistema. La
contribuci\'on decisiva de Onsager \citep{Onsa22} en 1936 fue la que
proporcion\'o una herramienta interpretativa, usado por 
qu\'imicos por muchos a\~nos: la simplicidad de las expresiones 
formales exitosamente elaboradas estimul\'o la aplicaci\'on del 
efecto efecto de varios solventes.

% La formal derivaci\'on del modelo est\'a basado en el uso de las
% t\'ecnicas de partici\'on  y promedio de aplicaci\'on general en 
% mec\'anica molecular. El lector puede encontrar presentaciones
% formales detalladas en los art\'iculos de \'Angy\'an \citep{Angy81}
% y de Tapia \citep{Tapi82}: este \'ultimo completa una serie de
% procedimientos, menos detallados, formulaciones m\'as transparentes
% para los lectores no especializados {63, 83, 84}.

\paragraph{Modelo Cu\'antico}
El modelo continuo b\'asico involucra una forma simplificada de la
interacci\'on potencial; $\hat V_{int}$ es reducido a su componente
electrost\'atica cl\'asica, y $g_s$ describe un continuo isotr\'opico
lineal, caracterizado por la constante diel\'ectrica cl\'asica 
$\epsilon$ del bulto del solvente, la cual depende de la temperatura 
$T$. Se usar\'a el s\'imbolo $hat V_{\sigma}$ en lugar de 
$\hat V_{int}$, cuando solo los efectos de polarizaci\'on 
electrost\'atica ser\'an considerados:
$$\hat V_{int}=\hat V_{\sigma}(\vec r,\vec Q,\rho_M,\epsilon)=
\sum_{\alpha}Z_{\alpha}\Phi_{\alpha}(\vec Q_{\alpha})-\sum_i
\Phi_{\sigma}(\vec q_i),$$
donde $\vec Q_{\alpha}$ son las coordenadas de los nucleos y 
$\vec q_i$ son las coordenadas de los electrones. Aqu\'i 
$\Phi_{\sigma}(\vec r)$ es el valor del campo electrost\'atico 
generado por el diel\'ectrico polarizado en la posici\'on $vec r$. La
contribuci\'on de la interacci\'on soluto-solvente a la energ\'ia 
total $E^{(f)}$ es dada por la integral
$$W_{MS}=\int \Psi^{(f)*}\hat V_{\sigma}\Psi^{(f)}d\vec q_1...
d\vec q_{N_{el}}=\int\rho_M(\vec r)\Phi_{\sigma}(\vec r)dr^3$$
$\hat V_{\sigma}$ es un operador monoelectr\'onico, y la evaluaci\'on
$W_{MS}$ solo es un paso en el c\'alculo. El problema cu\'antico es
tratado m\'as o menos en las t\'ecnicas est\'andar. El m\'as 
frecuente m\'etodo usado es Hertree-Fock, en este caso 
$\hat V_{\sigma}$ es simplemente agregado el operador de Fock.
