\section{Medio Polarizable Continuo}
%TomasiChemRev205.105.2999
%\paragraph{Interacci\'on Soluto-Solvente}
La f\'isica de la interacci\'on electrost\'atica soluto-solvente es 
simple. La distribuci\'on de carga $\rho_M$ del soluto, dentro de la
cavidad, polariza el diel\'ectrico continuo, el cual a su vez 
polariza la distribuci\'on de carga del soluto. Esta definici\'on de
la interacci\'on corresponde a un proceso autoconsistente, el cual 
es num\'ericamente resuelto siguiendo un proceso iterativo. Es 
importante remarcar que el correspondiente potencial de interacci\'on
es el que se pondr\'a en el hamiltoniano modelo \ref{eSit} 
\citep{Toma2005}.

%\paragraph{Evoluci\'on del ASC}
La evoluci\'on del m\'etodo de superficie de carga aparente (ASC, por
sus siglas en ingl\'es) del PCM m\'as viejo al presente no es solo un
simple c\'odigo, sino un conjunto de c\'odigos, todos basados en la 
misma filosof\'ia y compartiendo muchas caracter\'isticas, algunos
especializados para prop\'ositos espec\'ificos, otros de uso general,
pero con diferencias que no merecen menci\'on \citep{Toma2005}.

%\paragraph{Evoluci\'on del PCM}
La versi\'on original de PCM fue publicada en 1981, despu\'es de 
algunos a\~nos de elaboraci\'on, fue implementada en varios paquetes
de mec\'anica cu\'antica computacional. M\'as recientemente, PCM fue
renombrados D-PCM (D se refiere a diel\'ectrico) para distinguirlo de
las dos siguientes reformulaciones (CPCM y IEFPCM). Este acr\'onimo 
no es completamente correcto como tampoco lo son las otras 
reformulaciones que se refieren a medios diel\'ectricos. 

%\paragraph{Limitaciones del PCM}
El DPCM, como todos los miembros de la familia PCM, permite describir
un n\'umero ilimitado de solutos, cada uno equipado con su propia 
cavidad y su ASC, interactuando a trav\'es del diel\'ectrico. En esta
forma, DPCM permite una extensi\'on del modelo b\'asico para el 
fen\'omeno asociaci\'on-disociaci\'on, molecular clustering, etc., y 
esto puede considerar un cambio continuo de una cavidad a dos 
cavidades o m\'as cavidades durante la asociaci\'on.

%\paragraph{C\'omo lo calcula Gaussian}
El procedimiento predeterminado calcula los efectos del solvente en 
la densidad SCF y entonces aplica teor\'ia de perturbaci\'on MP2, 
mientras los c\'alculos de la interacci\'on externa los efectos del
solvente autoconsistentemente con respecto a la densidad MP2.

%Tomassi2005
%\paragraph{Campo de reacci\'on}
Normalmente se hace referencia al campo de reacci\'on del solvente 
para el potencial de interacci\'on obtenidos con modelos continuos (y
tambi\'en para modelos que usan mol\'eculas de solvente 
expl\'icitas). Esta etiqueta tiene una raz\'on hist\'orica, la cual 
est\'a relacionada con el art\'iculo de Onsager en el cual el soluto
fue reducido a un dipolo puntual polarizable y la interacci\'on 
electrost\'atica entre el medio polarizable y el dipolo fue expresada
en t\'erminos de una campo electrost\'atico, teniendo origen en la 
polarizaci\'on del diel\'ectrico. Actualmente conviene hablar en 
t\'erminos de potencial de reacci\'on del solvente, pues un potencial
es el t\'ermino que se introduce en el hamiltoniano.

%KlamtCOSMO-RS2005ElsevierHolanda2005
%\paragraph{COSMO y COSMO-RS}
%Uno de los modelos de solvataci\'on continua de diel\'ectrico es el 
%desarrollado por Klamt y colaboradores en los inicios de 1990. 
%Despu\'es un tratamiento especial termodin\'amico de las 
%interacciones moleculares.

%\paragraph{Idea b\'asica y su desarrollo}
En 1920, Max Born, public\'o un trabajo sobre la energ\'ia de libre
de solvataci\'on de un ion, $\Delta G^{ion}_{S}$. \'El present\'o la 
idea de aproximar al solvente alrededor de un ion como un continuo
diel\'ectrico. Definiendo una frontera esf\'erica entre el ion y el
continuo mediante un radio efectivo del ion, $R^{ion}$, \'el obtuvo 
el resultado simple 
$$\Delta G^{ion}_{S}=-\frac{\epsilon_S-1}{\epsilon_S}\frac{Q^{ion^2}}
{2R^{ion}}.$$
Aqu\'i, se usa la convenci\'on de referirse al solvente por un 
sub\'indice y al soluto por un super\'indice; entonces $\epsilon_S$ 
es la constante diel\'ectrica del solvente y $Q^{ion}$ es la carga 
total del ion. Born, seguramente, estaba consiente de que esta es una
aproximaci\'on cruda, pero su f\'ormula da un entendimiento 
cualitativo de los valores de las energ\'ias de solvataci\'on 
observados experimentalmente. Despu\'es la idea fue tomada por 
Kirkwood y Onsager, quienes extendieron la descripci\'on de la 
solvataci\'on por un medio continuo diel\'ectrico tomando en cuenta 
los momentos dipolares el\'ectrico. Kirkwood deriv\'o la f\'ormula 
general:
$$\Delta G^{ion}_{S}=-\frac{1}{2}\sum^{\inf}_{l=1}f_l(\epsilon_l)
\frac{M^{X^2}_l}{R^{X^{2l-1}}}$$
con
$$f_l(\epsilon_S)=\frac{\epsilon-1}{\epsilon + x_l}$$
y 
$$x_l=\frac{l-1}{l}.$$
Siendo el primer t\'ermino el de la f\'ormula de Born y el segundo el
modelo de Onsager de la energ\'ia de solvataci\'on de una mol\'ecula 
dipolar
$$\Delta G^X_S=\frac{1}{2}\frac{\epsilon_S-1}{\epsilon_S+\frac{1}{2}}
\frac{\mu^{X^2}}{R^{X^3}}.$$

%\paragraph{El teorema de Poisson}
De acuerdo con la electrost\'atica b\'asica, la polarizaci\'on de un 
diel\'ectrico continuo homog\'eneo puede ser descrito por la densidad de
polarizaci\'on de volumen $\vec P(\vec r)$ o por la densidad de carga
de polarizaci\'on superficial $\sigma(\vec r)$. Considerando el 
diel\'ectrico continuo alrededor de un soluto infinitamente 
extendido, la densidad de carga de polarizaci\'on, $\sigma$, en la
superficie externa puede ser ignorada. Como resultado, solo la 
densidad de carga de polarizaci\'on, $\sigma$, en la interfase entre
las mol\'eculas del soluto y el medio diel\'ectrico es relevante para
la interacci\'on soluto-diel\'ectrico. As\'i, no se requiere un corte
y los errores del c\'alculo solo dependen del tama\~no de los 
peque\~nos segmentos de superficie que construyen la cavidad.

%\paragraph{Cavidades}
Despu\'es de asignar radios, la uni\'on de las correspondientes
esferas centradas en los \'atomos es considerada como el interior de
la mol\'ecula. Tal regi\'on tienen forma de grietas y c\'uspides, las
cuales son fatales para la descripci\'on de un diel\'ectrico 
continuo, ya que el campo el\'ectrico enfrente de una c\'uspide
diel\'ectrica se vuelve infinito y las condiciones a la frontera ya 
no se pueden resolver.
