\subsection*{Base de \cite{Yang2005}}
%\paragraph{Orbital de Cristal}
%Esta informaci\'on fue tomada de old.iupac.org
Orbital de cristal (sin\'onimo banda orbital), es la funci\'on de un
electr\'on extendida a trav\'es de un cristal. T\'ipicamente una 
combinaci\'on lineal de orbitales de Bloch:
$$\Psi_j=\sum_{\mu}a_{j\mu}\phi_{\mu}.$$
Los orbitales de Bloch incorporan la simetr\'ia de translaci\'on
aplicando condiciones a la frontera
$$\phi_{\mu}(\vec r)=\sum_n\exp{ikr}\chi_{\mu}(\vec r)$$
donde $\chi_{\mu}(\vec r)$ es el n-\'esimo orbital at\'omico definido 
el vector de translaci\'on $\vec r$. Los diferentes valores adoptados
por el vector de onda $\vec k$ determinan la simetr\'ia y las 
propiedades nodales del orbital de cristal.

%\paragraph{?`Porqu\'e orbitales de cristal}
%YangTheorChemAcc2005.113.212
Gracias a los recientes logros en la descripci\'on y predicci\'on las
propiedades de bulto de los materiales, los c\'alculos de las 
estructuras electr\'onicas se han vuelto m\'as importantes tanto en
qu\'imica como en f\'isica de materia condensada para dise\~nar y 
preparar materiales con propiedades controladas. El acceso a 
c\'odigos complicados y poder de c\'omputo ha hecho posible 
experimentos computacionales {\it ab initio}. Estos c\'alculos de 
primeros principios fueron hechos com\'unmente usando bases de ondas 
planas junto con Pseudopotenciales y teor\'ia del funcional de la 
densidad en f\'isica del estado s\'olido. Los m\'etodos de Orbitales 
de Cristales gan\'o atenci\'on en el c\'alculo de s\'olidos desde la
implementaci\'on de la combinaci\'on lineal de orbitales at\'omicos 
Hartree-Fock (LCAO) en el programa CRYSTAL para el tratamiento de 
sistemas peri\'odicos. Los m\'etodos de ECP han sido aplicados a los 
lant\'anidos principalmente para eliminar la capa abierta 4f y los
principales problemas relacionados a esta. %(?`Cu\'ales son esos problemas?). 
Sin embargo, los conjuntos de valencia generados para pseudopotenciales 
consistentes con la energ\'ia de los lant\'anidos para aplicaciones
moleculares no pueden ser transferidos directamente a componentes 
peri\'odicos sin modificaci\'on. La raz\'on es que, frecuentemente,
los conjunto de bases at\'omicas con funciones difusas dan como 
resultado un traslape entre funciones de Bloch que construyen los 
orbitales de cristal y as\'i no solo dan como resultado un 
desperdicio de recursos computacionales sino que adem\'as pueden 
causar problemas con dependencia cuasilineal debido a las 
limitaciones num\'ericas. 

%\paragraph{Resumen del trabajo de \cite{Yang2005}}
\cite{Yang2005} optimizaron un conjunto de bases Gaussianas a la 
energ\'ia HF para toda la serie de los iones de lant\'anidos
trivalentes (desde lantano hasta lutecio), adem\'as las derivaron y 
adaptaron para c\'alculos de Orbitales de Cristal usando los 
pseudopotenciales de Stuttgart-Koel de consistencia energ\'etica. 

%\ocultartrue
%\begin{Vocultar}
%\paragraph{Breve Teor\'ia de los OC}
%La naturaleza de los orbitales de cristal $\psi_i(r;k)$ en sistemas 
%peri\'odicos, los cuales son una combinaci\'on lineal de funciones de
%Bloch $\phi_{\mu}(r;k)$ que son contraidas a funciones gaussianas 
%(\ref{OC}). Cada funci\'on de Bloch es definida como la suma de los 
%mismos orbitales at\'omicos (OA,$\varphi_{\mu}(r-A_{\mu}-g)$) 
%trasladados sobre un enrejado infinito con cada OA multiplicado por 
%una onda plana (\ref{FB}), las cuales son expresadas como una 
%combinaci\'on lineal funciones gaussianas generadas (\ref{FG}). 
%As\'i, el conjunto de bases real son las funciones de Bloch debido a
%la raz\'on los coeficientes de combinaci\'on $a_{\mu,i}$ est\'an 
%involucrados en el procedimiento variasional y contribuyen en los
%c\'alculos de los elementos de la matriz densidad. Ya que el esquema
%es conservado, la libertad variasional es constante, y las funciones
%de Bloch no son sensibles al aumento en el tama\~no de las funciones
%primitivas de (4s4p3d) a (6s6p5d) debido al t\'ermino de las ondas
%planas $\exp{ik\cdot g}$, aunque un aumento en el n\'umero de 
%primitivas proporciona una mejor descripci\'on de OA.
%
%\be\label{OC}
%\psi_i(r;k)=\sum_{\mu}a_{\mu,i}(k)\phi_{\mu}(r;k)
%\ee
%\be\label{FB}
%\phi_{\mu}(r;k)=\sum_{g}\varphi_{\mu}(r-A_{\mu}-g)\exp{ik\cdot g}
%\ee
%\be\label{FG}
%\varphi_{\mu}(r-A_{\mu}-g)=\sum^{n_G}_{j}d_jG(\alpha_j;r-A_{\mu}-g).
%\ee
%\end{Vocultar}
