\section{Mol\'ecula de Agua}
El agua es probablemente la sustancia m\'as estudiada de todas, esto
posiblemente se debe a su relevancia como un importante constituyente
de los organismos vivos, de la tierra, e incluso del espacio 
interestelar, tambi\'en por sus propiedades \'unicas tanto 
microsc\'opicas como macrosc\'opicas. La estructura qu\'imica del 
agua es simple, sin embargo muestra propiedades inusuales. Unas de 
las m\'as conocidas de estas propiedades son que su densidad m\'axima
ocurre a los 4$^{\circ}$C, sus altos calores espec\'ifico, de 
fusi\'on y de vaporizaci\'on \citep{Mori1997}. A pesar de la vasta 
literatura acerca del tema, la descripci\'on de las diferencias de 
sus caracter\'isticas en diferentes fases a\'un no son resueltas 
\citep{Sain2000}. El desarrollo de t\'ecnicas experimental permite el
estudio de las propiedades de una solo mol\'ecula de agua, del 
d\'imero de agua y de otros agregados de aguas, esto permiti\'o la
medida m\'as precisa de las propiedades del agua l\'iquida. 

En el presente trabajo se usan mol\'eculas de agua de manera 
expl\'icita que est\'an en presencia de un ion que est\'a altamente 
cargado, por tal motivo, se est\'a interesado en conocer algunas
propiedades estructurales de la mol\'ecula de agua, tanto en fase gas
como en fase l\'iquida. El conociemiento de la mol\'ecula de agua 
permitir\'a describir la deformaci\'on de las mol\'eculas de la 
primera capa de hidrataci\'on debida al i\'on, adem\'as se podr\'a 
comparar las deformaciones provocadas en fase gas a las provocadas en
fase l\'iquida. En la tabla \ref{tsf1} se presentan las distancias de
los hidr\'ogenos al ox\'igenos de las mol\'eculas de aguas en 
distintos estados de agragaci\'on. 

Adem\'as de la informaci\'on de la tabla \ref{tsf1} otro valor de las
propiedades de la mol\'ecula de aguas que nos intetresa es el valor 
del momento dipolar. Se sabe que el valor del momento dipolar 
experimental de una mol\'ecula de agua incrementa al pasar del 
vac\'io a agua en soluci\'on, pasando de $\mu=1.85$ D 
\citep{Clou1973} a un valor de $\mu=2.90$ D \citep{Bady2000}. Estos
valores ayudar\'an a evaluar el desempe\~no del m\'etodo usado para
modelar los efectos del bulto utilizado en este trabajo.

\begin{table}[h!]
\centering
\caption{\footnotesize Propiedades experimental de la estructura de 
la mol\'ecula de agua en fase gas y en fase l\'iquida, tomadas de 
trabajo de \cite{Mori1997}}
\begin{tabular}{c|cc}\hline\hline
Fase      & O-H & $\angle$ HOH \\ \hline
Gas       & 0.957 & 104.52 $^{\circ}$ \\ 
L\'iquida & 0.970 & 106.6  $^{\circ}$ \\ 
\hline \end{tabular}\label{tsf1}\end{table}




La informaci\'on ser\'a tomada de los art\'iculos
\begin{itemize}
 \item Dunning, Jr., {\it J. Chem. Phys. 1989}, {\bf 90}, 1007.
 \item R. A. Kendall, R. J. Harrison and T. H. Dunning, Jr., {\it J.
 Chem. Phys.}, 1992, {\bf 96}, 6796.
 \item H. Saint-Martin, J. Hern\'andez-Cobos, M. I. Bernal-Uruchurtu,
 I. Ortega-Blake and H. J. C. Berendsen, {\it J. Chem. Phys.}, 2000,
 {\bf 113}, 10899.
\end{itemize}
\paragraph{?`Qu\'e pedo con aug-cc-pVDZ?}
