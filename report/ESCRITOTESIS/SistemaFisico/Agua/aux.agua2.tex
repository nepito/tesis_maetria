\section{Molécula de Agua}
El agua es probablemente la sustancia más estudiada de todas, esto
posiblemente se debe a su relevancia como un importante constituyente
de los organismos vivos, de la tierra, e incluso del espacio 
interestelar, también por sus propiedades únicas tanto 
microscópicas como macroscópicas. La estructura química del 
agua es simple, sin embargo muestra propiedades inusuales. Unas de 
las más conocidas de estas propiedades son que su densidad máxima
ocurre a los 4$^{\circ}$C, sus altos calores específico, de 
fusión y de vaporización \citep{Mori1997}. A pesar de la vasta 
literatura acerca del tema, la descripción de las diferencias de 
sus características en diferentes fases aún no son resueltas 
\citep{Sain2000}. El desarrollo de técnicas experimental permite el
estudio de las propiedades de una solo molécula de agua, del 
dímero de agua y de otros agregados de aguas, esto permitió la
medida más precisa de las propiedades del agua líquida. 

En el presente trabajo se usan moléculas de agua de manera 
explícita que están en presencia de un ion que está altamente 
cargado, por tal motivo, se está interesado en conocer algunas
propiedades estructurales de la molécula de agua, tanto en fase gas
como en fase líquida. El conocimiento de la molécula de agua 
permitirá describir la deformación de las moléculas de la 
primera capa de hidratación debida al ion, además se podrá 
comparar las deformaciones provocadas en fase gas a las provocadas en
fase líquida. En la tabla \ref{tsf1} se presentan las distancias de
los hidrógenos al oxígenos de las moléculas de aguas en 
distintos estados de agregación. 

Además de la información de la tabla \ref{tsf1} otro valor de las
propiedades de la molécula de aguas que nos interesa es el valor 
del momento di polar. Se sabe que el valor del momento dipolar 
experimental de una molécula de agua incrementa al pasar del 
vacío a agua en solución, pasando de $\mu=1.85$ D 
\citep{Clou1973} a un valor de $\mu=2.90$ D \citep{Bady2000}. Estos
valores ayudarán a evaluar el desempeño del método usado para
modelar los efectos del bulto utilizado en este trabajo.

\begin{table}[h!]
\centering
\caption{\footnotesize Propiedades experimental de la estructura de 
la molécula de agua en fase gas y en fase líquida, tomadas de 
trabajo de \cite{Mori1997}}
\begin{tabular}{c|cc}\hline\hline
Fase      & O-H & $\angle$ HOH \\ \hline
Gas       & 0.957 & 104.52 $^{\circ}$ \\ 
Líquida & 0.970 & 106.6  $^{\circ}$ \\ 
\hline \end{tabular}\label{tsf1}\end{table}




La información será tomada de los artículos
\begin{itemize}
 \item Dunning, Jr., {\it J. Chem. Phys. 1989}, {\bf 90}, 1007.
 \item R. A. Kendall, R. J. Harrison and T. H. Dunning, Jr., {\it J.
 Chem. Phys.}, 1992, {\bf 96}, 6796.
 \item H. Saint-Martin, J. Hernández-Cobos, M. I. Bernal-Uruchurtu,
 I. Ortega-Blake and H. J. C. Berendsen, {\it J. Chem. Phys.}, 2000,
 {\bf 113}, 10899.
\end{itemize}
\paragraph{?`Qué pedo con aug-cc-pVDZ?}
