\section{Hidrataci\'on de Iones}
%OhtakiChemRev1993.93.1157
%\paragraph{?`Hidrataci\'on i\'onica?}
La hidrataci\'on i\'onica es uno
de los temas m\'as atractivos de la qu\'imica, especialmente para 
aquellos quienes est\'an interesados en reacciones en soluci\'on 
acuosa, y numerosos estudios de hidrataci\'on i\'onica, es decir, la 
determinaci\'on de los n\'umeros de hidrataci\'on de los iones, tasas
de intercambio de mol\'eculas de aguas coordinadas alrededor del ion,
y las energ\'ias de interacci\'on entre iones y mol\'eculas de agua 
han sido hechos desde Arrhenius \citep{Ohta1993}. El concepto de 
iones hidratados depende en gran manera en los m\'etodos de 
observaci\'on. Se puede clasificar la estructura de iones hidratados 
en tres categor\'ias dependiendo de los m\'etodos de investigaci\'on.

%\paragraph{Estructura espectrosc\'opica} 
Consideraciones 
energ\'eticas resultan en la discriminaci\'on de mol\'eculas de agua
combinadas fuertemente con iones de la aproximada interacci\'on de 
mol\'eculas con iones. Investigaciones espectrosc\'opicas, incluyendo
medidas de frecuencias Raman y espectroscopias IR y estudios 
termodin\'amicos en la hidrataci\'on de iones proporcionan 
informaci\'on de mol\'eculas de agua fuertemente coordinadas a los 
iones.

%\paragraph{M\'etodos dispersivos} 
Los m\'etodos de dispersi\'on pueden ser separados en dos tipos: uno 
es la medida de intensidades de electrones, fotones y neutrones 
el\'asticamente dispersados por \'atomos y el otro es el m\'etodo de 
dispersi\'on inel\'astica de neutrones; y entre ellos est\'an los 
m\'etodos de difracci\'on de rayos X, difracci\'on de neutrones, 
difracci\'on de electrones, difracci\'on de rayos X en \'angulos 
peque\~nos y difracci\'on de neutrones en \'angulos peque\~nos y la 
dispersi\'on cuasi-el\'astica de neutrones. Los resultados obtenidos 
por los m\'etodos de difracci\'on de neutrones y de rayos X son 
incluidos en la categor\'ia de la estructura est\'atica, en la cual 
las estructuras discutidas son promediadas en el tiempo y el espacio 
de las interacciones del ion-agua. 

%\paragraph{M\'etodos espectrogr\'aficos} 
Varios m\'etodos 
espectrogr\'afico han sido empleados en el estudio de hidrataci\'on
i\'onica. Entre ellos est\'an los de absorci\'on de extendida de 
rayos X en estructura fina (EXAFS) y Absorci\'on de rayos X cerca del
borde de la estructura (XANES), ambos m\'etodos est\'an conectados a
la absorci\'on de rayos X por \'atomos los cuales son afectados por 
sus estructuras ambientales. Tambi\'en est\'a la espectroscopia por
resonancia nuclear magn\'etica (NMR), que es muy \'util para 
determinar las estructuras est\'aticas y din\'amicas de iones 
hidratados. Tambi\'en la espectroscopias M\"ossbauer, Infrarroja, 
Raman y Raleigh-Brillouin se agrupan en los m\'etodos 
espectrogr\'aficos. Resultados obtenidos mediante medidas de NMR son 
el caso t\'ipico, de estructuras discutidas con base en las 
propiedades din\'amicas de las mol\'eculas de agua coordinadas. 

%\paragraph{Simulaciones computacionales} 
Las simulaciones por 
computadora son descripciones te\'oricas para obtener informaci\'on
de la estructura y din\'amica de las mol\'eculas y conjunto de ellas,
en fase condensada como en fase gaseosa. Las simulaciones pueden ser
clasificadas en tres categor\'ias: din\'amica molecular (MD), 
simulaciones Monte Carlo (MC) y c\'alculos de mec\'anica molecular 
(MM). La primera puede describir la estructura y las propiedades 
din\'amicas de un sistema, ya que en las din\'amicas moleculares 
existe el tratamiento del tiempo, mientras que en las otras dos no 
hay dependencia en el tiempo de los par\'ametros en el desarrollo de 
los c\'alculo y as\'i solo dan informaci\'on estructural o 
energ\'etica.

%\paragraph{Y$^{3+}$ y los iones Ln$^{3+}$} El ion de Ytrium (III) es
%casi siempre comparado con los iones de lant\'anidos (III) en sus
%reacciones y estructuras en soluci\'on por que tienen similar carga
%i\'onica y tama\~no. 

%\paragraph{Trabajo de D`Angelo}
