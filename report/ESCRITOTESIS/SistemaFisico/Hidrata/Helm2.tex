%Informacion tomada de JChemSoc2002.6.633
En recientes a\~nos la aplicaci\'on de din\'amica molecular cl\'asica
(MD, por sus siglas en ingl\'es) a simulaciones computacionales, 
as\'i como los c\'alculos {\it ab initio} y del funcional de la 
densidad (DFT) ganaron m\'as inter\'es. La simulaci\'on cl\'asica DM
de hidrataci\'on de iones met\'alicos en soluci\'on principalmente
depende en par aditivos, los potenciales de interacci\'on efectiva
obtenidos de c\'alculos mac\'anicos cu\'anticos o de optimizaci\'on
emp\'irica de valores termodin\'amicos como las energ\'ias de 
hidrataci\'on o la funci\'on de distribuci\'on radial. Los l\'imites
de este m\'etodo  a iones met\'alicos  y iones de metales de 
transici\'on, donde los efectos del campo ligando pueden ser 
ignorados (por ejemplo, lant\'anidos). La ventaja de los m\'etodos 
cl\'asicos que sistemas relativamente largos, conteniendo m\'as de
10$^3$ mol\'eculas de agua pueden ser simulados  por varios ns 
(10$^{-9}$ s). Los procesos de intercambio pueden ser seguidos 
directamente inspeccionando trayectorias, coordenadas moleculares y
velocidades en funci\'on del tiempo.

El recambio de una mol\'ecula de agua  de su primera esfera de 
hidrataci\'on es un paso importante en la formaci\'on de complejos
reacciones de cationes mat\'alicos tambi\'en como en muchos procesos
redox. La qu\'imica general de iones agua sobre la tabla peri\'odica
de los elementos, su s\'intesis, estructura y reactividad los 
revis\'o Richens los aspectos te\'oricos y experimentales del 
intercambio del agua han se discutieron de manera extensiva. Los
cationes met\'alicos son clasificados  en tres grupos. El primer
grupo est\'a representado por el grupo principal de iones 
met\'alicos, variando principalmente en el radio i\'onico y la carga
el\'ectrica. El n\'umero de aguas en la primera esfera de 
coordinaci\'on abarca de 4 a 10 o mayor. El segundo grupo consiste en
los iones de los metales d-transici\'on, los cuales son 
principalmente hexacoordinados con algunas excepciones como Pd$^{2+}$
y Pt$^{2+}$ con un cuadro planar y Sc$^{3+}$ que puede ser 
heptacoordinado. Sus constantes de tasas de intercambio est\'an 
relacionadas con la ocupaci\'on del orbital d. El tercer grupo
involucra los iones trivalentes de los lant\'anidos y los act\'inidos
los cuales tienen un numero de coordinaci\'on de 8 o 9. Su 
comportamiento cin\'etico es principalmente influenciado por el 
acortamiento de su radio i\'onico a lo largo de las series y 
subsecuentemente cambia el n\'umero de coordinaci\'on.
