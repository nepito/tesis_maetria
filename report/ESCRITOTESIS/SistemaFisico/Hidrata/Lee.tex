Una obrsevaci\'on detallada a los datos de obsorci\'on de rayos x
mostr\'o que los datos fluctuan, esas fluctuaciones son probocadas 
por los at\'omos alrededor del \'atomo absorbente. El an\'alisis de
esas fluctuaciones puede permitir deducir el arreglo de los \'atomos
vecimos al \'atomo absorbente \citep{Toni1989}. 

%Informacion sacada de LeeRevModPhys1981.53.769
En los a\~nos setentas del siglo pasado, el trabajo de Sayer 
{\it et al.} aumento el interes en los m\'etodos EXAFS gracias a la
posibilidad de obtener las distancias $r_i$ a partir de un an\'alisis
de los datos EXAFS. 

EXAFS permite la determinaci\'on de la estructura local de cada 
especie at\'omica individual. Y se pens\'o en aplicaciones en la
estructura de mol\'eculas bil\'ogicas complicadas, de aleaciones y 
materiales amorfos, soluciones qu\'imicas, cat\'alisis, etc. Una de 
las desventajas del m\'etodo EXAFS es la necesidad de conocer 
algunos par\'ametros, tales como la dependencia en $k$ del cambio de
fase, la amplitud de la dispersi\'on de regreso, los factores de 
p\'erdida inel\'astica y de Debye-Waller.
