\section{Mecanismo de Intercambio}
%HelmChemRev2005.105.1923
%\paragraph{Clasificaci\'on del mecanismo}
\cite{Helm2005} dividen la reacci\'on de sustituci\'on de ligando en 
tres categor\'ias para poder clasificarlas: asociativa ({\bf A}), 
donde un incremento intermedio en el n\'umero de coordinaci\'on puede
ser detectado; disociativa ({\bf D}), donde una disminuci\'on 
intermedia en el n\'umero de coordinaci\'on puede ser detectada; y de
intercambio ({\bf I}), donde no hay no hay estructura intermedia 
detectable \citep{Helm2005}. 

%\paragraph{La serie de lant\'anidos}
Como ya se dijo, dos de las caracter\'isticas de las configuraciones 
electr\'onicas de los lant\'anidos tienen relevancia en este trabajo:
las capas externas de los lant\'anidos son muy similares entre los 
elementos de la serie y su capa interna 4f es la que presenta mayor 
diferencia entre los lant\'anidos, est\'as caracter\'isticas se
acent\'ua m\'as en los iones triplemente cargados. La primera
caracter\'istica provoca que los quince iones de los lant\'anidos 
trivalentes representan una serie de iones met\'alicos qu\'imicamente 
similares. Por otro lado, el llenado de los orbitales 4f, desde 
La$^{3+}$ hasta Lu$^{3+}$ es acompa\~nado por una reducci\'on del 
radio at\'omico. El cambio en el n\'umero de coordinaci\'on, de nueve 
para los lant\'anidos ligeros a ocho para los pesados, es una 
consecuencia de la contracci\'on de los lant\'anidos y tiene 
consecuencia importantes en el intercambio cin\'etico de agua de la 
primera esfera de hidrataci\'on de los iones de los lant\'anidos y el 
bulto de la soluci\'on. La coordinaci\'on de ocho y nueve de los 
Ln$^{3+}$ en soluci\'on acuosa adoptan las geometr\'ias antiprisma 
cuadrado (SAP) y el prisma tri capa triangular (TTP), respectivamente.
La geometr\'ia de la primera esfera de hidrataci\'on es principalmente
determinada por la atracci\'on electrost\'atica de los iones 
fuertemente cargados en un solvente de mol\'eculas polares y por la 
repulsi\'on electrost\'atica y est\'erica entre las mol\'eculas del 
solvente. 

