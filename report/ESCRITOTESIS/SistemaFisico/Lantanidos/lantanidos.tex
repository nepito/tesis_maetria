\section{Lant\'anidos}
%{\it ?`Qu\'e vergas son los lant\'anidos}
%\paragraph{Breve historia de los lant\'anidos}
Los minerales de los lant\'anidos se analizaron a mediados del siglo
XIX. Por ejemplo, el qu\'imico finland\'es Johan Gadolin, en 1794,
descubri\'o un nuevo \'oxido met\'alico (o {\it tierra}) en un 
mineral obtenido de la cantera de Ytterby, cerca de Estocolmo, 
Suecia. Como la nueva tierra era mucho menos com\'un que otras como 
la s\'ilice, caliza y magnesia, se le dio el nombre de {\it tierra 
rara}. Quince a\~nos despu\'es se obtuvo el elemento iterbio (Yb).
El qu\'imico sueco Carl Gustav Mosander descubri\'o las tierras
raras lantano (La), erbio (Er), terbio (Tb) y didimio\footnote
{Cuarenta a\~nos despu\'es, en 1885, el qu\'imico austriaco 
Carl Auer encontr\'o que el didimio era una mezcla de dos elementos, 
praseodimio (Pr) y neodimio (Nd).}. Lecoq de Boisbaudran descubri\'o 
el samario (Sm), en 1879, y el disprocio (Dy) en 1886. En 1878, Cleve
descubri\'o otros dos: holmio (Ho) y tulio (Tm). Hacia 1907, cuando 
el qu\'imico franc\'es Georges Urbain descubri\'o la tierra rara 
lutecio (Lu), se hab\'ia descubierto ya un total de catorce de tales
elementos \citep{Asim2010}.


%\paragraph{Configuraci\'on Electr\'onica}
Los lant\'anidos se caracterizan por la ocupaci\'on gradual del 
subnivel $4f$. Las energ\'ias relativas de los orbitales $nd$ y
$(n-1)f$ son muy similares y sensibles a la ocupaci\'on de dichos
orbitales por lo cual, la configuraci\'on electr\'onica de los 
\'atomos neutros muestran ciertas irregularidades, las 
configuraciones de los \'atomos de lantano, cerio y gadolinio colocan
un electr\'on en el orbital 5d sin haber llenado la capa 4f, como 
sugiere el diagrama de Moeller. La tabla \ref{tL1} muestra las 
configuraciones electr\'onicas de todos los elementos de la serie de 
los lant\'anidos.

\linespread{1.0}
\begin{table}[h!]
\centering
\caption{\footnotesize Configuraciones electr\'onicas de los 
lant\'anidos, $[$Xe$]$ corresponde a la configuraci\'on electr\'onica
del Xen\'on 1s$^2$2s$^2$2p$^6$3s$^2$3p$^6$4s$^2$3d$^{10}$4p$^6$5s$^2$
4d$^{10}$5p$^6$.}
\begin{tabular}{c|cc}\hline\hline
Ln & Gas Noble & Subconfiguraci\'on \\ \hline
La & $[$Xe$]$ & 5d$^1$6s$^2$ \\ 
Ce & $[$Xe$]$ & 4f$^1$5d$^1$6s$^2$  \\ 
Pr & $[$Xe$]$ & 4f$^3$6s$^2$ \\ 
Nd & $[$Xe$]$ & 4f$^4$6s$^2$ \\ 
Pm & $[$Xe$]$ & 4f$^5$6s$^2$ \\ 
Sm & $[$Xe$]$ & 4f$^6$6s$^2$ \\ 
Eu & $[$Xe$]$ & 4f$^7$6s$^2$ \\ 
Gd & $[$Xe$]$ & 4f$^7$5d$^1$6s$^2$  \\ 
Tb & $[$Xe$]$ & 4f$^9$6s$^2$ \\ 
Dy & $[$Xe$]$ & 4f$^{10}$6s$^2$ \\ 
Ho & $[$Xe$]$ & 4f$^{11}$6s$^2$ \\ 
Er & $[$Xe$]$ & 4f$^{12}$6s$^2$ \\ 
Tm & $[$Xe$]$ & 4f$^{13}$6s$^2$ \\ 
Yb & $[$Xe$]$ & 4f$^{14}$6s$^2$ \\ 
Lu & $[$Xe$]$ & 4f$^{14}$5d$^1$6s$^2$  \\ 
\hline \end{tabular}\label{tL1}\end{table}
\linespread{1.75}

%DolgTheorActa1989.pdf
Desde el punto de vista qu\'imico los elementos de las tierras raras
son usualmente trivalentes, menos frecuentes bivalentes o 
tetravalentes. Tanto Eu$^{2+}$ e Yb$^{2+}$ son las especies con carga
2+ m\'as estables.  Estos iones est\'an estabilizados por las 
configuraciones $4f^7$ y $4f^{14}$ que gozan de especial estabilidad 
de los subniveles llenos y semillenos (provenientes de la energ\'ia 
de intercambio). Otros lant\'anidos forman compuestos Ln(II), los 
cuales son estables como s\'olidos y hay pruebas de que es posible 
hacer que todos los lant\'anidos formen cationes divalentes estables
en una red cristalina que los contenga \citep{Huhe1981}. Los estados 
de oxidaci\'on superiores son poco frecuentes en los lant\'anidos. El
cerio forma una especie estable 4+, la cual es un agente oxidante muy
fuerte en soluci\'on acuosa aunque la reacci\'on con el agua es lo 
suficientemente lenta como para permitir su existencia 
\citep{Huhe1981}; terbio tambi\'en presenta un estado de oxidaci\'on
4+ estable. En general, las subconfiguraciones correspondientes a los
diferentes iones de los lant\'anidos en mol\'eculas $4f^n$ 
($n=0-14$ para La-Lu), $4f^{n+1}$ ($n=0-13$, para La-Yb) o $4f^{n-1}$
($n=1-14$ para Ce-Lu), para iones trivalentes, divalentes y 
tetravalentes, respectivamente.

Las energ\'ias orbitales de los c\'alculos de todos los electrones HF
sugiere que los orbitales $4f$ est\'en en el espacio de valencia, sin 
embargo su extensi\'on espacial sugiere que ellos pueden ser 
agregados al carozo \citep{Dolg1989}. Por ejemplo, los valores de
expectaci\'on de la posici\'on, $\langle r \rangle$ de los orbitales
4f, para el \'atomo Ce y su ion Ce$^+$, son modificados menor de un 
1\% para diferentes configuraciones de valencia. Aunque los orbitales
4f forman una capa abierta ellos tienen un comportamiento de como de
carozo, y los orbitales 5d y 6s como de valencia, pues son 
responsables del comportamiento qu\'imico de los lant\'anidos. Los 
orbitales 4f casi no participan en los enlaces qu\'imicos de los 
componentes de las tierras raras, este comportamiento se acent\'ua en
los \'ultimos elementos de la serie de los lant\'anidos. Este 
comportamiento fue aprovechado por \cite{Dolg1989} para proponer
pseudopotenciales que permitieran c\'alculos cu\'anticos m\'as 
r\'apidos en los que participan los lant\'anidos.

