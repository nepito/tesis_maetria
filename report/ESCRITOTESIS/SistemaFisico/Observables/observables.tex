\section*{Observables}
\subsection*{Observables Experimentales}
Experimentalmente, la investigaci\'on de la solvataci\'on no es una 
tarea trivial porque la qu\'imica de coordinaci\'on de los lant\'anidos
muestra una gran diversidad estructural debido a la ausencia del 
efecto de un campo fuerte de fuerza y la presencia de un r\'apido 
intercambio cin\'etico. Esto resulta en peque\~nas diferencias 
energ\'eticas entre diferentes arreglos geom\'etricos y n\'umeros de
coordinaci\'on. Una de las pocas caracter\'isticas confiables de la
qu\'imica de coordinaci\'on de los lant\'anidos es el decremento del
radio i\'onico de los lant\'anidos trivalentes o de la distancia
interat\'omica lant\'anido-ligando con el incremento del n\'umero
at\'omico, la llamada {\it contracci\'on lant\'anida} 
\citep{Shar2013}. 

El tama\~no, la estructura y la din\'amica de las esferas de 
hidrataci\'on alrededor de los iones son los factores m\'as 
fundamentales para determinar sus propiedades qu\'imicas y f\'isicas
en soluci\'on acuosa \citep{Shar2013}. Para interpretar los 
experimentos el conocimiento de las caracter\'isticas b\'asicas de la
estructura de hidrataci\'on, es decir, las distancias Ln-OH$_2$ y el 
n\'umero de coordinaci\'on, es obligado \citep{Dang2012}.  
Los datos te\'oricos que se obtienen en el presente trabajo son las 
distancias promedio de las mol\'eculas de agua al ion y el n\'umero 
de coordinaci\'on de la primera esfera de hidrataci\'on, con la 
intenci\'on de validar la metodolog\'ia y poder proporcionar 
informaci\'on te\'orica en aquellos casos en los cuales la 
informaci\'on experimental no sea tan confiable, abundante o 
existente. Los datos experimentales de las distancias promedio 
Ln-OH$_2$ reportadas en el trabajo de \cite{Dang2012} y los valores
te\'oricos m\'as cercanos a los experimentales por ellos citados son
presentados en la tabla \ref{tDTyE}.
\linespread{1.0}
\begin{table}[h!]
\centering
\caption{\footnotesize Resultados expetimentales y te\'oricos de la
distancia promedio lant\'anidos ox\'igeno. Los c\'alculos a nivel
MP2 son obtenidos por \cite{Ciup2010}. Los c\'alculos utilizando el 
DFT B3LYP son obtenidos por \cite{Kuta2010}.}%\label{t1}
\begin{tabular}{c|ccc}\hline\hline
Ln & MP2 & B3LYP & EXAFS \\ \hline
La$^{\mbox{{\tiny III}}}$  & 2.561 & 2.58 & 2.600 \\ 
                   & 2.591 & 2.62 &       \\ 
Ce$^{\mbox{{\tiny III}}}$  & 2.540 & 2.55 & 2.570 \\ 
                   & 2.571 & 2.59 &       \\ 
Pr$^{\mbox{{\tiny III}}}$  & 2.520 & 2.53 & 2.550 \\ 
                   & 2.551 & 2.57 &       \\ 
Nd$^{\mbox{{\tiny III}}}$  & 2.503 & 2.51 & 2.525 \\ 
                   & 2.535 & 2.55 &       \\ 
Pm$^{\mbox{{\tiny III}}}$  & 2.486 & 2.50 &       \\ 
                   & 2.518 & 2.54 &       \\ 
Sm$^{\mbox{{\tiny III}}}$  & 2.470 & 2.48 & 2.490 \\ 
                   & 2.503 & 2.52 &       \\ 
Eu$^{\mbox{{\tiny III}}}$  & 2.455 & 2.47 & 2.470 \\ 
                   & 2.488 & 2.51 &       \\ 
Gd$^{\mbox{{\tiny III}}}$  & 2.441 & 2.45 & 2.455 \\ 
                   & 2.475 & 2.50 &       \\ 
Tb$^{\mbox{{\tiny III}}}$  & 2.427 & 2.45 & 2.440 \\ 
                   & 2.462 & 2.50 &       \\ 
Dy$^{\mbox{{\tiny III}}}$  & 2.414 & 2.43 & 2.425 \\ 
                   & 2.449 &      &       \\ 
Ho$^{\mbox{{\tiny III}}}$  & 2.401 & 2.41 & 2.405 \\ 
                   & 2.436 &      &       \\ 
Er$^{\mbox{{\tiny III}}}$  & 2.389 & 2.40 & 2.390 \\ 
                   & 2.424 & 2.45 &       \\ 
Tm$^{\mbox{{\tiny III}}}$  & 2.378 & 2.39 & 2.375 \\ 
                   & 2.413 & 2.44 &       \\ 
Yb$^{\mbox{{\tiny III}}}$  & 2.366 & 2.38 & 2.360 \\ 
                   & 2.402 & 2.43 &       \\ 
Lu$^{\mbox{{\tiny III}}}$  & 2.359 & 2.37 & 2.345 \\ 
                   & 2.395 & 2.42 &       \\ 
\hline \end{tabular}\label{tDTyE}\end{table}
\linespread{1.75}


