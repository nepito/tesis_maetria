\section*{AB Initio}
% Esto es una traducción de la primera parte del capítulo sexto del
% libro "Essential Computer Chemestry" de Carter.
La suposici\'on fundamental de la teor\'ia Hartree-Fock, que cada 
electr\'on ve a todos los otros como un campo promedio, permite un
tremendo progreso ser hecho en hacer los c\'alculos de orbitales
moleculares (OM). Sin embargo, ignorar la correlaci\'on electr\'onica
tiene profundas consecuencias qu\'imicas cuanto esto viene a 
determinar funciones de ondas precisas y propiedades derivadas de 
ellas. La construcci\'on de teor\'ias semiemp\'iricas fue motivada en
parte por la esperanza de esfurzos en parametrizaciones juiciosas
podr\'ian compensar esta caracteristica de la teor\'ia HF. Mientras
tales compensaciones no tienen fundamentos rigurosos, para extender
esto nos permite hacer predicciones qu\'imicas precisas, esto puede
tener gran utilidad pr\'actica.

Los primeros constructores de la as\'i llamada teor\'ia HF {\it ab 
initio} (del lat\'in desde el principio), tendi\'o a ser menos 
enfocada en hacer predicciones de t\'erminos pequ\~nos, y m\'as 
enfocada en t\'erminos largos contrucci\'on de una metodolog\'ia
rigurosa que ser\'ia importante esperar (una tensi\'on din\'amica
entre la necesidad de hacer predicciones ahora y la necesidad de hacer
mejores predicciones ma\~nana es parecidamente por caraterizar la 
qu\'imica computacional en el futuro). Por supuesto, el rigor 
fundamental es la ecuaci\'on de Schr\"odinger, pero tal ecuaci\'on es
insoluble en una esencia pr\'actica para la mayor\'ia de los sistemas
simples. As\'i, la teor\'ia HF, a pesar de sus fallas en las 
suposiciones fundamentales, fue adoptada como \'util en la 
filosof\'ia {\it ab initio} porque proporciona un muy bien definida 
piedra de paso en el camino a teor\'ias m\'as sofisticadas (por 
ejemplo, teor\'ias que se acercan m\'as a la soluci\'on exacta de la
ecuaci\'on de Schr\"odinger). Un enorme cantidad de esfuerzos 
hab\'ian sido gastados en la construcci\'on matem\'atica y t\'ecnica 
computacionales para extender el l\'imite HF, lo cual es decir 
resolver las ecuaciones HF con la equivalente  de un conjunto de base 
infinito, sin aproximaci\'on adisional. Si el l\'imite HF es logrado,
entonces el error asociado a la energ\'ia de la aproximaci\'on HF del
sistema dado, el as\'i llamado energ\'ia de correlaci\'on 
electr\'onica, puede ser deteminada como
$$E_{corr}=E-E_{HF},$$
donde $E$ es la energ\'ia ``verdadera'' y $E_{HF}$ es la energ\'ia del 
sistema en el l\'imite HF.

HF lo largo del camino esto se volvi\'o claro que, tal vez 
sorpresivamente,  las energ\'ia HF qu\'imicamente \'utilies. 
T\'ipicamente su utilidad fue manifiesta para situaciones donde el 
error asociado con la ignorancia de la energ\'ia de correlaci\'on 
pod\'ia no ser importante en virtud de comparar dos o m\'as sistemas
para los cuales el error pod\'ia cancelarse. La t\'ecnica de usar las
ecuaciones isodesmicas representa un ejemplo de como tal 
comparaci\'on puede ser exitosamente hecha.

La disponibilidad de las funciones de onda HF hace posible la prueba
de que tan \'util tales funciones pueden ser para la predicci\'on de 
otras adem\'as de la energ\'ia. Simplemente porque las funciones de 
onda HF pueden ser arbitrariamente lejos de ser una eigenfinci\'on 
del hamiltoniano no implide {\it a priori} de estar razonablemente
cerca de otra eigenfunci\'on de alg\'un otro operador mec\'anico 
cu\'antico.
