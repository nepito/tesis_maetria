\section{Antecedentes}
%\parr{1}{?`C\'omo surgi\'o la idea del estudio?}
%KutaInorgChem2010.49.7808
El inter\'es de los cient\'ificos por lant\'anidos ha crecido en las
\'utlimas cinco d\'ecadas en diferentes ramas de la ciencia y la 
tecnolog\'ia. Los lant\'anidos son usados, por qu\'imicos m\'edico
y bioqu\'imicos, como agentes de contraste (MRI) y en la creaci\'on 
de nuevos reactivos biom\'edicos para probar ``rutas'' metab\'olicas.
La ciencia de materiales impuls\'o el desarrollo de s\'olidos 
magneto-\'optico, cer\'amicas, etc. que contiene lant\'anidos. La
industr\'ia hace uso de los lant\'anidos por su fuerte activaci\'on
en la cat\'alisis de la polimerizaci\'on de olefinas, en los
compuestos de lant\'anidos complejos y organomet\'alicos, entre otros
usos. La industr\'ia automotriz de carros h\'ibridos requiere grandes
cantidades de lant\'anidos en la fabricaci\'on de los motores 
el\'ectricos (neodiomio, terbio y disprocio) y en las baterias de los
carros (lantano); adem\'as, los lant\'anidos son necesarios en carros
con asientos y ventanas el\'ectricas. Una aleaci\'on de lant\'anidos 
es utilizada en la creaci\'on de imanes de alta resistencia que se 
utilizan en la industr\'ia de electr\'onicos que producen sonidos o 
que tienen peque\~nos moteres dentro. Este inter\'es tegnol\'ogico a
provocado problemas pol\'iticos: a principios de la segunda
d\'ecada del siglo 21, Estados Unidos, la Uni\'on Europea y Jap\'on 
presentaron una denuncia contra China por las limitaciones en el 
suministro de lant\'anidos, pues elevar\'ia los costos de 
producci\'on \citep{Ciup2010,Kuta2010,Jesk1985,Rao1989}.

Por otra parte, el inter\'es en los lant\'anidos tambi\'en a surgido
desde un punto de vista ecol\'ogico. Los lant\'anidos sirven de 
contaminantes trazas (en qu\'imica ambiental). Tambi\'en el uso de
energ\'ia nuclear provoc\'o el estudio de los lant\'anidos, ya que 
\'estos son parte de los productos de fisi\'on en el ciclo de 
combusti\'on. 
%El estado termodin\'amicamente m\'as estable de 
%lant\'anidos en soluci\'on es $3+$; adem\'as, ya que los estudios 
%te\'orico de los lant\'anidos en soluci\'on acuosa ayudan a la 
%comprenci\'on de la qu\'imica del proceso de separaci\'on 
%lant\'anidos-act\'inidos, productos en la basura nuclear, \'este 
%justifica el estudios te\'oricos de la hidrataci\'on de lant\'anidos 
%trivalentes. 
El estado termodin\'amicamente m\'as estable de los
lant\'anidos en soluci\'on es $3+$;  y como los estudios 
te\'orico ayudan a la comprenci\'on de la qu\'imica del proceso de 
separaci\'on lant\'anidos-act\'inidos, productos en la basura 
nuclear, esto es una justificaci\'on del estudio te\'oricos de la 
hidrataci\'on de lant\'anidos trivalentes. 
Adem\'as de lo antes mencionado, el estudio de la 
hidrataci\'on de lant\'anidos puede ayudar a desarrollar una 
metodolog\'ia te\'orica ``confiable'' para el estudio de los 
act\'inidos, \'esto se basa en el supuesto que se podr\'a tener un
tratamiento similar por ser series de elementos que van llenando una 
capa electr\'onica interna f de electrones, 4f en el caso de los 
lant\'anidos y 5f los act\'inidos. El inter\'es surge por la falta de
datos experimentales, debida a la radioactividad de los act\'inidos
\citep{Ciup2010,Kuta2010,Bunz2006}.


%\parr{4}{Estado de la cuesti\'on}
Las mejoras en las t\'ecnicas experimentales y el desarrollo de 
modelos te\'oricos m\'as sofisticados y realizables permitieron un 
creciente n\'umero de estudios en los \'ultimos diez a\~nos, tanto 
experimentales como te\'oricos, para obtener un mejor entendimiento 
de las propiedades de hidrataci\'on de iones de los lant\'anidos 
\citep{Dang2012}. Las discrepancias entre los datos experimentales
y los te\'oricos son otro motivo para llevar a cabo una cuidadosa 
investigaci\'on de las propiedades termodin\'amicas y de las 
estructuras geom\'etricas de la hidrataci\'on de toda la serie, a\'un 
cuando es generalmente aceptado que existen muchas similitudes entre
todos los elementos trivalentes de la serie.

El estudio t\'eorico de la hidrataci\'on de lant\'anidos tiene 
algunos problemas provocados, por ejemplo, por el gran n\'umero de
orbitales y electr\'ones, y la importancia de los efectos de 
correlaci\'on y relativistas que est\'an presentes en los 
lant\'anidos. Los potenciales at\'omicos son un punto de partida en 
la predicci\'on de propiedades moleculares, cuyos par\'ametros pueden
ser ajustados a c\'alculos {\it ab initio} de alta calidad. Para 
mejorar la precisi\'on de los modelos en la reproducci\'on de los 
datos experimentales puede ser necesario cambiar las formas 
funcionales del potencial por unas que sean m\'as fieles a la 
descripci\'on molecular, pero tambi\'en es posible lograrlo mediante
una mayor calidad de los datos {\it ab initio}. Por
lo anterior, el estudio de la hidrataci\'on de lant\'anidos con 
diferentes niveles de c\'alculo {\it ab initio} dar\'a informaci\'on
de la metodolog\'ia necesaria para obtener los datos a los que se
ajustar\'an los par\'ametros de los modelos utilizados en
simulaciones moleculares 	\citep{Cund1993,Sadu1999,Hern2005}.

\subsection{Trabajos Experimentales}
%\parr{5}{Introducci\'on al Trabajo Experimental}
Se han empleado varios m\'etodos espectrosc\'opicos en el estudio de
hidrataci\'on i\'onica. La espectroscop\'ia por absorci\'on de rayos 
X (XAS, por sus siglas en ingl\' es) ha permitido determinar, con 
alta presici\'on, las caracter\'isticas de la estructura de 
hidrataci\'on. En particular, la estrcuctura fina por absorci\'on 
extendida de rayos X (EXAFS, por sus siglas en ingl\'es) tiene una 
gran ventaja sobre los m\'etodos usales de difracci\'on de rayos X 
debido a su alta selectividad del elemento central, el \'atomo 
absorbente, la estructura alrededor del \'el puede ser determinada
\citep{Ohta1993}. Sin embargo, para determinar la informaci\'on 
estructural, a partir de los datos experimentales de EXAFS, es 
necesario el uso de un modelo que puede ser obtenido de simulaciones 
de din\'amica molecular \citep{Dang2011}. 

%\paragraph{P\'ar.:7 Mezcla de t\'ecnicas}
\cite{Dang2011} desarrollaron y aplicaron, en la investigaci\'on de
soluciones acuosas de lant\'anidos, un m\'etodo para analizar 
el espectro EXAFS de sistemas l\'iquidos ajustando funciones de 
distribuci\'on radial, $g(r)$, obtenidas de simulaciones de 
din\'amica molecular. Esta metodolog\'ia combinada permite la prueba 
y validaci\'on de las funciones de distribuci\'on radial obtenidas de
la simulaci\'on y consecuentemente la seriedad del esquema te\'orico
usado en la simulaci\'on con base en los datos experimentales EXAFS 
\citep{Dang2012}. Los datos experimetales reportados por 
\cite{Dang2011} son los que sirven de datos de referencia en el 
presente trabajo.

\subsection{Trabajos Te\'oricos}
%\paragraph{P\'ar.:9 Introducci\'on al Trabajo Te\'orico}
Desde un punto de vista te\'orico dos cuestiones contribuyeron
al entendimiento de las propiedades de hidrataci\'on de los 
Ln$^{3+}$: el desarrollo de pseudopotenciales (PP) y de la teor\'ia 
del funcional de la densidad, en la qu\'imica cu\'antica y el 
desarrollo de una nueva generaci\'on de campos de fuerzas, 
incluyendo polarizabilidad, en simulaciones de din\'amica molecular,
para proporcionar resultados en acuerdo con los datos experimentales
\citep{Dang2012} (una breve explicaci\'on de las mejora te\'oricas 
ser\'a dada en los siguientes cap\'itulos).
%\paragraph{Propuesta }
%{\it Resumen del trabajo de} \cite{Dolg1989,Cao2001}{\it y de la 
%correcci\'on de bases y la base extendida de} 
%\cite{Dolg1993,Yang2005}.

%\paragraph{P\'ar.:15 Trabajos te\'oricos de inter\'es}
Recientemente \cite{Dang2012} publicaron una revisi\'on de las 
investigaciones de la hidrataci\'on de lat\'anidos. Entre los 
trabajos te\'oricos citados por ellos, los trabajos de 
\cite{Vill2009}, \cite{Ciup2010} y \cite{Kuta2010} son los que nos 
interesan discutir ya que utilizan metodolog\'ias distintas y los
resultados, accesibles con esas metodolog\'ias, son los m\'as 
cercanos a los datos experimentales. 

\citeauthor{Vill2009} hacen un estudio de la estructura y la 
din\'amica de la hidrataci\'on de tres lant\'anidos trivalentes 
(neodimio, gadolinio e iterbio) mediante din\'amica molecular. Los 
potenciales modelos que utilizaron son flexibles y polarizables 
\citep{Sain2000}, ajustados a datos {\it ab initio}. Ellos hicieron 
los c\'alculos cu\'anticos a nivel MP2 y utilizaron el 
pseudopotencial largo con la correspondiente base \citep{Dolg1989} 
para los lant\'anidos y la base aug-cc-pVDZ para las mol\'eculas de 
agua \citep{Dunn1989}. Los n\'umeros de coordinaci\'on predichos 
est\'an en acuerdo con los datos experimentales y las constantes de 
las tasas de intercambio son las mejores reportadas en el trabajo de 
\cite{Dang2012}, pero las distancias promedio del lant\'anido a las 
mol\'eculas de agua fueron aproximadamente 0.1 $\AA$  mayores a las 
distancia experimental.


%\paragraph{P\'ar.:16 Trabajo de \cite{Kuta2010}}
\cite{Kuta2010} hicieron c\'alculos cu\'anticos a nivel DFT/B3LYP y
utilizando el pseudopotencial de Stuttgart pseudorelativista con 28 
electrones en la capa interna de carozo (pseudopotencial corto) 
\citep{Cao2001} y las bases segmentadas %(14s13p10d8f6g)/[10s8p5d4f3g] 
\citep{Cao2002} encontrando que la distancia promedio 
lant\'anido-ox\'igeno, tanto para los sistemas con coordinaci\'on 
ocho como los de nueve, disminuye cuadr\'aticamente, como deber\'ia 
esperarse de acuerdo con con la contracci\'on de los lant\'anidos. 
Las distancias que reportaron fueron sistem\'aticamente mayores 
($\sim$ 0.05 \AA) a las experimentales, con solo una esfera de 
hidrataci\'on. Adem\'as, cuando agregaron la segunda esfera de 
hidrataci\'on de manera expl\'icita lograron una contracci\'on de la 
longitud promedio lant\'anido ox\'igeno (de 0.02-0.07 \AA) mejorando 
los resultados, dando valores mucho m\'as cercanos a los 
experimentales.

%\paragraph{P\'ar.:17 Trabajo de \cite{Ciup2010}}
Por otro parte, \cite{Ciup2010} en sus c\'alculos usaron un PP 
que incluyera la capa f, el psudopotencial largo \citep{Dolg1989}, y 
la base extendida para los electrones de valencia de los iones de 
lant\'anidos \citep{Yang2005}, %fue (6s6p5d)/[4s4p4d] con la adisi\'on 
%de 2, 1 y 1 funsiones difusas s, p y d y funciones de polarizaci\'on
%3f y 2g, es decir (8s7p6d3f2g)/[6s5p5d3f2g], 
y para las mol\'eculas de agua utilizaron la base aug-cc-pVDZ 
\citep{Dunn1989}. Encontraron la necesidad, para obtener el n\'umero 
de coordinaci\'on experimental, de calcular las energ\'ias de los 
sistemas optimizados a nivel MP2 con el PP y las bases mencionadas 
para los lant\'anidos y con la base aug-cc-pVTZ para las mol\'eculas 
de agua a un nivel SCS-MP2 \citep{Grim2003} adem\'as de agregar un 
modelo de medio continuo polarizable (PCM). Las distancias que ellos 
obtuvieron son las m\'as cercanas a los datos experimentales 
reportados por D`Angelo \citep{Dang2012}.

%\paragraph{P\'ar.:18 Concluci\'on del la investigaci\'on bibiogr\'afica}
Tanto en el trabajo de Kuta como el de Ciupka se concluye que es 
necesaria la segunda capa de hidrataci\'on para obtener las 
estructuras de hidrataci\'on correcta. Adem\'as Ciupka muestra que no
es posible obtener el n\'umero de coordinaci\'on correcto usando DFT
y agregar expl\'icitamente la segunda capa de hidrataci\'on a un 
nivel SCS-MP2 no es realizable, computacionalmente hablando, 
y la inclusi\'on de los efecto de bulto mediante un modelo de medio
polarizable continuo en un c\'alculo de optimizaci\'on es 
extremadante dif\'icil debido a los problemas de convergencia y la 
gran cantidad de puntos sillas.

La intenci\'on del presente trabajo era mejorar los resultados
obtenidas por \citep{Vill2009} y completar el estudio de toda la 
serie lant\'anidos, por lo que se hicieron nuevos c\'alculos 
modificando los par\'ametros, haciendo el ajustando al l\'imite de 
base completa, pero en ninguno de los resultados se pudo llegar al 
n\'umero de coordinaci\'on experimental, por lo que se decidi\'o 
mejorar los c\'alculos cu\'anticos. Al encontrar que la metodolog\'ia
en los c\'alculos cu\'anticos para obtener el n\'umero de 
coordinaci\'on, las distancias promedio de los lant\'anidos a los
\'atomos de ox\'igeno y las energ\'ias libres de hidrataci\'on 
experimentales no era clara o incosistente en diferentes procesos del
c\'alculo se decidi\'o tratar de encontrar una metodolog\'ia que 
predijera los resultados experimentales de una forma autoconsistente.

%\paragraph{P\'ar.:19 Introduci\'on a AG}
Una de las propuestas del presente trabajo es la utilizaci\'on de
algoritmos gen\'eticos para resolver los problemas de convergencia en
la utilizaci\'on de PCM en las optimizaciones de sistemas cu\'anticos.
Los algoritmos gen\'eticos (AG) son m\'etodos de optimizaci\'on, los 
cuales no incluyen los c\'alculos de las dirivadas, y est\'an basados
en el proceso de evoluci\'on darwiniano. Los AG fueron exitosamente
usados en muchas \'areas de la ciencia y la tecnolog\'ia cuando el 
\'optimo global para funciones complejas y de muchos par\'ametros 
necesita ser encontrado. Los AG ya mostraron ser una t\'ecnica 
efectiva en la b\'usqueda de m\'inimos globales en sistemas
qu\'imicos, por ejemplo el trabajo de \cite{Alex2005} y los trabajos
ah\'i citados.

