
%\paragraph{El m\'etodo}
%Cuando la hidrataci\'on de Ln$^{3+}$ es abordada computacionalmente,
%los problemas que surgen son debidos al gran n\'umero de electrones 
%involucrados, la importancia de los efectos relativistas y la capa f 
%abierta de los lant\'anidos. La mayor\'ia de las investigaciones usan
%Pseudopotenciales (PP) y teor\'ia del funcional de la densidad con 
%los que se resuelven los dos primeros problemas y si la capa f es 
%agregada al pseudopotencial, tambi\'en el tercer problema 
%\citep{Ciup2010}. Las distancias Ln$^{3+}$-O son calculadas largas 
%en algunos casos probablemente debido a ignorar los efectos del bulto 
%de solvataci\'on. Para tomar en cuenta las esferas de hidrataci\'on 
%mayores, un modelos medio continuo polarizable fue usado. En la 
%combinaci\'on de pseudopotenciales escalares relativistas con la
%capa 4f en el carozo junto con DFT y m\'etodos de teor\'ia de 
%perturbaci\'on fueron utlizados. El concepto de PP con la capa f en
%el carozo, significa que todos los electrones con un n\'umero 
%principal menor a cinco son incluidos en el PP y no son considerados
%expl\'icitamente. Esto resulta en un efectivo de ocho electrones con 
%la ocupaci\'on $5s^25p^65d^06s^0$ para los Ln$^{3+}$.

%\paragraph{P\'ar.:6 Trabajo de D`Angelo}
%La combinaci\'on de simulaciones de din\'amica molecular y la 
%espectroscop\'ia XAS se volvi\'o en los \'ultimos diez a\~nos en una
%t\'ecnica conocida para descubrir la estructura de soluciones 
%i\'onicas, especialmente para cationes met\'alicos en un ambiente
%desordenado. Los espectros de XAS son t\'ipicamente extendidos dentro
%de una regi\'on de baja energ\'ia, nombrada XANES (por sus siglas en 
%ingl\'es: {\it X-ray absorption near-edge structure}), regi\'on que
%se extiende desde alrededor de 50 eV a m\'as de 1000 eV, arriba de la
%cual es llamada EXAFS ({\it extended X-ray absorption fine
%structure}). Comparada con la t\'ecnica de difracci\'on, EXAFS 
%muestra una buena resoluci\'on cerca del ambiente vecino del \'atomo
%fotoabsorbedor \citep{Dang2012}.

