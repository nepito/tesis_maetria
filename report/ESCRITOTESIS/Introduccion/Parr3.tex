Los potenciales at\'omicos son un punto de partida en la predicci\'on
de propiedades moleculares, cuyos par\'ametros pueden ser ajustados
a c\'alculos {\it ab initio} de alta calidad \citep{Sadu1999}. Para 
mejorar la precisi\'on de los modelos en la reproducci\'on de los 
datos experimentales puede ser necesario cambiar las formas 
funcionales del potencia por unas que sean m\'as fieles a la 
descripci\'on molecular, pero tambi\'en es posible lograrlo mediante
una mayor calidad de los datos {\it ab initio} \citep{Hern2005}. Por
lo anterior, el estudio de la hidrataci\'on de lant\'anidos con 
diferentes niveles de c\'alculo {\it ab initio} dar\'a informaci\'on
de la metodolog\'ia necesaria para obtener los datos a los que se
ajustar\'an los par\'ametros de los modelos utilizados en
simulaciones moleculares.
