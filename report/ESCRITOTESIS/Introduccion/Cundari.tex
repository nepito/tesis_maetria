%Cundari93_jcp.pdf
En el trabajo de Cuandari de 1993, se describe un extensi\'on de los
esquemas de Stevens {\it et al.}\footnote{No tengo la bibliograf\'ia:
W.J.Stevens, M.Krauss, H.Basch, P.G.Jasien, Can.J.Chem. 70, 612 
(1992).} para la derivaci\'on de potenciales efectivos de carozo
y conjunto de base de valencia, provando un conjunto consistente para
la tabla peri\'odica hasta el rad\'on. La pertinente caracter\'istica
del presente m\'etodo es el uso de representaciones compactas para 
los  ECP y el conjuto de la base de valencia. Se usan \colorbox
{Rhodamine}{los metales de transici\'on}, y la qu\'imica conocida de
los lant\'anidos, en la gu\'ia de toma de desiciones computacionales.
Se hace notar que el uso del t\'ermino lant\'anido (Ln) para 
describir los elementos del cerio ($Z=58$) hasta lutecio ($Z=71$). El
lant\'ano fue tratado como un metal de transici\'on. El t\'ermino de
tierras raras es evitado ya que t\'ipicamente incluye los metales
Sc-triad.

Estos ECP son consistentes no solo son la serie de lant\'anidos, sino 
tambi\'en con la tercera columna de los metales de transici\'on los
cuales los agrupan. Un carozo de 46 electrones fue escogido para 
proporcionar el mejor acuerdo entre el esfuerzo computacional y la
presici\'on qu\'imica. As\'i, los orbitales $5s$ y $5p$ son incluidos 
afuera del potencial de carozo mientras los orbitales at\'omicos de 
enereg\'ia menor son remplazados por el ECP.
