%Resumen de 3.3.6 Ab Initio Potencials, p\'agina 64, de Sadus 1999.
El potencial at\'omico por pares es un importante punto de partida de
de predicci\'on de propiedades moleculares. Hist\'oricamente, una
aproximaci\'on emp\'irica fue usada con los par\'ametros del
potencial siendo obtenidos de los datos experimentales y su 
evaluaci\'on utilizando simulaci\'on molecular es dirigida a 
incrementar la importancia de la predicci\'on en fluido moleculares. 
Los potenciales {\it ab initio} pueden ser una alternativa precisa a 
los potenciales emp\'iricos. Los c\'alculos {\it ab initio} pueden 
ser usados para determinar la interacci\'on entre \'atomos y ajustar 
los par\'ametros de los potenciales.
%Adem\'as un resumen del tema 6.3.9, p\'agina 183 de Berendsen 2007.
%El \'ultimo p\'arrafo de la p\'agina 184 de Berendsen 2007, el tema 
%la presici\'on en le campo de fuerza.
En general, los modelos de campos de fuerza para sistemas moleculares
no son de la calidad suficiente para para ser usado de manera 
universal. Pero para realizar un modelo se debe tener en cuenta: la
simplicidad, la robustes, la transferibilidad, la presici\'on y la
compatibilidad con los c\'alculos {\it ab initio}. Idealmente, un 
modelo del campo de fuerzas muy preciso es deseado, derivable de 
c\'alculos cu\'anticos, debe ser construido en forma jar\'arquica, 
por lo que el modelo debe tener aspectos separables, donde ninguna
resultar\'a en la omisi\'on de ningun otro aspecto f\'isico. As\'i
poder quitar t\'erminos que fluctuan del modelo principal y usar
valores promedios en los modelos secundarios.
%El \'ultimo p\'arrafo de pa p\'agina 187 de Berendsen 2007.
En la simulaci\'on molecular se usan modelos de las fuerzas las deben
ser para metrizados por ajustes de c\'alculos {\it ab initio} de alto
nivel.
%Revisar el tema 5 Aplicaci\'on de la metodolog\'ia a la 
%descripci\'on de agua l\'iquida, p\'agina 134 de las memorias de la
%VI escuela de verano. Recamier 1997.
%Traducci\'on del art\'iculo SaintMartinJChemPhts2005.122.224509
%Primer p\'arrafo de la discusi\'on
Este trabajo estudio los efectos de restringir los grados de libertad
de un modelo molecular generando cuatro versiones que corresponde a 
la misma funci\'on de energ\'ia potencial. Manteniendo los mismos 
par\'ametros para todos los potenciales se prueba que, con la 
energ\'ia de interacci\'on propia, tomando la flexibilidad 
intramolecular del agua conduce a una dram\'atica p\'erdida de la
estructura del agua. Pero cuando el promedio de la geometr\'ia
molecular en el l\'iquido es usada, la mol\'ecula r\'igida se 
desempe\~na casi de manera id\'entica a la mol\'ecula restringida 
flexible. Remover la polarizabilidad, fijando el centro de la 
distribuci\'on de carga puede producir rasonablemente buenos 
resultados, provistos por el dipolo molecula promedio en el l\'iquido
es tambi\'en proporcionado. Evidentemente, el \'ultimo potencial 
logra predicciones pobres de los agregados de agua, desde que este
hab\'ia sido puesto en un potencial efectivo restringiendo sus 
propiedades moleculares a los promedios en el l\'iquido en 
condiciones ambiente. EL mismo procedimiento deber\'ia de permitir
construir potenciales efectivos en cualquier condici\'on: despu\'es
de obtener los promedios de propiedades moleculares de simulaciones
cortas con un modelo completamnte flexible, esots promedios pueden 
ser fijados y mantenidos para continuar la simulaci\'on. Esta es una
alternativa autoconsistente para construir potenciales casados en 
c\'alculos {\it ab initio} y propiedades de una sola mol\'ecula, sin 
reparametrizar para cada uno de los conjuntos diferentes de las
condiciones termodin\'amicas, y con la ventaja que incluso modelos
simples reflejar\'an mejoras de los datos {\it ab initio} usados en
el aprendizaje. Esto es particularmente importante para la variedad
de sistemas moleculares para los cuales los datos experimentales son
m\'as escasos.
%Traducci\'on del art\'iculo HernandezJChemPhys2005.123.044506
%Segundo p\arrafo de la conclusi\'on.
En adisi\'on al mejoramiento de los modelos por una mayor calidad de
datos {\it ab initio}, es decir, una superficie de energ\'ia 
potencial m\'as precisa, los modelos son tambi\'en mejorados con 
formas funcionales que son m\'as fieles a la descripci\'on molecular.
