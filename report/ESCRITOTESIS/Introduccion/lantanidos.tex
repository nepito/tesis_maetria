%DAngeloChemEurJ2012.18.11162
La hidrataci\'on de iones es uno de los temas  m\'as atractivos y 
viejos de la qu\'imica, especialmente para aquellos quienes est\'an
interesados en reacciones en soluciones acuosas. Desde el tiempo de
Arrhenius, numerosos estudios en hidrataci\'on i\'onica, tales como
la determinaci\'o del n\'umero de hidrataci\'on  de los iones, las
distancias de equilibrio ion-solvente, el radios i\'onico, las
tasas de intercambio de las mol\'eculas de agua coordinadas alrededor
del ion y los coeficientes de difusi\'on han sido hechos. Los 
lant\'anidos, los cuales forman el bloque f en la tabla per\'odica, 
son casi siempre presentes  en el agua como 3$^+$ bare cationes (con 
excepciones de Ce y Yb, ya has puesto esta informaci\'on). 

En los \'ultimos d\'iez a\~nos un creciente n\'umero de estudios 
dedicados, tanto experimental como te\'oricos, para el entendimiento 
de las propiedades de hidrataci\'on de los iones lant\'anidos en 
agua, y una clara descripci\'on emerg\'ia. Esto fue posible gracias 
al mejoramiento de las t\'ecnicas experimentales y al desarrollo de 
modelos te\'oricos m\'as sofisticados y realizables. 
\begin{itemize}
\item Desde un punto de vista te\'orico dos cuestiones han 
contribuido al entendimiento de las propiedades de hidrataci\'on de 
los Ln$^{III}$: el desarrollo de pseudopotenciales y el desarrollo de
una nueva generaci\'on de campos de fuerzas, incluyendo 
polarizabilidad, as\'i que simulaciones de din\'amica molecular
proporcionaron resultados en acuerdo con los datos experimentales.
\item La espectroscop\'ia por absorci\'on de rayos X jug\'o un papel
invaluable  en muchos aspectos de la hidrataci\'on de lant\'anidos, 
en particular en la determinaci\'on las caracter\'isticas de la 
estructura de hidrataci\'on, las distancias ion-agua y el n\'umero de
coordinaci\'on de la primera esf\'era de hidrataci\'on. Esto fue 
posible gracias a la combinaci\'on de experimentos con simulaciones 
te\'oricas.
\end{temize}


%Cundari93_jcp.pdf
El uso de potenciales efectivos de carozo (ECP) para remplazar los 
electrones internos, los cuales no participan en reacciones 
qu\'imicas normales, ha sido exitoso para el tratamiento de \colorbox
{Rhodamine}{los metales de transici\'on.}

%DolgTheorActa1989.pdf
Desde el punto de vista qu\'imico los elementos de las tierras raras
son usualmente trivalentes, menos frecuentes bivalentes (Eu e Yb) o 
tetravelentes (Ce y Tb). Esta simple descripci\'on corresponde a la
presencia de subconfiguraciones en mol\'eculas at\'omicas $4f^4$ 
($n=0-14$ para La-Lu), $4f^{n+1}$ ($n=0-13$, para La-Yb) o $4f^{n-1}$
($n=1-14$ para Ce-Lu).
Las energ\'ias orbitales de los c\'alculos de todos los electrones HF
sugiere que los orbitales $4f$ est\'en en el espacio de valencia, sin 
embargo su extenci\'on espacial sugiere que ellos pueden ser 
agregados al carozo. 
