\section{Antecedentes}
%\parr{1}{?`C\'omo surgi\'o la idea del estudio?}
%KutaInorgChem2010.49.7808
En los \'ultimos 50 a\~nos, los qu\'imicos mostraron inter\'es en los
lant\'anidos dentro de los campos de la ciencia de materiales (en 
s\'olidos magneto-\'opticos, cer\'amicas, etc.), energ\'ia nuclear 
(como productos de fisi\'on en el ciclo de combusti\'on), qu\'imica 
m\'edica (como agentes de contraste en MRI) y en qu\'imica ambiental 
(como contaminantes traza) \citep{Kuta2010}. En las \'ultimas tres 
d\'ecadas
%Cundari93_jcp.pdf
la qu\'imica de lant\'anidos experiment\'o un gran progreso debido
al inter\'es qu\'imico en los compuestos de lant\'anidos, complejos
y organomet\'alicos, en la fuerte activaci\'on en la cat\'alisis de 
la polimerizaci\'on de las olefinas \citep{Jesk1985}, en los 
superconductores a altas temperaturas \citep{Rao1989}, y en nuevos 
reactivos biom\'edicos para probar ``rutas'' metab\'olicos. A pesar
del creciente inter\'es qu\'imico por los lant\'anidos, la teor\'ia 
sufr\'ia un rezago respecto al experimento \citep{Cund1993}, debido
a los problemas computacionales, por ejemplo, el gran n\'umero de 
orbitales y electrones, y la importancia de los efectos de 
correlaci\'on y relativistas.

%\parr{2}{Hidrataci\'on de lant\'anidos trivalentes}
%CiupkaPhysChemChemPhys2010.12.13215
Ya que el uso de la energ\'ia nuclear es acompa\~nado por basura 
nuclear, la qu\'imica acuosa de iones de lant\'anidos y act\'inidos,
principalmente en el proceso de separaci\'on de 
lant\'anidos-act\'inidos y act\'inidos-act\'inidos, ha recibido un 
interes especial \citep{Ciup2010}. Esto se debe a que los m\'etodos 
de separaci\'on en dos fases, solvente de extracci\'on y ion de 
intercambio casi siempre involucran una fase acuosa. Adem\'as, ya que
el estado de oxidaci\'on m\'as estable termodin\'amicamente de los 
lant\'anidos, en soluci\'on, es 3+ \citep{Bunz2006}, el trabajo en 
este campo es concerniente a tal estado de oxidaci\'on. 

%\parr{3}{Necesidad te\'orica}
Los potenciales at\'omicos son un punto de partida en la predicci\'on
de propiedades moleculares, cuyos par\'ametros pueden ser ajustados
a c\'alculos {\it ab initio} de alta calidad \citep{Sadu1999}. Para 
mejorar la precisi\'on de los modelos en la reproducci\'on de los 
datos experimentales puede ser necesario cambiar las formas 
funcionales del potencia por unas que sean m\'as fieles a la 
descripci\'on molecular, pero tambi\'en es posible lograrlo mediante
una mayor calidad de los datos {\it ab initio} \citep{Hern2005}. Por
lo anterior, el estudio de la hidrataci\'on de lant\'anidos con 
diferentes niveles de c\'alculo {\it ab initio} dar\'a informaci\'on
de la metodolog\'ia necesaria para obtener los datos a los que se
ajustar\'an los par\'ametros de los modelos utilizados en
simulaciones moleculares.


%\parr{4}{Estado de la cuesti\'on}
Las mejoras en las t\'ecnicas experimentales y el desarrollo de 
modelos te\'oricos m\'as sofisticados y realizables permitieron un 
creciente n\'umero de estudios en los \'ultimos diez a\~nos, tanto 
experimentales como te\'oricos, para obtener un mejor entendimiento 
de las propiedades de hidrataci\'on de iones de los lant\'anidos 
\citep{Dang2012}. Las discrepancias entre los datos experimentales
y los te\'oricos motivan a una cuidadosa investigaci\'on de las 
propiedades termodin\'amicas y de las estructuras geom\'etricas de la
hidrataci\'on de toda la serie, a\'un cuando es generalmente aceptado
que existen muchas similitudes entre todos los elementos trivalentes
de la serie.

\subsection{Trabajos Experimentales}
%\parr{5}{Introducci\'on al Trabajo Experimental}
Se han empleado varios m\'etodos espectrosc\'opicos en el estudio de
hidrataci\'on i\'onica. La espectroscop\'ia por absorci\'on de rayos 
X (XAS, por sus siglas en ingl\' es) ha permitido determinar, con 
alta presici\'on, las caracter\'isticas de la estructura de 
hidrataci\'on. En particular, la estrcuctura fina por absorci\'on 
extendida de rayos X (EXAFS, por sus siglas en ingl\'es) tiene una 
gran ventaja sobre los m\'etodos usales de difracci\'on de rayos X 
debido a su alta selectividad del elemento central, el \'atomo 
absorbente, la estructura alrededor del \'el puede ser determinada
\citep{Ohta1993}. Sin embargo, para determinar la informaci\'on 
estructural, a partir de los datos experimentales de EXAFS, es 
necesario el uso de un modelo que puede ser obtenido de simulaciones 
de din\'amica molecular \citep{Dang2011}. 

%\paragraph{P\'ar.:7 Mezcla de t\'ecnicas}
\cite{Dang2011} desarrollaron y aplicaron, en la investigaci\'on de
soluciones acuosas de lant\'anidos, un m\'etodo para analizar 
el espectro EXAFS de sistemas l\'iquidos ajustando funciones de 
distribuci\'on radial, $g(r)$, obtenidas de simulaciones de 
din\'amica molecular. Esta metodolog\'ia combinada permite la prueba 
y validaci\'on de las funciones de distribuci\'on radial obtenidas de
la simulaci\'on y consecuentemente la seriedad del esquema te\'orico
usado en la simulaci\'on con base en los datos experimentales EXAFS 
\citep{Dang2012}. Los datos experimetales reportados por 
\cite{Dang2011} son los que sirven de datos de referencia en el 
presente trabajo.

\subsection{Trabajos Te\'oricos}
%\paragraph{P\'ar.:9 Introducci\'on al Trabajo Te\'orico}
Desde un punto de vista te\'orico dos cuestiones contribuyeron
al entendimiento de las propiedades de hidrataci\'on de los 
Ln$^{3+}$: el desarrollo de pseudopotenciales (PP) y de la teor\'ia 
del funcional de la densidad, en la qu\'imica cu\'antica y el 
desarrollo de una nueva generaci\'on de campos de fuerzas, 
incluyendo polarizabilidad, en simulaciones de din\'amica molecular,
para proporcionar resultados en acuerdo con los datos experimentales
\citep{Dang2012} (una breve explicaci\'on de las mejora te\'oricas 
ser\'a dada en los siguientes cap\'itulos).
%\paragraph{Propuesta }
%{\it Resumen del trabajo de} \cite{Dolg1989,Cao2001}{\it y de la 
%correcci\'on de bases y la base extendida de} 
%\cite{Dolg1993,Yang2005}.

%\paragraph{P\'ar.:15 Trabajos te\'oricos de inter\'es}
Recientemente \cite{Dang2012} publicaron una revisi\'on de las 
investigaciones de la hidrataci\'on de lat\'anidos. Entre los 
trabajos te\'oricos citados por ellos, los trabajos de 
\cite{Ciup2010} y \cite{Kuta2010} son los que nos interesa discutir 
ya que sus resultados son los m\'as cercanos a los datos 
experimentales, hacen uso de psuedopotenciales y trabajan con la 
serie completa de los lant\'anidos.

%\paragraph{P\'ar.:16 Trabajo de \cite{Kuta2010}}
\cite{Kuta2010} hicieron c\'alculos cu\'anticos a nivel DFT/B3LYP y
utilizando el pseudopotencial de Stuttgart pseudorelativista con 28 
electrones en la capa interna de carozo (pseudopotencial corto) 
\citep{Cao2001} y las bases segmentadas %(14s13p10d8f6g)/[10s8p5d4f3g] 
\citep{Cao2002} encontrando que la distancia promedio 
lant\'anido-ox\'igeno, tanto para los sistemas con coordinaci\'on 
ocho como los de nueve, disminuye cuadr\'aticamente, como deber\'ia 
esperarse de acuerdo con con la contracci\'on de los lant\'anidos. 
Las distancias que reportaron fueron sistem\'aticamente mayores 
($\sim$ 0.05 \AA) a las experimentales, con solo una esfera de 
hidrataci\'on. Adem\'as, cuando agregaron la segunda esfera de 
hidrataci\'on de manera expl\'icita lograron una contracci\'on de la 
longitud promedio lant\'anido ox\'igeno (de 0.02-0.07 \AA) mejorando 
los resultados, dando valores mucho m\'as cercanos a los 
experimentales.

%\paragraph{P\'ar.:17 Trabajo de \cite{Ciup2010}}
Por otro parte, \cite{Ciup2010} en sus c\'alculos usaron un PP 
que incluyera la capa f, el psudopotencial largo \citep{Dolg1989}, y 
la base extendida para los electrones de valencia de los iones de 
lant\'anidos \citep{Yang2005}, %fue (6s6p5d)/[4s4p4d] con la adisi\'on 
%de 2, 1 y 1 funsiones difusas s, p y d y funciones de polarizaci\'on
%3f y 2g, es decir (8s7p6d3f2g)/[6s5p5d3f2g], 
y para las mol\'eculas de agua utilizaron la base aug-cc-pVDZ 
\citep{Dunn1989}. Encontraron la necesidad, para obtener el n\'umero 
de coordinaci\'on experimental, de calcular las energ\'ias de los 
sistemas optimizados a nivel MP2 con el PP y las bases mencionadas 
para los lant\'anidos y con la base aug-cc-pVTZ para las mol\'eculas 
de agua a un nivel SCS-MP2 \citep{Grim2003} adem\'as de agregar un 
modelo de medio continuo polarizable (PCM). Las distancias que ellos 
obtuvieron son las m\'as cercanas a los datos experimentales 
reportados por D`Angelo \citep{Dang2012}.

%\paragraph{P\'ar.:18 Concluci\'on del la investigaci\'on bibiogr\'afica}
Tanto en el trabajo de Kuta como el de Ciupka se concluye que es 
necesaria la segunda capa de hidrataci\'on para obtener las 
estructuras de hidrataci\'on correcta. Adem\'as Ciupka muestra que no
es posible obtener el n\'umero de coordinaci\'on correcto usando DFT
y agregar expl\'icitamente la segunda capa de hidrataci\'on a un 
nivel SCS-MP2 no es realizable, computacionalmente hablando, 
y la inclusi\'on de los efecto de bulto mediante un modelo de medio
polarizable continuo en un c\'alculo de optimizaci\'on es 
extremadante dif\'icil debido a los problemas de convergencia y la 
gran cantidad de puntos sillas.

%\paragraph{P\'ar.:19 Introduci\'on a AG}
Una de las propuestas del presente trabajo es la utilizaci\'on de
algoritmos gen\'eticos para resolver los problemas de convergencia en
la utilizaci\'on de PCM en las optimizaciones de sistemas cu\'anticos.
Los algoritmos gen\'eticos (AG) son m\'etodos de optimizaci\'on, los 
cuales no incluyen los c\'alculos de las dirivadas, y est\'an basados
en el proceso de evoluci\'on darwiniano. Los AG fueron exitosamente
usados en muchas \'areas de la ciencia y la tecnolog\'ia cuando el 
\'optimo global para funciones complejas y de muchos par\'ametros 
necesita ser encontrado. Los AG ya mostraron ser una t\'ecnica 
efectiva en la b\'usqueda de m\'inimos globales en sistemas
qu\'imicos, por ejemplo el trabajo de \cite{Alex2005} y los trabajos
ah\'i citados.

