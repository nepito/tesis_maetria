%Dolg Theor Chim Acta(1989) 75:175-194
\paragraph{Pseudopotenciales}
Los m\'etodos de pseudopotenciales proporcionan una t\'ecnica 
conveniente y realizable para los c\'alculos cu\'anticos en las
componentes de los metales de transici\'on\footnote{M.Dolg, Wedig, 
H.Stoll,H.Preuss, J.Chem.Phys. (1987) 85:2123}. Debido a la 
posibilidad de usar un peque\~no conjunto de bases de valencia 
optimizadas, el esfuerzo computacional es reducido considerablemente,
al menos a nivel campo autoconsistente (SCF, por sus siglas en 
ingl\'es), en comparaci\'on con los c\'alculos de todos los 
electrones.

Para los elementos del grupo de las tierras raras no estamos 
concientes de alg\'un conjunto de pseudopotenciales realizables
publicados en la literatura. As\'i se presenta pseudopotenciales
ajustados a la energ\'ia no relativistas y quasirelativistas, junto
con sus conjunto de bases de valencia optimizadas, para los elementos
$4f$ de La hasta Lu.

