En este trabajo se calcul\'o las estructuras moleculares, las 
energ\'ias de amarre, las energ\'ias de hidrataci\'on y las
enrg\'ias libres de hidrataci\'on para \lnoh con $n=7, 8, 9$, para
la serie de los lant\'anidos. Los c\'alculos se hicieron utilizando
los pseudopotenciales que contienen a los electrones de la capa 4f y
los efectos relativistas son considerados mediante, el nivel de 
teor\'ia fue SCS-MP2 y se model\'o con un medio polarizable continuo
diel\'ectrico. Los resultados obtenidos pueden considerarse 
cuantitativamente correctos ya que los n\'umeros de coordinaci\'on de
la primera capa de hidrataci\'on son las observadas en los 
experimentos y las energ\'ias libres de hidrataci\'on est\'an son
cercanas a las experimentales (un 3\% de error en el peor de los 
casos). Se encontr\'o que las distancias promedios de los 
lant\'anidos a los \'atomos de ox\'igeno son mayores a las 
experimentales si los efectos del disolvente no son agregados, a un 
nivel MP2, adem\'as si el nivel de c\'alculo es SCS-MP2 las 
distancias dan a\'un mayor. Si se agrega el efecto del disolvente, 
mediante un medio diel\'ectrico continuo y polarizable, las 
distancias promedios son menores a las experimentales, a un nivel de 
c\'alculo MP2. Por lo anterior se concluye que para encontrar 
geometr\'ias en acuerdo con los datos experimentales de una manera 
autoconcistente, con el mismo nivel de c\'alculo y hamiltoniano se 
logren tanto las distancias como los n\'umeros de coordinaci\'on 
experimentales, es necesario hacer los c\'alculos a nivel SCS-MP2 y
modelando los efectos del disolvente mediante el medio diel\'ectrico
continuo y polarizable. Tambi\'en se encontr\'o que al agregar los
efectos del disolventes, tanto en las geometr\'ias optimizadas en 
fase gas como aqeuellas optimizadas tomando en cuenta los efecctos de
bulto al agregar el medio polarizable para ver cual es la energ\'ia 
del sistema en esa configuraci\'on, sin volver a hacer la 
optimizaci\'on, las diferencias en las energ\'ias de los sistemas con
diferentes n\'umeros de coordinaci\'on dan diferencias energ\'eticas
esperadas, aunque no ocurre los mismo para las diferencias de las 
energ\'ias libres, a menos que la geometr\'ia sea la optimizada con
el medio polarizable. Por \'ultimo se encontr\'o que no es necesario
utilizar los pseudopotenciales conrtos, es decir, aquellos que toman
en cuenta los efectos de los electrones de la capa 4f de manera 
expl\'icita pues las distancias obtenidas no son las correctas. Lo
mismo ocurre si se utiliza la base extendida y los efectos del 
disolvente sin importar si el nivel de c\'alculo es MP2 o SCS-MP2. 

%Como se muestra en las figuras \ref{fDLO2} y \ref{fDLO3} usar el 
%Pseudopotencial corto y sus bases correspondientes da como resultado
%distancias m\'as cortas que las observadas en los experimentos. 
