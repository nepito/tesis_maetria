En este trabajo se calculó las estructuras moleculares, las 
energías de amarre, las energías de hidratación y las
energías libres de hidratación para $\lnoh$ con $n=7, 8, 9$, para
la serie de los lantánidos. Los cálculos se hicieron utilizando
los pseudopotenciales que contienen a los electrones de la capa 4f y
los efectos relativistas, el nivel de teoría fue SCS-MP2 y los 
efectos del disolvente se modeló con un medio dieléctrico 
polarizable continuo. Los resultados obtenidos pueden considerarse 
cuantitativamente correctos ya que los números de coordinación de
la primera capa de hidratación son los observados en los 
experimentos y las energías libres de hidratación son
cercanas a las experimentales (un 3\% de error en el peor de los 
casos). Se encontró que las distancias promedios de los 
lantánidos a los átomos de oxígeno son mayores a las 
experimentales si los efectos del disolvente no son agregados, a un 
nivel MP2, además si el nivel de cálculo es SCS-MP2 las 
distancias dan aún mayor. Si se agrega el efecto del disolvente, 
mediante un medio dieléctrico continuo y polarizable, las 
distancias promedios son menores a las experimentales, a un nivel de 
cálculo MP2. Por lo anterior se concluye que para encontrar 
geometrías en acuerdo con los datos experimentales de una manera 
autoconsistente, que con el mismo nivel de cálculo y hamiltoniano 
se logren tanto las distancias como los números de coordinación 
experimentales, es necesario hacer los cálculos a nivel SCS-MP2 y
modelando los efectos del disolvente mediante el medio dieléctrico
continuo y polarizable. También se encontró que al agregar los
efectos del disolventes, tanto en las geometrías optimizadas en 
fase gas como aquellas optimizadas tomando en cuenta los efectos de
bulto, sin volver a hacer la optimización, las diferencias en las 
energías de los sistemas con diferentes números de coordinación
dan el signo de las diferencias energéticas esperado, aunque no 
ocurre los mismo para las diferencias de las energías libres, a 
menos que la geometría sea la optimizada con
%continuo y polarizable. También se encontró que al agregar los
%efectos del disolventes, tanto en las geometrías optimizadas en 
%fase gas como aquellas optimizadas tomando en cuenta los efectos de
%bulto, al agregar el medio polarizable para ver cual es la energía 
%del sistema en esa configuración, sin volver a hacer la 
%optimización, las diferencias en las energías de los sistemas con
%diferentes números de coordinación dan diferencias energéticas
%esperadas, aunque no ocurre los mismo para las diferencias de las 
%energías libres, a menos que la geometría sea la optimizada con
el medio polarizable ({\it ?`y qué pasa con el nivel de 
cálculo}). Por último se encontró que no es necesario utilizar 
los pseudopotenciales cortos, es decir, aquellos que toman
en cuenta los efectos de los electrones de la capa 4f de manera 
explícita, pues las distancias obtenidas no son las correctas, son
más cortas que las experimentales. Lo mismo ocurre si se utiliza la
base extendida de los pseudopotenciales largos y los efectos del 
disolvente sin importar si el nivel de cálculo es MP2 o SCS-MP2 
({\it ?`y qué efecto tiene en las energías?}). 

%Como se muestra en las figuras \ref{fDLO2} y \ref{fDLO3} usar el 
%Pseudopotencial corto y sus bases correspondientes da como resultado
%distancias más cortas que las observadas en los experimentos. 
