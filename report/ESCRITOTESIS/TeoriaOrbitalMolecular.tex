\section*{Orbitales Mol\'eculares}
% Lo siguiente es una traducción de la sección "Semienpirical 
% Implementations of Molecular Orbital Theory" del quinto capítulo
% del libro "Essentials of Computational Chemestry" de C. J. Cramer
A pesar de los obst\'aculos te\'oricos muchos qu\'imicos reconocieron
el rol potencialmente cr\'itico que pod\'ia jugar en avanzar los 
progresos experimentales en cualquiera de sus fronteras. Y el interes 
de la poblaci\'on de qu\'imicos no estaba restringido a de solo un 
poco de \'atomos. Aceptando la teor\'ia Hartree-Fock (HF) como sistema
de trabajo, varios grupos de investigadores volvieron su atenci\'on a
la implementaci\'on de la teor\'ia que har\'ia m\'as tratable, y tal 
vez m\'as precisa, para mol\'eculas de tama\~no moderado. Estos pasos
se desviaron, llevaron a una cierta bifurcaci\'on de los esfuerzos en
el \'area de teor\'ia de orbital molecular (aunque ciertamente algunos
grupos de investigadores fruncieron los t\'opicos en ambas 
direcciones) tal que persiste en nuestros dias. Los c\'alculos 
semiemp\'iricos continuan apareciendo en un gran n\'umero de 
literatura qu\'imica; ya que ah\'i siempre habr\'a investigadores 
interesados en mol\'eculas que exedan el tama\~no de aquellos 
pr\'actimcamente accesibles a los m\'etodos {\it ab initio}, los 
niveles de la teor\'ia de orbitales moleculares est\'an ciertamente
para continuar siendo construidos y aplicados. 
 
% Lo siguiente es una copia íntegra de la tesis "efectos de hidratación 
% de un modelos estereodinámico para el agua líquida" presentada por
% Nina Pastor. En la seccicón 2.2.1.1.
Dentro  del modelo de orbitales moleculares, el tratamiento aproximado
de la distribuci\'on electr\'onica y su movimiento se hace asignando 
electrones individuales a funciones monoelectr\'onicas llamadas esp\'in-
orbitales $\xi$. \'Estos est\'an formandos por un producto de funciones
especiales, llamadas orbitales molecuares $\psi$, y funciones de 
esp\'in  $\alpha$ o $\beta$. Los esp\'in-orbitales tienen completa 
libertad de extenderse por toda la mol\'ecula, y la asignaci\'on de
electrones a esp\'in-orbitales particulares define la configuraci\'on
electr\'onica.

Los orbitales $\psi$ pueden ser doblemente ocupados, semiocupados o no
ocupados (tambi\'en llamados virtuales). Para mol\'eculas que tienen 
un n\'umero par de electrones, el estado base se puede representar a
primera aproximaci\'on con una configuraci\'on electr\'onica donde hay
orbitales doblemente ocupados y vac\'ios. A este tipo de 
configuraci\'on se le llama de capa cerrada.

