\section*{Hatree-Fock}
Una vez que se ha construido el determinante de Slater, el siguiente
paso es encontrar los valores de los coeficientes de la base. Esto se
hace con el m\'etodo variacional dentro de la teor\'ia Hartree-Fock 
(HF).

F\'isicamente, la aproximaci\'on de HF desprecia la repulsi\'on 
instantanea de los electrones. Dentro de HF, cada electr\'on 
interactua con la distribuci\'on promedio de carga que proviene del 
resto de los electrones. Esto significa que los electrones se ven unos
a otros como distribuciones en lugar de part\'iculas discretas.
Matem\'aticamente, el resultado es que el t\'ermino de repulsi\'on de 
dos cuerpos en el hamiltoniano molecular \colorbox{Rhodamine}{se 
reduce a un t\'ermino de un cuerpo.}

Los orbitales moleculares \'optmios son los que satisfacen la 
ecuaci\'on de Hartree-Fock (HF):
\be\label{ehf}
\hat F(k)\phi_i(k)=\epsilon_i\phi_i(k)
\ee
donde $\epsilon_i$ es la energ\'ia del orbital $i$ y $k$ representa 
un electr\'on. El operador de Fock $\hat F$ contiene el hamiltoniano
de un solo electr\'on: su energ\'ia cin\'etica y su energ\'ia 
potencial con los n\'ucleos y los dem\'as electrones. \colorbox
{Rhodamine}{La suma de estos operadores no produce el hamiltoniano 
total porque se cuenta de m\'as las repulsiones interelectr\'onicas.} 
\colorbox{Yellow}{Esto es una consecuencia del t\'ermino de dos 
cuerpos, que hace que problema no sea de variables separables.} La
aproximaci\'on SCF cambia el t\'ermino de repulsi\'on  por un 
operador que representa el campo promedio generado por los electrones,
y esto s\'i permite la separaci'on de variables porque todos los 
operadores son de un cuerpo:
$$\hat F=\hat h(k)+\hat g(k)$$
donde $\hat h$ es el hamiltoniano para el electr\'on en el campo de 
los n\'ucleos y $\hat g$ representa el campo promedio que el 
electr\'on $k$experimenta debido a la presincia de los otros 
electrones.

HF-SCF no da una descripci\'on completa de la distribuci\'on 
electr\'onica. Para efecto de los modelos que se usan para resolver 
ecuaci\'on de Sch\"odinger, los electrones son part\'iculas e 
interactuan como tales. Dada la repulsi\'on lectr\'onica, una 
funci\'on de onda verdadera  debe dar una probabilidad menor de 
encontrar un electr\'on en la vecindad de otro en cualquier instante,
no solo en promedio.
