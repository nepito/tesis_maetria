\paragraph{Divergencia MP}
Un  estudio de la convergencia de las series perturbativas MP de core
congelado para peque\~nos \'atomos y mol\'eculas \citep{Olse1996} 
obtuvo que para la base cc-pVDZ, la serie converge ususalmente, pero
cuando la base se aument\'o con funciones difusas, las series MP a
menudo diverg\'ian. Por ejemplo, para el \'atomo de Ne, las 
contribuciones MP$n$/aug-cc-pVDZ para $n>16$ aumenta en magnitud
conforme aumenta $n$. (Las contribuciones MP$n$ de alto orden no se
calcularon directamente, sino que se obtuvieron como producto de los
c\'alculos FCI.) Para el ion F$^-$, la divergencia de las series
perturbativas afect\'o a la fiabilidad de los resultados 
MP4/aug-cc-pVDZ. Los c\'alculos MP de mol\'eculas diat\'omicas con
bases grandes mostraron resultados que se ``alejaban de la 
convergencia MP4'', con errores para MP4 rn $R_e$ y $\nu_e$ a menudo
mayores que los errores MP2. Esos estudios sugieren dudas acerca del
uso de la teor\'ia de perturbaci\'on MP para calcular las propiedades
moleculares.

La experiencia indica que la mayor parte de los c\'alculos de 
correlaci\'on electr\'onica el error en el truncamiento de la base es
mayor que el error debido al truncamiento del tratamiento de 
correlaci\'on. 
