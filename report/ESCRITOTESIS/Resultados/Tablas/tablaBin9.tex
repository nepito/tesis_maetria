\begin{table}[h!]
\centering
\caption{\footnotesize Negativo de la energ\'ia de amarre en fase gas
$D_o$ y la diferencia $\Delta E_{SCRF}$ entre de los sistemas [Ln$\cdot$
(H$_2$O)$_{n}$]$^{3+}$ y (H$_2$O)$_{n}$]$^{3+}$ (en eV).}
\begin{tabular}{c|cccccc}\hline\hline
Ion & $-D_o$ & $\Delta E_{\mbox{{\scriptsize SCRF}}}$ & 
$\Delta E_{\mbox{{\scriptsize H}}} $ 
& $-D_o^4$ & $\Delta E_{\mbox{{\scriptsize SCRF}}}^4$ & 
$\Delta E_{\mbox{{\scriptsize H}}}^4 $ \\ \hline
La$^{3+}$ &-21.9429 &-13.1888 &-35.1317 &-21.7184 &-13.1832 &-34.901\\ 
Gd$^{3+}$ &-24.1857 &-14.9439 &-39.1296 &-23.9702 &-14.9451 &-38.9153\\ 
Lu$^{3+}$ &-26.0301 &-14.0602 &-40.0903 &-25.8279 &-14.0559 &-39.8839\\ 
\hline 
\multicolumn{7}{p{11.0cm}}{
{\footnotesize $^\clubsuit$Pseudopotencial Largo \citep{Dolg1989}. 
$^\spadesuit$ Pseudopotencial Corto \citep{Cao2001}}.
{\footnotesize $^1$ Utilizando la base de \cite{Dolg1993}.} 
{\footnotesize $^2$ Utilizando la base de \cite{Yang2005}.}
{\footnotesize $^3$ Utilizando la base de \cite{Cao2002}.}  
{\footnotesize $^4$ Nivel SCS-MP2 \citep{Grim2003} y utilizando MPC 
\citep{Toma2005}.}}
\end{tabular}\label{tEB}\end{table}
