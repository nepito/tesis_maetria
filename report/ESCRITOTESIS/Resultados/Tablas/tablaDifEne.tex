\begin{table}[h!]
\centering
\caption{\footnotesize Diferencias en la energ\'ia $(\Delta E 
[\frac{KJ}{mol}])$ de los sistemas $E$[Ln$\cdot$
(H$_2$O)$_{n+(9-n)}$]$^{3+}$-$E$[Ln$\cdot$(H$_2$O)$_{(n-1)+(10-n)}$]$^{3+}$.}
%%\resizebox*{1.\textwidth}{!}{
\begin{tabular}{c|ccccc}\hline\hline
Ion & HF$^{\clubsuit,1}$ & MP2$^{\clubsuit,1}$ & MP2$^{\clubsuit,2}$ & 
MP2$^{\spadesuit,3}$ & MP2$^{\clubsuit,1,4}$     \\ \hline
La$^{3+}$(n=9) & 5.90835 & 6.08942 & 9.50653 & 16.08627 & 
-17.04981 \\ 
Eu$^{3+}$(n=9) &         &16.47709 &20.69336 &          &
 -5.63170 \\
Gd$^{3+}$(n=9) &20.00788 &18.19787 &22.75232 &          &
 -4.67339 \\ 
Tb$^{3+}$(n=9) &         &         &         &          &
 -2.88815 \\
Lu$^{3+}$(n=9) &35.26074 &32.69055 &37.59342 & 53.47987 &
 10.52741 \\ 
Lu$^{3+}$(n=8) & 0.74643 & 0.91630 & 2.46640 &  9.65396 &
-16.24490 \\ 
\hline 
\multicolumn{6}{p{11.0cm}}{
{\footnotesize $^\clubsuit$Pseudopotencial Largo \citep{Dolg1989}. 
$^\spadesuit$ Pseudopotencial Corto \citep{Cao2001}}.
{\footnotesize $^1$ Utilizando la base de \cite{Dolg1993}.} 
{\footnotesize $^2$ Utilizando la base de \cite{Yang2005}.}
{\footnotesize $^3$ Utilizando la base de \cite{Cao2002}.}  
{\footnotesize $^4$ Nivel SCS-MP2 \citep{Grim2003} y utilizando MPC 
\citep{Toma2005}.}}
\end{tabular}\label{tDE}\end{table}
