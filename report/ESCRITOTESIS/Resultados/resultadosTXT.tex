\section{Informaci\'on Geom\'etrica}
%\paragraph{Geometr\'ia} 
Es muy bien aceptado que el n\'umero de
coordinaci\'on (NC) de los lant\'anidos trivalentes cambia a lo largo
del periodo 4f, teniendo un NC=9 para los primeros Ln(III), mientras
un equilibrio es establecido entre NC=8 y NC=9 en los elementos de la
mitad de la serie, los \'ultimos tienen un NC=8. Teniendo lo anterior
en cuenta dos gr\'aficas de las distancias promedios te\'oricas de 
los lant\'anidos a los ox\'igenos, en sistemas con coordinaci\'on 8 y 
9, as\'i como de los datos experimentales son mostradas en las figuras 
\ref{fDLO2} y \ref{fDLO3}, en donde los lant\'anidos de la mitad de la
serie son repetidos en las dos gr\'aficas, Eu-Tb, y en la tabla 
\ref{tDis} los valores son presentados. 

%\paragraph{Simbolog\'ia}
Los resultados fueron 
obtenidos con diferentes pseudopotenciales, diferentes bases para el
mismo pseudopotencial y a diferentes niveles de c\'alculo y utilizando
teor\'ia de perturbaciones M\o ller-Plesset y la correcci\'on SCS 
\citep{Grim2003}. Los c\'alculos con diferentes pseudopotenciales son
identificados con diferentes figuritas, $\clubsuit$ corresponde a la
utilizaci\'on del pseudopotencial largo \citep{Dolg1989} y 
$\spadesuit$ corresponde al pseudopotencial corto \citep{Cao2001}. 
Las bases son representadas por n\'umeros, del 1 al 3, 1 corresponde 
a la base corregida de \cite{Dolg1993}, 2 la base extendida de 
\cite{Yang2005} y 3 corresponde a la base segmentada de 
\cite{Cao2002}. El n\'umero 4 representa c\'alculos utilizando la
correcci\'on de \cite{Grim2003} al m\'etodo MP2 adem\'as de agragar
los efectos de bulto. Finalmente a los datos experimentales se les
identifica con el n\'umero 5.
% y presentada en la tabla \ref{tDLO}

%\paragraph{Figura \ref{fDLO2}}
La figura \ref{fDLO2} corresponde a las distancias promedio
ox\'igeno-lant\'anido, para sistemas con n\'umero de coordinaci\'on 
nueve. Los resultados obtenidos a nivel MP2 utilizando el 
psuedopotencial largo \citep{Dolg1989} y su base corregida 
\citep{Dolg1993} reproducen la tendencia, tri\'angulos 
negros en la gr\'afica, pero la distancia es mayor a la experimental,
cruces rojas. Por ejemplo, para el lantano se obtiene una distancia de
2.6196 \AA, mientras que el valor experimental reportado es 2.600 \AA.
En cambio, utilizando la base extendida \citep{Yang2005} y el mismo 
pseudopotencial, tri\'angulo verde, la distancia es menor a la 
experimental, para el caso del lantano e intermedia entre la 
experimental y la te\'orica utilizando la base corregido para los 
lant\'anidos Eu-Tb. Si se cambia de pseudopotencial, al 
pseudopotencial corto \citep{Cao2001} y utilizando su base segmentada
\citep{Cao2002}, c\'irculo rojo, la distancia promedio te\'orica del 
lant\'ano a los ox\'igenos es la m\'as peque\~na, 2.5798 \AA. Si se 
utiliza el pseudopotencial largo y su base corregida, pero agregando 
los efectos de bulto, debidos a las capas de hidrataci\'on mayores, 
utilizando un modelo de medio polarizable continuo (PCM, por sus 
siglas en ingl\'es) la distancia no es tan peque\~na como la obtenida
con el pseudopotencial corto de \cite{Cao2001} ni tampoco tan grande 
como cuando no se considera los efectos de bulto utilizando la misma 
base y pseudopotencial, para el lantano se obtiene una distancia de 
2.5926 \AA. Se observa que es el resultado con la segunda diferencia 
m\'as peque\~na respecto al valor experimental, solo mayor a los 
resultados MP2$^{\clubsuit,2}$, utilizando el pseudopotencial largo y
la base extendida. Se observa que la diefencia entre las distancias
te\'oricas a nivel MP2$^{\clubsuit,1}$ y las experimentales va 
aumentando a lo largo de la serie.

%%%%%%%%%%%%%%%%%%%%%%%%%%%%%%%%%%%%%%%%%%%%%%%%%%5555
\begin{figure}[h]
\centering
\includegraphics[height=10cm,angle=-90]{Graficas/lnO9dist2.ps}
\caption{\small{Gr\'afica de la variaci\'on de la distancia promedio
lant\'anido ox\'igenoa a trav\'es de la serie 4f con n\'umero de 
coordinaci\'on igual a nueve. $^{\clubsuit}$Pseudopotencial Largo 
\citep{Dolg1989}. $^{\spadesuit}$Pseudopotencial Corto 
\citep{Cao2001}. $^1$Utilizando la base de \cite{Dolg1993}}, 
$^2$utilizando la base de \cite{Yang2005}, $^3$utilizando la base de 
\cite{Cao2002}, $^4$nivel SCS-MP2 \citep{Grim2003} y utilizando MPC 
\citep{Toma2005}, $^5$datos experimentales reportados por 
\cite{Dang2012}.}
\label{fDLO2}
\end{figure}
%%%%%%%%%%%%%%%%%%%%%%%%%%%%%%%%%%%%%%%%%%%%%%%%%%%%%%
%\paragraph{Figura \ref{fDLO3}}
La figura \ref{fDLO3} corresponde a las distancias promedio
ox\'igeno-lant\'anido, para sistemas con n\'umero de coordinaci\'on 
ocho. Los resultados obtenidos a nivel MP2 utilizando el 
psuedopotencial largo \citep{Dolg1989} y su base corregida 
\citep{Dolg1993} reproducen la tendencia, tri\'angulos 
negros en la gr\'afica, pero la distancia es mayor a la experimental,
cruces rojas. Por ejemplo, para el europio se obtiene una distancia de
2.4756 \AA, mientras que el valor experimental reportado es 2.455 \AA.
En cambio, utilizando la base extendida \citep{Yang2005} y el mismo 
pseudopotencial, tri\'angulo verde, la distancia es menor a la 
experimental, para el caso del gadolinio, e intermedia entre la 
experimental y la te\'orica utilizando la base corregido para los 
lant\'anidos Ho-Lu. Si se cambia de pseudopotencial, al 
pseudopotencial corto \citep{Cao2001} y utilizando su base segmentada
\citep{Cao2002}, c\'irculo rojo, la distancia promedio te\'orica del 
lutecio a los ox\'igenos es la m\'as peque\~na, 2.3252 \AA, mientras 
que la distancia experiemental es de 2.345 \AA. Si se utiliza el 
pseudopotencial largo y su base corregida, pero agregando los efectos
de bulto, debidos a las capas de hidrataci\'on mayores, utilizando un
modelo de medio polarizable continuo la distancia es la m\'as cercana
al dato experimental para el lutecio, se obtiene una distancia de 
2.3489 \AA. Se observa el mismo comportamiento en las diferencias 
entre las distancias te\'oricas a nivel MP2$^{\clubsuit,1}$ y las 
experimentales a las observadas en la figura \ref{fDLO2}, la 
diferencia va aumentando a lo largo de la serie de los lant\'anidos.
%%%%%%%%%%%%%%%%%%%%%%%%%%%%%%%%%%%%%%%%%%%%%%%%%%5555
\begin{figure}[h]
\centering
\includegraphics[height=10cm,angle=-90]{Graficas/lnO9dist3.ps}
\caption{\small{Gr\'afica de la variaci\'on de la distancia promedio
lant\'anido ox\'igeno a trav\'es de la serie 4f con n\'umero de 
coordinaci\'on igual a ocho. $^{\clubsuit}$Pseudopotencial Largo
\citep{Dolg1989}. $^{\spadesuit}$Pseudopotencial Corto 
\citep{Cao2001}. $^1$Utilizando la base de \cite{Dolg1993}}, 
$^2$utilizando la base de \cite{Yang2005}, $^3$utilizando la base de
\cite{Cao2002}, $^4$nivel SCS-MP2 \citep{Grim2003} y utilizando MPC 
\citep{Toma2005}, $^5$datos reportados por \cite{Dang2012}.}
\label{fDLO3}
\end{figure}
%%%%%%%%%%%%%%%%%%%%%%%%%%%%%%%%%%%%%%%%%%%%%%%%%%%%%%
%\paragraph{Las dos figuras}
En las figuras \ref{fDLO2} y \ref{fDLO3} se observa la contracci\'on 
de los lant\'anidos, cuya reproducci\'on te\'orica ya se report\'o
\citep{Kuta2010,Ciup2010,Dang2012}. Adem\'as se observa que las 
distancias promedios obtenidas a partir del pseudopotencial largo, la
base carregida y el medio polarizable continuo son menores a las 
obtenidas con el mismo pseudopotencial pero usando la base extendida.
\'Estas, a su vez, son menores a las obtenidas con el pseudopotencial
largo y su base corregida a nivel MP2, las cuales siempre fueron 
mayores a las reportadas en los experimentos y las distancias 
te\'oricas mayores. Las distancias m\'as peque\~nas se obtuvieron con
el pseudopotencial corto y siempre fueron menores a las reportadas 
experimentalmente.
  


%\paragraph{Tabla \ref{tDis}}
%%\begin{table}[h!]
\centering  %%
%\begin{center}
%\begin{table}[h!]
\caption{\footnotesize Distancias promedio del Lantano a los ox\'igenos
y los hidr\'ogenos, en {\AA}ngstr\"om.
{\footnotesize $^1$ Utilizando la base de \cite{Dolg1993}.} 
{\footnotesize $^2$ Utilizando la base de \cite{Yang2005}.}
{\footnotesize $^3$ Utilizando la base de \cite{Cao2002}.}  
{\footnotesize $^4$ Nivel SCS-MP2 \citep{Grim2003} y utilizando MPC 
\citep{Toma2005}.}}
\begin{tabular}{c|cc}\hline\hline
C\'alculo & $\overline R$(Ln-O) & $\overline R$(Ln-H) \\ \hline
HF/LCRECP & 2.6628 & 3.3244 \\ 
MP2/LCRECP$^1$ & 2.6196 & 3.3060  \\ 
MP2/LCRECP$^2$ & 2.5945 & 3.2804  \\ 
MP2/SCRECP$^3$ & 2.5798 & 3.2662  \\ 
MP2$^4$/LCRECP$^1$ & 2.5926 & 3.2793  \\ 
\hline \end{tabular}\label{tD9+0}\end{table}
%\end{center}

%%\begin{table}[h!]
\centering  %%
%\begin{center}
%\begin{table}[h!]
\caption{\footnotesize Distancias promedio del Gadolinio a los ox\'igenos
y los hidr\'ogenos, en {\AA}ngstr\"om.
{\footnotesize $^1$ Utilizando la base de \cite{Dolg1993}.} 
{\footnotesize $^2$ Utilizando la base de \cite{Yang2005}.}
{\footnotesize $^3$ Utilizando la base de \cite{Cao2002}.}  
{\footnotesize $^4$ Nivel SCS-MP2 \citep{Grim2003} y utilizando MPC 
\citep{Toma2005}.}}
\begin{tabular}{c|cc}\hline\hline
C\'alculo & $\overline R$(Ln-O) & $\overline R$(Ln-H) \\ \hline
HF/LCRECP & 2.5442 & 3.2069 \\ 
MP2/LCRECP$^1$ & 2.5028 & 3.1897  \\ 
MP2/LCRECP$^2$ & 2.4837 & 3.1701  \\ 
MP2/SCRECP$^3$ & x.xxxx & x.xxxx  \\ 
MP2$^4$/LCRECP$^1$ & 2.4785 & 3.1651  \\ 
\hline 
\multicolumn{3}{l}{\footnotesize{Esto es una mamada.}}
\end{tabular}\label{tDGd9+0}\end{table}
%\end{center}

\begin{table}[h!]
\centering  %%
%\begin{center}
%\begin{table}[h!]
\caption{\footnotesize Distancias promedio de los lant\'anidos a los 
ox\'igenos, en {\AA}ngstr\"om.}
\begin{tabular}{p{1cm}|p{1.8cm}p{1.8cm}p{1.8cm}p{1.8cm}p{1.8cm}p{1.8cm}}\hline\hline
Ion & HF$^{\clubsuit,1}$ & MP2$^{\clubsuit,1}$ & MP2$^{\clubsuit,2}$ & MP2$^{\spadesuit,3}$ & MP2$^{\clubsuit,1,4}$ & EXAFS$^5$ \\ \hline
La $^{3+}$ & 2.6628(9+0) & 2.6196(9+0) & 2.5945(9+0) & 2.5798(9+0) 
& 2.5926(9+0) & 2.600(9.1) \\ 
%Eu $^{3+}$ &             & 2.5147(9+0) & 2.4958(9+0) &       (9+0)  
%& 2.4889(9+0) & 2.455(9.0) \\ 
%           & 2.4952(8+1) & 2.4756(8+1) &       (8+1) &       (8+1) 
%& 2.4447(8+1) & 2.455(9.0)  \\ 
Gd $^{3+}$ & 2.5442(9+0) & 2.5028(9+0) & 2.4837(9+0) & x.xxxx(9+0)  
& 2.4785(9+0) & 2.455(9.0) \\ 
           & 2.4952(8+1) & 2.4609(8+1) & 2.4406(8+1) & x.xxxx(8+1) 
& 2.4389(8+1) & 2.455(9.0)  \\ 
%Tb $^{3+}$ &             & 2.4902(9+0) & 2.4710(9+0) &       (9+0)
%& 2.4581(9+0) & 2.440(9.0)  \\
%           &             & 2.4469(8+1) & 2.4266(8+1) &       (8+1)
%& 2.4276(8+1) & 2.440(9.0)  \\
Lu $^{3+}$ & 2.4057(8+1) & 2.3737(8+1) & 2.3595(8+1) & 2.3252$^*$(8+1) 
& 2.3489(8+1) & 2.345(8.2) \\ 
\hline 
\multicolumn{7}{p{14cm}}{\footnotesize{
$^{\clubsuit }$Pseudopotencial Largo \citep{Dolg1989}.}%} 
$^{\spadesuit}$Pseudopotencial Corto \citep{Cao2001}. %} \\
{\footnotesize $^1$ Utilizando la base de \cite{Dolg1993}.}% } \\
{\footnotesize $^2$ Utilizando la base de \cite{Yang2005}.}% } \\
{\footnotesize $^3$ Utilizando la base de \cite{Cao2002}.} % }} \\
{\footnotesize $^4$ Nivel SCS-MP2 \citep{Grim2003} y utilizando MPC 
\citep{Toma2005}. $^5$ Datos experimentales \citep{Dang2012}.}}%}
\end{tabular}\label{tDis}\end{table}
%\end{center}

En la tabla \ref{tDis} se observan los datos te\'oricos de las 
distancias promedios ox\'igeno-lant\'anido, presentados en las
figuras \ref{fDLO2} y \ref{fDLO3}. A partir de sus valores se 
obtiene, para el lant\'ano, que las distancias con menor diferencia a
los experimentales son aquellos que utilizan el pseudopotencial largo 
\citep{Dolg1989} y tanto las bases extendidas \citep{Yang2005} a 
nivel MP2$^{\clubsuit,2}$ como las bases corregidas \citep{Dolg1993} 
con el medio polarizable continuo a un nivel de c\'alculo 
SCS-MP2$^{\clubsuit,1,4}$ . Los dos resultados difieren en la tercera
decimal, 0.212 \% y 0.285\%, respectivamente, del dato experimental.
Lo mismo ocurre para el gadolinio, en el sistema con coordinaci\'on
nueve, pero ahora la menor diferencia es obtenida con el medio 
polarizable, 0.957\% del dato experimental y 1.169\% utilizando la
base extendida. En cambio, para el sistema de coordinaci\'on ocho, la
menor diferencia se obtiene con el pseudopotencial largo y su base 
corregida a nivel MP2 sin tomar en cuenta ni los efectos de bulto ni
la correcci\'on SCS, teniendo un diferencia del 0.240\% del dato
experimental y seguida por el c\'alculo con la base extendida con una
diferencia del 0.587\%. Para el lutecio se repiten los resultados
obtenidos para gadolinio con nueve aguas, el mejor resultados, con 
menor diferencia al dato experiemtnal es el c\'alculo 
MP2$^{\clubsuit,1,4}$ con una diferencia de 0.166\% y seguida de la
obtenida con el c\'alculo MP2$^{\clubsuit,2}$ con una diferencia del
0.618\%.

La tabla \ref{t3} presenta los efectos en las distancias promedio de 
los lant\'anido a los ox\'igenos que tiene hacer c\'alculos tomando
en cuenta los efectos del disolvente, hacer c\'alculos de m\'as alto
nivel, respecto al c\'alculo MP2, y la combinaci\'on de ambos, 
utilizando dos bases diferentes para los lant\'anidos. Se  observa 
que el nivel de c\'alculo SCS tiene un efecto de alargar las 
distancias respecto a las distancias obtenidas al nivel MP2, mientras 
que el efecto del disolvente es de comprimir al sistema obteniendo 
distancias m\'as cortas a las obtenidas sin PCM.
\begin{table}[h!]
\centering
\caption{\footnotesize Distancias promedio del lantano a los 
ox\'igeno con dos bases diferentes.}
\begin{tabular}{c|cccc|c}\hline\hline
Ln(base)  & MP2    & SCS    & MP2/PCM & SCS/PCM & Exp.  \\ \hline
La(Stoll) & 2.6196 & 2.6219 & 2.5791  & 2.5926  & 2.600 \\ 
La(Yang)  & 2.5945 & 2.6097 &         & 2.5693  & 2.600 \\ 
\hline \end{tabular}\label{t3}\end{table}


%%Adem\'as se puede observar que las diferencias en las distancias 
%%promedios del lantano 
%%a los ox\'igenos menos la del lantano a los hidr\'ogenos es de la 
%%misma magnitud $\sim$ 0.646 lo que sugiere que los c\'alculos con 
%%diferentes pseudopotenciales y diferentes bases deforma de manera 
%%equivalentes a las mol\'eculas de agua.
\section{N\'umero de coordinaci\'on de la primera capa de 
hidrataci\'on}

%\paragraph{Criterio de preferencia}
Para responder a la pregunta del n\'umero de coordinaci\'on de la 
primera capa de hidrataci\'on se usa la metodolog\'ia seguida por
\cite{Ciup2010}, la cual se basa en la energ\'ia de transici\'on como
criterio de preferencia y tambi\'en se consider\'o la estabilidad a
partir de la energ\'ia libre de hidrataci\'on ya que as\'i se toma
en cuenta la contribuci\'on entr\'opica. La reacci\'on es una 
transferencia hipot\'etica de una mol\'ecula de agua de la segunda 
esfera de hidrataci\'on a la primera.
%, una diferencia de energ\'ia 
%positiva indica una preferencia de la primera configuraci\'on sobre la 
%segunda. 

%\paragraph{Tabla \ref{tDE}}
%En la tabla \ref{tDE} se observa que el sistema 
%[La$\cdot$(H$_2$O)$_{8+1}$]$^{3+}$ es m\'as estable en todos los 
%casos en los que se utiliza el pseudopotencial largo, sin importar la
%bases,excepto cuando se usa un medio polarizable continuo para 
%modelar la influencia de las esferas de hidrataci\'on mayores, 
%efectos de bulto. El mismo resultado se obtiene en el caso del 
%gadolinio. En el caso del lutecio, los resultados muestran una 
%preferencia por un n\'umero de coordinaci\'on de ocho sobre el nueve 
%y del siete sobre el ocho, excepto en el c\'alculo que considera los 
%efectos de bulto, MP2$^{\clubsuit,1,4}$. El \'unico c\'alculo que 
%reproduce el n\'umero de coordinaci\'on experimental, tanto para el 
%lantano como para el lutecio, es aquel que toma en cuenta los efectos
%de bulto. La preferecia te\'orica por un n\'umero de coordinaci\'on 
%menor ya se report\'o, tanto a nivel MP2 como en c\'alculos DFT,
%por \cite{Ciup2010}. Los resultados utilizando el pseudopotencial 
%corto tambi\'en est\'an en desacuerdo con en n\'umero de 
%coordinaci\'on experimental.

%%\paragraph{Diferencias de energ\'ia}
%%Usando el pseudopotencial largo propuesto por \cite{Dolg1989} 
%%(incluci\'on de la capa 4f en el pseudopotencial) y su 
%%correspondiente base corregida \citep{Dolg1993} tanto a nivel 
%%Hartree-Fock como MP2, o usando la base extendida \citep{Yang2005} 
%%tambi\'en a nivel MP2 se obtienen resultados, respecto al n\'umero de
%%coordinaci\'on, en desacuerdo con los datos experimentales. Lo mismo 
%%ocurre cuando se usa el pseudopotencial corto propuesto por 
%%\cite{Cao2001}, el cual trata los electrones en las capas con 
%%n\'umero principal mayor a 3 de manera expl\'icita, y la base 
%%segmentada propuesta por ellos mismo \citep{Cao2002} a nivel MP2.
%\be
%\mbox{[Ln(H$_2$O)$_h\cdot$H$_2$O]$^{3+}$(aq)}\mathop{\longrightarrow}
%\limits^{\mbox {trans.}}\mbox{[Ln(H$_2$O)$_{h+1}$]$^{3+}$(aq)}
%\ee
%\begin{table}[h!]
\centering
\caption{\footnotesize Diferencias en la energ\'ia $(\Delta E 
[\frac{KJ}{mol}])$ de los sistemas $E$[Ln$\cdot$
(H$_2$O)$_{n+(9-n)}$]$^{3+}$-$E$[Ln$\cdot$(H$_2$O)$_{(n-1)+(10-n)}$]$^{3+}$.}
%%\resizebox*{1.\textwidth}{!}{
\begin{tabular}{c|ccccc}\hline\hline
Ion & HF$^{\clubsuit,1}$ & MP2$^{\clubsuit,1}$ & MP2$^{\clubsuit,2}$ & 
MP2$^{\spadesuit,3}$ & MP2$^{\clubsuit,1,4}$     \\ \hline
La$^{3+}$(n=9) & 5.90835 & 6.08942 & 9.50653 & 16.08627 & 
-17.04981 \\ 
Eu$^{3+}$(n=9) &         &16.47709 &20.69336 &          &
 -5.63170 \\
Gd$^{3+}$(n=9) &20.00788 &18.19787 &22.75232 &          &
 -4.67339 \\ 
Tb$^{3+}$(n=9) &         &         &         &          &
 -2.88815 \\
Lu$^{3+}$(n=9) &35.26074 &32.69055 &37.59342 & 53.47987 &
 10.52741 \\ 
Lu$^{3+}$(n=8) & 0.74643 & 0.91630 & 2.46640 &  9.65396 &
-16.24490 \\ 
\hline 
\multicolumn{6}{p{11.0cm}}{
{\footnotesize $^\clubsuit$Pseudopotencial Largo \citep{Dolg1989}. 
$^\spadesuit$ Pseudopotencial Corto \citep{Cao2001}}.
{\footnotesize $^1$ Utilizando la base de \cite{Dolg1993}.} 
{\footnotesize $^2$ Utilizando la base de \cite{Yang2005}.}
{\footnotesize $^3$ Utilizando la base de \cite{Cao2002}.}  
{\footnotesize $^4$ Nivel SCS-MP2 \citep{Grim2003} y utilizando MPC 
\citep{Toma2005}.}}
\end{tabular}\label{tDE}\end{table}

%\begin{table}[h!]
\centering
\caption{\footnotesize Diferencias en la energ\'ia $(\Delta E 
[\frac{KJ}{mol}])$ de los sistemas $E$[Ln$\cdot$
(H$_2$O)$_{n+0}$]$^{3+}$-$E$[Ln$\cdot$(H$_2$O)$_{(n-1)+1}$]$^{3+}$.
La energ\'ia calculada es a la geometr\'ia obtenida con el m\'etodo
SCS-MP2$^{\clubsuit,4}$}
\begin{tabular}{c|cccccc}\hline\hline
Ion & MP2$^{\clubsuit,1}$ & SCS-MP2$^{\clubsuit,1}$ & MP2$^{\clubsuit,4}$ &
SCS-MP2$^{\clubsuit,4}$ & MP2$^{\clubsuit,2}$ & SCS-MP2$^{\clubsuit,2}$ 
\\ \hline
La$^{3+}$(n=9) & 1,39152 & 0.84016 & -10.50882 & -13.33098 & -11.58022 & -12.47113  \\ 
Gd$^{3+}$(n=9) &18,60954 &18.16846 & -53.66076 & -49.66369  \\ 
Lu$^{3+}$(n=9) &32.95003 &32.63495 &   9.68284 &   9.06060  \\ 
Lu$^{3+}$(n=8) & 0.86642 & 0,91893 & -15.93679 & -16.85046  \\ 
\hline 
\multicolumn{7}{p{14.5cm}}{
{\footnotesize $^\clubsuit$Pseudopotencial Largo \citep{Dolg1989}. 
$^\spadesuit$ Pseudopotencial Corto \citep{Cao2001}}.
{\footnotesize $^1$ Utilizando la base de \cite{Dolg1993}.} 
{\footnotesize $^2$ Utilizando la base de \cite{Yang2005}.}
{\footnotesize $^3$ Utilizando la base de \cite{Cao2002}.}  
{\footnotesize $^4$ Nivel SCS-MP2 \citep{Grim2003} y utilizando MPC 
\citep{Toma2005}.}}
\end{tabular}\label{tSP}\end{table}

%\paragraph{Tabla \ref{tSP}}

Las tablas \ref{t1} y \ref{t2} presentan las diferencias de las 
energ\'ias del sistema a diferentes niveles de c\'alculo y las 
energ\'ias libres de las configuraciones optimizadas a nivel 
MP2/AUG-cc-pVDZ en fase gas (tabla \ref{t1}) y SCS/AUG-cc-pVDZ/PCM 
modelando los efectos del disolvente con un medio polarizable 
diel\'ectrico, D-PCM (tabla \ref{t2}), para los sistemas 
[Ln(H$_2$O)$_{n+m}$]$^{3+}$, con $n=9,8,7$ y $m+n=9$. Las diferencias
se tomaron
Ln$^{3+}_{n_1\to n_2}$=U$\left([\mbox{Ln(H}_2\mbox{O)}_{n_2+m_2}]^{3+}\right)$-
U$\left([\mbox{Ln(H}_2\mbox{O)}_{n_1+m_1}]^{3+}\right)$, los 
c\'alculos de las diferencias de las energ\'ias libres se calcularon 
de la misma manera. Diferencias negativas en las energ\'ias libres
quieren decir que el sistema con $n=n_2$ es m\'as estable que el 
sistema con $n=n_1$.

En la tabla \ref{t1} se observa que cuando no se usa un PCM el sistema
[La(H$_2$O)$_{8+1}$]$^{3+}$, es m\'as estable que el sistema 
[La(H$_2$O)$_{9+0}$]$^{3+}$, al presentar una diferencia positiva en 
las energ\'ias libres, contrario a lo que se encuentra en los 
experimentos. Para lutecio se tiene que su configuraci\'on estable es
[Lu(H$_2$O)$_{8+1}$]$^{3+}$, la misma encontrada en los experimentos. 
En el caso de gadolinio se tiene una estabilidad marcada del sistema
[Gd(H$_2$O)$_{8+1}$]$^{3+}$, mientras que los datos experimentales 
confirman una mezcla de los sistemas con $n=9$ y $n=8$. Cuando se
agregan los efectos del disolvente la estabilidad de lantano 
corresponde a la experimental, la del gadolinio cambia prediciendo
una n\'umero de coordinaci\'on de $n=9$ y para lutecio la diferencias
entre $n=9$ y $n=8$ se reduce.
\linespread{1.}
\begin{table}[h!]
\centering
\caption{\footnotesize Estabilidades energ\'eticas (kJ/mol) de las
configuraci\'on para determinar el n\'umero de coordinaci\'on. Las
energ\'ias son calculadas en las geometr\'ias optimizadas 
MP2/AUG-cc-pVDZ en fase gas.}%\label{t1}
\begin{tabular}{c|ccc|ccc|}\hline\hline
   & \multicolumn{3}{c}{Sin PCM} & \multicolumn{3}{|c|}{Con PCM} \\ 
Ln & $\Delta U$ (MP2)  & $\Delta U$ (SCS) & $\Delta G$          
   & $\Delta U$ (MP2)  & $\Delta U$ (SCS) & $\Delta G$ \\ \hline
La$^{3+}_{8\to 9}$ &   5.9966 &  10.2946 &   4.8598   
                & -10.5099 & -13.3323 & -28.9698 \\
Gd$^{3+}_{8\to 9}$ &  18.1973 &  17.1603 &  17.5095    
                & -53.6626 & -49.6640 & -17.1393 \\ 
Lu$^{3+}_{8\to 9}$ &  32.6560 &  31.6977 &  31.5900    
                &  10.4705 &   9.0606 &   2.1949 \\ 
Lu$^{3+}_{7\to 8}$ &  -1.3259 &  -2.6202 &  -5.5897    
                & -15.9342 & -16.8506 & -20.7204 \\ 
\hline \end{tabular}\label{t1}\end{table}

\linespread{1.75}

En la tabla \ref{t2} se presentan las diferencias de las energ\'ias 
pero de configuraci\'on optimizada considerando los efecto del 
disolvente, en la primera parte se presentan las diferencias de las 
energ\'ias sin considerar el PCM y en la segunda consider\'andolo. En
los c\'alculos sin PCM se observa que [La(H$_2$O)$_{9+0}$]$^{3+}$ es
m\'as estable en concordancia con la evidencia experimental, de nuevo
una estabilidad marcada para gadolinio y una peque\~na preferencia de
$n=8$ sobre $n=7$ para lutecio. Cuando los c\'alculos se hacen 
tomando en cuenta los efectos del disolvente los resultados son la
preferencia de $n=9$ sobre $n=8$ para lantano, una diferencia 
energ\'etica peque\~na para gadolinio, y una preferencia $n=8$ sobre
$n=9$ y $n=7$ para lutecio.
\linespread{1.}
\begin{table}[h!]
\centering
\caption{\footnotesize Estabilidad energ\'etica (kJ/mol) de las 
configuraciones para determinar n\'umero coordinaci\'on. Las 
energ\'ias son calculadas en las geometr\'ias optimizadas a nivel 
SCS/AUG-cc-pVDZ/PCM.}%\label{t1}
\begin{tabular}{c|ccc|ccc|}\hline\hline
& \multicolumn{3}{ c}{Sin PCM}&\multicolumn{3}{|c|}{Con	PCM} \\
Ln & $\Delta U$ (MP2)  & $\Delta U$ (SCS) & $\Delta G$ & 
     $\Delta U$ (MP2)  & $\Delta U$ (SCS) & $\Delta G$ \\ \hline
La$^{3+}_{8\to 9}$ &   1.4388 &   0.8349 &  -6.9943 
                & -16.8032 & -17.0474 & -23.4772 \\  
Gd$^{3+}_{8\to 9}$ &  28.9435 &  23.5744 &  20.5025 
                &  -4.3294 &  -4.1745 &   1.2917 \\    
Lu$^{3+}_{8\to 9}$ &  32.9553 &  32.6481 &  34.3809 
                &   8.4804 &   8.5145 &  10.3576 \\    
Lu$^{3+}_{7\to 8}$ &   0.8927 &   0.6879 &  -2.1634 
                & -18.4599 & -18.2787 & -22.9442 \\    
\hline \end{tabular}\label{t2}\end{table}

\linespread{1.75}

%La tabla \ref{t3} presenta los efectos en las distancias promedio de 
%los lant\'anido a los ox\'igenos que tiene hacer c\'alculos tomando
%en cuenta los efectos del disolvente, hacer c\'alculos de m\'as alto
%nivel, respecto al c\'alculo MP2, y la combinaci\'on de ambos, 
%utilizando dos bases diferentes para los lant\'anidos. Se  observa 
%que el nivel de c\'alculo SCS tiene un efecto de alargar las 
%distancias respecto a las distancias obtenidas al nivel MP2, mientras 
%que el efecto del disolvente es de comprimir al sistema obteniendo 
%distancias m\'as cortas a las obtenidas sin PCM.
%\begin{table}[h!]
\centering
\caption{\footnotesize Distancias promedio del lantano a los 
ox\'igeno con dos bases diferentes.}
\begin{tabular}{c|cccc|c}\hline\hline
Ln(base)  & MP2    & SCS    & MP2/PCM & SCS/PCM & Exp.  \\ \hline
La(Stoll) & 2.6196 & 2.6219 & 2.5791  & 2.5926  & 2.600 \\ 
La(Yang)  & 2.5945 & 2.6097 &         & 2.5693  & 2.600 \\ 
\hline \end{tabular}\label{t3}\end{table}


La optimizaci\'on de las configuraciones tomando en cuenta los 
efectos del bulto tanto para el sistema [Ln(H$_2$O)$_{n+m}$]$^{3+}$
como para el non\'amero de agua a un mismo nivel de c\'alculo nos
permite hacer predicciones las energ\'ias libres de hidrataci\'on y 
compararlas con los datos experimetales. La tabla \ref{t3} muestra 
las energ\'ias de hidrataci\'on de las configuraciones optimizadas a 
nivel SCS/AUG-cc-pVDZ/PCM para diferentes valores de $n$ y dos de los
datos experimentales reportados y usados como referencia en los 
art\'iculos de Ciupka {\it et al.} y Kutta y Clark. Se observa que en
el peor de los casos se tiene un error del 3\% del valor experimental.
%\begin{table}[h!]
\centering
\caption{\footnotesize Diferencias de energ\'ias sin PCM en las
         configuraciones SCS/PCM.}%\label{t1}
\begin{tabular}{c|ccc}\hline\hline
Ln & $\Delta U$ (MP2)  & $\Delta U$ (SCS) & $\Delta G$ \\ \hline
La$^{3+}_{9-8}$ &   1.4388 &   0.8349 &  -6.9943 \\
Gd$^{3+}_{9-8}$ &  28.9435 &  23.5744 &  20.5025 \\ 
Lu$^{3+}_{9-8}$ &  32.9553 &  32.6481 &  34.3809 \\ 
Lu$^{3+}_{8-7}$ &   0.8927 &   0.6879 &  -2.1634 \\ 
\hline \end{tabular}\label{t2}\end{table}

%\begin{table}[h!]
\centering
\caption{\footnotesize Diferencias de energ\'ias con PCM en la 
configuraci\'on MP2}%\label{t1}
\begin{tabular}{c|ccc}\hline\hline
Ln & $\Delta U$ (MP2)  & $\Delta U$ (SCS) & $\Delta G$ \\ \hline
La$^{3+}_{9-8}$ & -10.5099 & -13.3323 & -28.9698 \\
Gd$^{3+}_{9-8}$ & -53.6626 & -49.6640 & -17.1393 \\ 
Lu$^{3+}_{9-8}$ &  10.4705 &   9.0606 &          \\ 
Lu$^{3+}_{8-7}$ & -15.9342 & -16.8506 & -20.7204 \\ 
\hline \end{tabular}\label{t3}\end{table}

\linespread{1.}
\begin{table}[h!]
\centering
\caption{\footnotesize Energ\'ias libres de hidrataci\'on}%\label{t1}
\begin{tabular}{c|c|cc|c}\hline\hline
Ln      & $\Delta G_H$ & $\Delta G_H^{exp}$& $\Delta G_H^{exp}$ & e\% \\ \hline
La(9+0) &  3152.1081   & 3145   & 3060.7 & 2.99  \\
La(8+1) &  3129.0578   &        &        & 2.23  \\ \hline
Gd(9+0) &  3353.2807   & 3375   & 3291.7 & 1.87 \\
Gd(8+1) &  3354.5725   &        &        & 1.91 \\ \hline
Lu(9+0) &  3520.4253   & 3515   & 3487.7 & 0.94 \\
Lu(8+1) &  3530.7829   &        &        & 1.23 \\
Lu(7+2) &  3507.8387   &        &        & 0.58 \\
\hline \end{tabular}\label{t4}\end{table}

\linespread{1.75}

