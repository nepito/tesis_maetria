% Clarck JChemTheoryComput2008.4.708
Los an\'alisis de poblaci\'on son una forma com\'un de caracterizar
la estructura electr\'onica de los \'atomo met\'alicos. El conjunto 
de bases largas necesitan describir el bloque d y f de los metales
haciendo los m\'etodos no optimos de Mulliken y L\"owdin. El 
an\'alisis de poblaci\'on natural (NPA, por sus siglas en ingl\'es)
ha emergido como uno de los m\'etodos elegidos para tal sistemas 
porque su partici\'on del espacio orbital decrece la dependencia en 
la base. La NPA divide la carga molecular en componentes at\'omicas
an\'alogo al m\'etos de L\"owdin, donde una transformaci\'on 
sim\'etrica de ocupancia pesada es usada para seleccionar un AO y una
partici\'on para el conjunto de orbitales de carozo, de valencia y de
Rydberg, cada uno cntribuye deferentemente a la densidad. Los 
orbitales de carozo contribuyen exactamente con $2e$ a la poblaci\'on 
at\'omica, mientras los orbitales de valencia var\'ian en su 
contribuci\'on y el conjunto de Rydberg participa m\'inimamente en la
carga. En previos estudios, nuestro grupo y otros han destacado la
sensibilidad de las cargas NPA a la partici\'on inicial de la bases 
NAO dentro del conjunto de valencia y Rydberg. Esto es 
particularmente cierto  con ver a los metales, donde el bloque d la 
partici\'on de los NAO excluye el conjunto vac\'io de los orbitales p
del espacio de valencia del \'atomo met\'alico. Incluyendo el 4p AO 
en el conjunto de Rydberg  puede dar una gran carga positiva en el 
metal tanto si se considera parte del espacio de valencia, debindo la
interacci\'on  entre los orbitales vacios p y los ligando. La 
correcta partici\'on  de los NAO tambi\'en influye las predicciones 
en las cargas at\'omicas dentro de una serie.

Se examin\'o apropiadamente la forma de poner los NAO en el espacio 
de valencia  de los trivalentes La y Lu como su dependencia de la
carga NPA en el funcional y el conjunto de bases. Esto es 
particularmente importante como NPA hab\'ia sido extensivamente usado
para entender la variedad de efectos electr\'onicos incluyendo la 
influencia del alto n\'umero de coordinaci\'on de las reacciones
energ\'eticos de Ln(III) y la escructura electr\'onica. La 
partici\'on por normal (por de fault) en La(III) en La(H$_2$O)
$_8^{3+}$ ponen 4s y 5s, 4p y 5p, y 4d en el carozo NAO; el 6s y 4f
en la valencia; y 7-10s,6-10p, y 5h y 6f en el espacio Rydberg.
