%CiupkaPhysChemChemPhys2010.12.13215
La energ\'ia de amarre (?`Energ\'ia de interacci\'on) $D_0$ 
corresponde a
\be
Ln^{3+}(g)+(H_2O)_h(g)\rightarrow [Ln(H_2O)_h]^{3+}(g)
\ee
la de amarre corregida ener\'ia de punto cero (ZPE) $D_0$ con 
respecto a las estructuras de agua de equivalente tama\~no. Las
energ\'ia de amarre $D_0$ incrementan con el incremento de $Z$ debido
a la reducci\'on del radio del ion y as\'i la menor distancia 
Ln$^{III}$-O acompa\~nada por la fuerte interacci\'on 
electrost\'atica. $D_0$incrementa tambi\'en con $h$ ya que m\'as 
mol\'eculas de agua est\'an involucradas en la coordinaci\'on, aun la
energ\'ia de amarre por mol\'ecula decrece con el incremento de $h$ 
lo que es atribuido al crecimiento de la importancia de la 
interacci\'on est\'erica $H_2O$-$H_2O$ y la repulsi\'on en la 
primera esfera de hidrataci\'on. 

$\Delta E_{SCRF}$ incrementa en magnitud con el incremento de $Z$ 
por la disminuci\'on del tama\~no de la cavidad PCM mientras la 
tendencia se invierte con el incremento de $h$. Cuantitativamente, 
las energ\'ias de hidrataci\'on de los Ln$^{III}$
