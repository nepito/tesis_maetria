Las redes de enlaces de hidr\'ogenos presentes en la primera esfera
de hidrataci\'on del ion son reflejo del grado de organizaci\'on del 
agua y son una componente integral para el entendimiento de la 
conexi\'on entre la geometr\'ia y la termodin\'amica y la cin\'etica
de solvataci\'on. Mientras la definici\'on de un enlace de 
hidr\'ogeno es variado en la literatura, en el presente trabajo se 
utiliz\'o la definici\'on de Kuta 2010, la cual a su vez es la 
definici\'on propuesta por Hammerich 2008 y con una distancia de 
corte. En Hammerich 2008 un par de mol\'eculas de agua (A) y (B) 
est\'a ligada por un enlace de hidr\'ogeno si una de las distancias 
intermoleculares O$_{(A)}$$\cdots$H$_{(B)}$ satisfacen los siguientes 
dos criterios: 
\begin{itemize}
\item O$_{(A)}$ es la el vecino m\'as cercano de H$_{(B)}$ al que no 
est\'a qu\'imicamente ligado;
\item H$_{(B)}$ es el primero o el segundo vecino cercano 
intermolecular de O$_{(A)}$. 
\end{itemize}
Adicionalmente, la distancia de corte es $r_{O\cdots H}$ de $3.0$ \AA.
