Las tablas \ref{t1} y \ref{t2} presentan las diferencias de las 
energ\'ias del sistema a diferentes niveles de c\'alculo y las 
energ\'ias libres de las configuraciones optimizadas a nivel 
MP2/AUG-cc-pVDZ en fase gas (tabla \ref{t1}) y SCS/AUG-cc-pVDZ/PCM 
modelando los efectos del disolvente con un medio polarizable 
diel\'ectrico, D-PCM (tabla \ref{t2}), para los sistemas 
[Ln(H$_2$O)$_{n+m}$]$^{3+}$, con $n=9,8,7$ y $m+n=9$. Las diferencias
se tomaron
Ln$^{3+}_{n_1\to n_2}$=U$\left([\mbox{Ln(H}_2\mbox{O)}_{n_2+m_2}]^{3+}\right)$-
U$\left([\mbox{Ln(H}_2\mbox{O)}_{n_1+m_1}]^{3+}\right)$, los 
c\'alculos de las diferencias de las energ\'ias libres se calcularon 
de la misma manera. Diferencias negativas en las energ\'ias libres
quieren decir que el sistema con $n=n_2$ es m\'as estable que el 
sistema con $n=n_1$.

En la tabla \ref{t1} se observa que cuando no se usa un PCM el sistema
[La(H$_2$O)$_{8+1}$]$^{3+}$, es m\'as estable que el sistema 
[La(H$_2$O)$_{9+0}$]$^{3+}$, al presentar una diferencia positiva en 
las energ\'ias libres, contrario a lo que se encuentra en los 
experimentos. Para lutecio se tiene que su configuraci\'on estable es
[Lu(H$_2$O)$_{8+1}$]$^{3+}$, la misma encontrada en los experimentos. 
En el caso de gadolinio se tiene una estabilidad marcada del sistema
[Gd(H$_2$O)$_{8+1}$]$^{3+}$, mientras que los datos experimentales 
confirman una mezcla de los sistemas con $n=9$ y $n=8$. Cuando se
agregan los efectos del disolvente la estabilidad de lantano 
corresponde a la experimental, la del gadolinio cambia prediciendo
una n\'umero de coordinaci\'on de $n=9$ y para lutecio la diferencias
entre $n=9$ y $n=8$ se reduce.
\linespread{1.}
\begin{table}[h!]
\centering
\caption{\footnotesize Estabilidades energ\'eticas (kJ/mol) de las
configuraci\'on para determinar el n\'umero de coordinaci\'on. Las
energ\'ias son calculadas en las geometr\'ias optimizadas 
MP2/AUG-cc-pVDZ en fase gas.}%\label{t1}
\begin{tabular}{c|ccc|ccc|}\hline\hline
   & \multicolumn{3}{c}{Sin PCM} & \multicolumn{3}{|c|}{Con PCM} \\ 
Ln & $\Delta U$ (MP2)  & $\Delta U$ (SCS) & $\Delta G$          
   & $\Delta U$ (MP2)  & $\Delta U$ (SCS) & $\Delta G$ \\ \hline
La$^{3+}_{8\to 9}$ &   5.9966 &  10.2946 &   4.8598   
                & -10.5099 & -13.3323 & -28.9698 \\
Gd$^{3+}_{8\to 9}$ &  18.1973 &  17.1603 &  17.5095    
                & -53.6626 & -49.6640 & -17.1393 \\ 
Lu$^{3+}_{8\to 9}$ &  32.6560 &  31.6977 &  31.5900    
                &  10.4705 &   9.0606 &   2.1949 \\ 
Lu$^{3+}_{7\to 8}$ &  -1.3259 &  -2.6202 &  -5.5897    
                & -15.9342 & -16.8506 & -20.7204 \\ 
\hline \end{tabular}\label{t1}\end{table}

\linespread{1.75}

En la tabla \ref{t2} se presentan las diferencias de las energ\'ias 
pero de configuraci\'on optimizada considerando los efecto del 
disolvente, en la primera parte se presentan las diferencias de las 
energ\'ias sin considerar el PCM y en la segunda consider\'andolo. En
los c\'alculos sin PCM se observa que [La(H$_2$O)$_{9+0}$]$^{3+}$ es
m\'as estable en concordancia con la evidencia experimental, de nuevo
una estabilidad marcada para gadolinio y una peque\~na preferencia de
$n=8$ sobre $n=7$ para lutecio. Cuando los c\'alculos se hacen 
tomando en cuenta los efectos del disolvente los resultados son la
preferencia de $n=9$ sobre $n=8$ para lantano, una diferencia 
energ\'etica peque\~na para gadolinio, y una preferencia $n=8$ sobre
$n=9$ y $n=7$ para lutecio.
\linespread{1.}
\begin{table}[h!]
\centering
\caption{\footnotesize Estabilidad energ\'etica (kJ/mol) de las 
configuraciones para determinar n\'umero coordinaci\'on. Las 
energ\'ias son calculadas en las geometr\'ias optimizadas a nivel 
SCS/AUG-cc-pVDZ/PCM.}%\label{t1}
\begin{tabular}{c|ccc|ccc|}\hline\hline
& \multicolumn{3}{ c}{Sin PCM}&\multicolumn{3}{|c|}{Con	PCM} \\
Ln & $\Delta U$ (MP2)  & $\Delta U$ (SCS) & $\Delta G$ & 
     $\Delta U$ (MP2)  & $\Delta U$ (SCS) & $\Delta G$ \\ \hline
La$^{3+}_{8\to 9}$ &   1.4388 &   0.8349 &  -6.9943 
                & -16.8032 & -17.0474 & -23.4772 \\  
Gd$^{3+}_{8\to 9}$ &  28.9435 &  23.5744 &  20.5025 
                &  -4.3294 &  -4.1745 &   1.2917 \\    
Lu$^{3+}_{8\to 9}$ &  32.9553 &  32.6481 &  34.3809 
                &   8.4804 &   8.5145 &  10.3576 \\    
Lu$^{3+}_{7\to 8}$ &   0.8927 &   0.6879 &  -2.1634 
                & -18.4599 & -18.2787 & -22.9442 \\    
\hline \end{tabular}\label{t2}\end{table}

\linespread{1.75}

%La tabla \ref{t3} presenta los efectos en las distancias promedio de 
%los lant\'anido a los ox\'igenos que tiene hacer c\'alculos tomando
%en cuenta los efectos del disolvente, hacer c\'alculos de m\'as alto
%nivel, respecto al c\'alculo MP2, y la combinaci\'on de ambos, 
%utilizando dos bases diferentes para los lant\'anidos. Se  observa 
%que el nivel de c\'alculo SCS tiene un efecto de alargar las 
%distancias respecto a las distancias obtenidas al nivel MP2, mientras 
%que el efecto del disolvente es de comprimir al sistema obteniendo 
%distancias m\'as cortas a las obtenidas sin PCM.
%\begin{table}[h!]
\centering
\caption{\footnotesize Distancias promedio del lantano a los 
ox\'igeno con dos bases diferentes.}
\begin{tabular}{c|cccc|c}\hline\hline
Ln(base)  & MP2    & SCS    & MP2/PCM & SCS/PCM & Exp.  \\ \hline
La(Stoll) & 2.6196 & 2.6219 & 2.5791  & 2.5926  & 2.600 \\ 
La(Yang)  & 2.5945 & 2.6097 &         & 2.5693  & 2.600 \\ 
\hline \end{tabular}\label{t3}\end{table}


La optimizaci\'on de las configuraciones tomando en cuenta los 
efectos del bulto tanto para el sistema [Ln(H$_2$O)$_{n+m}$]$^{3+}$
como para el non\'amero de agua a un mismo nivel de c\'alculo nos
permite hacer predicciones las energ\'ias libres de hidrataci\'on y 
compararlas con los datos experimetales. La tabla \ref{t3} muestra 
las energ\'ias de hidrataci\'on de las configuraciones optimizadas a 
nivel SCS/AUG-cc-pVDZ/PCM para diferentes valores de $n$ y dos de los
datos experimentales reportados y usados como referencia en los 
art\'iculos de Ciupka {\it et al.} y Kutta y Clark. Se observa que en
el peor de los casos se tiene un error del 3\% del valor experimental.
%\begin{table}[h!]
\centering
\caption{\footnotesize Diferencias de energ\'ias sin PCM en las
         configuraciones SCS/PCM.}%\label{t1}
\begin{tabular}{c|ccc}\hline\hline
Ln & $\Delta U$ (MP2)  & $\Delta U$ (SCS) & $\Delta G$ \\ \hline
La$^{3+}_{9-8}$ &   1.4388 &   0.8349 &  -6.9943 \\
Gd$^{3+}_{9-8}$ &  28.9435 &  23.5744 &  20.5025 \\ 
Lu$^{3+}_{9-8}$ &  32.9553 &  32.6481 &  34.3809 \\ 
Lu$^{3+}_{8-7}$ &   0.8927 &   0.6879 &  -2.1634 \\ 
\hline \end{tabular}\label{t2}\end{table}

%\begin{table}[h!]
\centering
\caption{\footnotesize Diferencias de energ\'ias con PCM en la 
configuraci\'on MP2}%\label{t1}
\begin{tabular}{c|ccc}\hline\hline
Ln & $\Delta U$ (MP2)  & $\Delta U$ (SCS) & $\Delta G$ \\ \hline
La$^{3+}_{9-8}$ & -10.5099 & -13.3323 & -28.9698 \\
Gd$^{3+}_{9-8}$ & -53.6626 & -49.6640 & -17.1393 \\ 
Lu$^{3+}_{9-8}$ &  10.4705 &   9.0606 &          \\ 
Lu$^{3+}_{8-7}$ & -15.9342 & -16.8506 & -20.7204 \\ 
\hline \end{tabular}\label{t3}\end{table}

\linespread{1.}
\begin{table}[h!]
\centering
\caption{\footnotesize Energ\'ias libres de hidrataci\'on}%\label{t1}
\begin{tabular}{c|c|cc|c}\hline\hline
Ln      & $\Delta G_H$ & $\Delta G_H^{exp}$& $\Delta G_H^{exp}$ & e\% \\ \hline
La(9+0) &  3152.1081   & 3145   & 3060.7 & 2.99  \\
La(8+1) &  3129.0578   &        &        & 2.23  \\ \hline
Gd(9+0) &  3353.2807   & 3375   & 3291.7 & 1.87 \\
Gd(8+1) &  3354.5725   &        &        & 1.91 \\ \hline
Lu(9+0) &  3520.4253   & 3515   & 3487.7 & 0.94 \\
Lu(8+1) &  3530.7829   &        &        & 1.23 \\
Lu(7+2) &  3507.8387   &        &        & 0.58 \\
\hline \end{tabular}\label{t4}\end{table}

\linespread{1.75}
