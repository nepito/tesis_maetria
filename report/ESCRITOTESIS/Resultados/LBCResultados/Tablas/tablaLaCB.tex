 \begin{center}                                                                            
 \begin{table}[h!]
 \centering
 \caption{\footnotesize C\'alculos BCL para el sistema de La$^{3+}$ y 
una mol\'ecula agua, usando el el pseudo potencial de 28 electrones 
de Stuttgart y la correspondiente base para La, el pseudopotencial
de Stuttgart para el ox\'igeno y las bases AUG-cc-pViZ (i=D, T, Q y 5) 
para el hidr\'ogeno.}                                                                  
 \begin{tabular}{c|ccc}\hline\hline                                                          
 Sistema & $\langle r_{Ln-O}\rangle$ & $\Delta_{\textnormal{hyd}}\tn{H}$ 
 & $\Delta_{\textnormal{hyd}}\tn{H}_{cp}$ \\ \hline                                               
La1+0MP2CaBeDZ & 2.2982 & -401.35549 & -388.37423 \\
La1+0MP2CaBeTZ & 2.3154 & -387.60339 & -379.14067 \\
La1+0MP2CaBeQZ & 2.3186 & -382.93170 & -376.29999 \\
La1+0MP2CaBe5Z & 2.3176 & -382.21094 & -376.33214 \\
 \hline \end{tabular}\label{tLaCB}      
 \end{table}                                                           
 \end{center}                                                                              
