 \begin{center}                                                                            
 \begin{table}[h!]
 \centering
 \caption{\footnotesize C\'alculos BCL para el sistema de La$^{3+}$ y 
una mol\'ecula agua, usando el el pseudo potencial de 28 electrones 
de Stuttgart y la correspondiente base para La y las bases AUG-cc-pViZ
(i=D, T, Q y 5) para la mo\'ecula de agua.}                                                                  
 \begin{tabular}{c|ccc}\hline\hline                                                          
 Sistema & $\langle r_{Ln-O}\rangle$ & $\Delta_{\textnormal{hyd}}\tn{H}$ 
 & $\Delta_{\textnormal{hyd}}\tn{H}_{cp}$ \\ \hline                                               
La1+0MP2CaoDZ & 2.2958 & -396.35664 & -388.76668 \\
La1+0MP2CaoTZ & 2.2867 & -398.51879 & -394.17115 \\
La1+0MP2CaoQZ & 2.2802 & -401.72554 & -397.57012 \\
La1+0MP2Cao5Z & 2.2745 & -405.37835 & -397.97437 \\
 \hline \end{tabular}\label{tLaCao} 
 \end{table}                                                           
 \end{center}                                                                              
