 \begin{center}                                                                            
 \begin{table}[h!]
 \centering
 \caption{\footnotesize C\'alculos BCL para el sistema de Lu$^{3+}$ y 
una mol\'ecula agua, usando el el pseudo potencial de 28 electrones 
de Stuttgart y la correspondiente base para Lu y las bases AUG-cc-pViZ
(i=D, T, Q y 5) para la mo\'ecula de agua.}                                                                  
 \begin{tabular}{c|ccc}\hline\hline                                                          
 Sistema & $\langle r_{Ln-O}\rangle$ & $\Delta_{\textnormal{hyd}}\tn{H}$ 
 & $\Delta_{\textnormal{hyd}}\tn{H}_{cp}$ \\ \hline                                               
Lu1+0MP2CaoDZ & 2.0715 & -516.81044 & -506.30335 \\
Lu1+0MP2CaoTZ & 2.0658 & -520.04478 & -512.23601 \\
Lu1+0MP2CaoQZ & 2.0551 & -525.39512 & -514.69588 \\
Lu1+0MP2Cao5Z & 2.0402 & -532.78832 & -516.18853 \\
 \hline \end{tabular}\label{tLuCao}    
 \end{table}                                                           
 \end{center}                                                                              
