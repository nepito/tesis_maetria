 \begin{center}                                                                            
 \begin{table}[h!]
 \centering
 \caption{\footnotesize C\'alculos BCL para el sistema de Lu$^{3+}$ y 
una mol\'ecula agua, usando el el pseudo potencial de 28 electrones 
de Stuttgart y la correspondiente base para La, el pseudopotencial
de Stuttgart para el ox\'igeno y las bases AUG-cc-pViZ (i=D, T, Q y 5)
para el hidr\'ogeno.}                                                                  
 \begin{tabular}{c|ccc}\hline\hline                                                          
 Sistema & $\langle r_{Ln-O}\rangle$ & $\Delta_{\textnormal{hyd}}\tn{H}$ 
 & $\Delta_{\textnormal{hyd}}\tn{H}_{cp}$ \\ \hline                                               
Lu1+0MP2CaBeDZ & 2.0599 & -519.94047 & -505.15755 \\
Lu1+0MP2CaBeTZ & 2.0786 & -505.94920 & -494.93235 \\
Lu1+0MP2CaBeQZ & 2.0851 & -501.30933 & -491.64679 \\
Lu1+0MP2CaBe5Z & 2.0877 & -500.97550 & -491.26398 \\
 \hline \end{tabular}
 \label{tLuCB}     
 \end{table}                                                           
 \end{center}                                                                              
