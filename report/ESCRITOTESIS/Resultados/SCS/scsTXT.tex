%Resultados de los valores de las energ\'ias obtenidas con AG.
%Resultados de la tabla: CalCua/Calculos/CalCua/Energias
Sistem\'aticamente los valores de las energ\'ias SCS-MP2 son menos
estables que los valores de las energ\'ias MP2. 

%Solo c\'alculos hechos con Gaussian
La diferencia en la estabilidad va creciendo mientras se avanza en la
serie de los lant\'anidos teniendo la menor diferencia en el Lantano 
y la mayor en el Lutecio. Tambi\'en la diferencia aumenta cuando el
n\'umero de aguas en la primera esfera de hidrataci\'on decrece. 
%Hablando de las enrg\'ias $\alpha\alpha$ y $\alpha\beta$
La diferencia entre las magnitudes de las energ\'ias $\alpha\alpha$
y $\alpha\beta$ aumentan conforme la serie de los lant\'anidos avanza
pero disminuye para un mismo lant\'anido cuando disminuye el n\'umero
de mol\'eculas de agua en la primera capa de hidrataci\'on.
%%%%%%%%%%%%%%%%%%%%
\begin{figure}[h]
\centering
\includegraphics[height=8cm,angle=0]{../../../CalCua/Calculos/CalCua/VALIDACION/cSCS/Figuras/LaEneSCS.eps}
\caption{\small{Gr\'afica de las distancias promedio La-O para los
diferentes niveles de c\'alulo y diferentes optimizadores. Los 
c\'alculos hechos con AG son los que tienen la primera letra del 
s\'imbolo escrita en mayuscula y los hechos con el optimizador de 
Gaussian est\'a escrito en min\'uscula.}}
\label{fig11}
\end{figure}
%%%%%%%%%%%%%%%%%%%%
%%%%%%%%%%%%%%%%%%%%
\begin{figure}[h]
\centering
\includegraphics[height=8cm,angle=0]{../../../CalCua/Calculos/CalCua/VALIDACION/cSCS/Figuras/LaEneASCS.eps}
\caption{\small{Gr\'afica de las distancias promedio La-O para los
diferentes niveles de c\'alulo y diferentes optimizadores. Los 
c\'alculos hechos con AG son los que tienen la primera letra del 
s\'imbolo escrita en mayuscula y los hechos con el optimizador de 
Gaussian est\'a escrito en min\'uscula.}}
\label{fig111}
\end{figure}
%%%%%%%%%%%%%%%%%%%%

%Resultados obtenidos con Algoritmos gen\'eticos
Cuando para los c\'alculos de las mol\'eculas de agua se utiliza la
base 6-31G* la tendencia se repite, es decir, la energ\'ia MP2 es 
m\'as estable que la energ\'ia SCS-MP2 y la diferencia entre ellas
aumenta conforme aumenta la serie de los lant\'anidos y para un mismo
lant\'anido la diferenica aumenta cuando el n\'umero de mol\'eculas 
de agua en la primera esfera disminuye.%Al parecer los resultados son
%independientes de la base que se utiliza para el agua.
 Adem\'as, cuando se optimiza utilizando un medio polarizable continuo
con la finalidad de agregar los efectos de bulto, la diferencia entre
las energ\'ias MP2 y SCS-MP2 es mayor para cada uno de los casos
que su contraparte cuando no se utiliza el PCM. Tambi\'en la 
tendencia de la diferencia entre las energ\'ias $\alpha\alpha$ y
$\alpha\beta$ se mantiene: aumenta cuando se avanza en la serie de
los lant\'anidos y para un mismo miembro de la serie, la diferencia
disminuye cuando en n\'umero de mol\'eculas de agua es menor en la
primera capa de hidrataci\'on, pero la proporci\'on que diminuye la 
diferencia es m\'as peque\~na que la que se observa con la base
AUG-cc-pVDZ. Al agregar PCM se repite el comportamiento general, la 
diferencia es mayor cuando se avanza en la serie y {\it revisar:} menor cuanto menor
sea el n\'umero de aguas en la primera esfera para un sistema dado, 
con dos diferencias, la disminuci\'on de la diferencia al pasar de un
sistema de m\'as mol\'eculas a otro con menos mol\'eculas es mayor 
que el observado cuando no de usa PCM y para Lutecio con nueve agua en
la primera capa con PCM la diferencia aumenta.

%%%%%%%%%%%%%%%%%%%%
\begin{figure}[h]
\centering
\includegraphics[height=8cm,angle=0]{../../../CalCua/Calculos/CalCua/VALIDACION/cSCS/Figuras/GdEneSCS.eps}
\caption{\small{Gr\'afica de las distancias promedio La-O para los
diferentes niveles de c\'alulo y diferentes optimizadores. Los 
c\'alculos hechos con AG son los que tienen la primera letra del 
s\'imbolo escrita en mayuscula y los hechos con el optimizador de 
Gaussian est\'a escrito en min\'uscula.}}
\label{fig12}
\end{figure}
%%%%%%%%%%%%%%%%%%%%
%%%%%%%%%%%%%%%%%%%%
\begin{figure}[h]
\centering
\includegraphics[height=8cm,angle=0]{../../../CalCua/Calculos/CalCua/VALIDACION/cSCS/Figuras/GdEneASCS.eps}
\caption{\small{Gr\'afica de las distancias promedio La-O para los
diferentes niveles de c\'alulo y diferentes optimizadores. Los 
c\'alculos hechos con AG son los que tienen la primera letra del 
s\'imbolo escrita en mayuscula y los hechos con el optimizador de 
Gaussian est\'a escrito en min\'uscula.}}
\label{fig121}
\end{figure}
%%%%%%%%%%%%%%%%%%%%
% Los mismos resultados pero ahora con PCM pero usando AG.
La tendencia en los resultados obtenidos con AG es la misma a los
obtenidos cuando se utiliza el optimizador de Gaussian: energ\'ias 
MP2 m\'as estables que las energ\'ias SCS-MP2. Incremento en la 
diferencia de la estabilidad cuando la serie avanza, mientras el
n\'umero de mol\'eculas de agua en la primera capa disminuye o cuando
para un mismo sistema (lant\'anido y n\'umero de mol\'eculas de agua
en la primera capa iguales) pero se utiliza el PCM para simular los
efectos de bulto. Los mismo ocurre con la tendencia en la diferencia 
de energ\'ia $\alpha\alpha$ y $\alpha\beta$.

%%%%%%%%%%%%%%%%%%%%
\begin{figure}[h]
\centering
\includegraphics[height=8cm,angle=0]{../../../CalCua/Calculos/CalCua/VALIDACION/cSCS/Figuras/LuEneSCS.eps}
\caption{\small{Gr\'afica de las distancias promedio La-O para los
diferentes niveles de c\'alulo y diferentes optimizadores. Los 
c\'alculos hechos con AG son los que tienen la primera letra del 
s\'imbolo escrita en mayuscula y los hechos con el optimizador de 
Gaussian est\'a escrito en min\'uscula.}}
\label{fig13}
\end{figure}
%%%%%%%%%%%%%%%%%%%%
%%%%%%%%%%%%%%%%%%%%
\begin{figure}[h]
\centering
\includegraphics[height=8cm,angle=0]{../../../CalCua/Calculos/CalCua/VALIDACION/cSCS/Figuras/LuEneASCS.eps}
\caption{\small{Gr\'afica de las distancias promedio La-O para los
diferentes niveles de c\'alulo y diferentes optimizadores. Los 
c\'alculos hechos con AG son los que tienen la primera letra del 
s\'imbolo escrita en mayuscula y los hechos con el optimizador de 
Gaussian est\'a escrito en min\'uscula.}}
\label{fig131}
\end{figure}
%%%%%%%%%%%%%%%%%%%%
%Comparaci\'on de las energ\'ias SCS-MP2 y MP2 cuando se optimiza 
%SCS.
Cuando las energ\'ias de optimizaci\'on son las SCS-MP2 es decir
cuando se busca minimizar la energ\'ia SCS-MP2 en lugar de la MP2, la
diferencia entre las energ\'ias SCS-MP2 y las energ\'ias MP2 es menor
con respecto a las diferencias de energ\'ias cuando la optimizaci\'on
es MP2, pero las energ\'ias MP2 siguen siendo m\'as estables a 
aquellas en la misma configuraci\'on pero SCS-MP2. La tendencia sigue
siendo la misma, la diferencia crece, mientras se avanza en la serie
de los lant\'anidos y para un mismo lant\'anido la la diferencia 
crece cuando el n\'umero de mol\'eculas de agua en la primera capa de
hidrataci\'on disminuye. Para el caso de las diferencias entre las
energ\'ias $\alpha\alpha$ y $\alpha\beta$ se repiten los resultados
encontrados en los casos anteriores.

%Comparaci\'on de los efectos de la optimizaci\'on SCS-MP2 en las 
%energ\'ias SCS-MP2 y MP2 con respecto a los efecto de la 
%optimizaci\'on MP2.
Cuando se compara la diferencia de las energ\'ias SCS-MP2 y MP2 de
las configuraciones optimizadas MP2 y aquellas SCS-MP2 ambas usando
AG como optimizador, se encuentra que las diferencias entre las 
energ\'ias son mayores cuando solo se optimiza MP2, sin importar el
lant\'anido con el que se trabaje o el n\'umero de mol\'eculas de 
agua en la primera capa de hidrataci\'on. El mismo comportamiento es 
encontrado con las energ\'ias $\alpha\alpha$ y $\alpha\beta$. Cuando
se compara las diferencias de las energ\'ias SCS-MP2 y MP2 de las 
optimizadas MP2 y aquellas SCS-MP2, estas usando AG y aquellas el 
optimizador de Gaussian, otra vez la diferencia m\'as grande se 
encuentra con las optimizaciones hechas a nivel MP2 y lo mismo para
las energ\'ias $\alpha\alpha$ y $\alpha\beta$. Comparando las 
diferecnias de las energ\'as $\alpha\alpha$ y $\alpha\beta$ de las
configuraciones optimizadas con los tres m\'etodos la mayor 
diferencia es encontrada en las optimizaciones MP2 utilizando 
Gaussian mientras que las optimizaciones hechas con AG las 
diferencias son mucha m\'as cercanas entre s\'i que lo que son a la
obtenida por Gaussian, pero en las diferencia entre las energ\'ias
MP2 y SCS-MP2 las diferencias m\'as cercanas son las optimizaciones
MP2, excepto para el sistema La9+0.

%Las diferencias con PCM
En las optimizaciones utilizando PCM las diferencias entre las 
energ\'ias MP2 y SCS-MP2 siguen el mismo patr\'on las menores 
diferencias se obtienen en las cofiguraciones optimizadas a nivel 
SCS-MP2, aumentando mientras se avanza en la serie o se diminuye el
n\'umero de mol\'eculas de agua en la primera capa de hidrataci\'on.
En el caso de las diferencias de las energ\'ias $\alpha\alpha$ y
$\alpha\beta$ las diferencias crece cuando se avanza en la serie de
lant\'anidos y es muy parecida en el caso de las optimizaciones 
hechas con AG. %¿Se le puede atribuir esto al hecho de que las 
%optimizaciiones no se realice a nivel HF? 

%Efecto en las distancias del SCS
El efecto en las distancias de la optimizaci\'on SCS-MP2 es la de 
obtener distancias m\'as grandes que aquellas obtenidas cuando se usa
la energ\'ia MP2 como criterio de optimizaci\'on sin importar si las
configuraciones son logradas con Gaussian o con AG. Esto ocurre para
los tres lant\'anidos y para las dos situaciones del n\'umero de
mol\'eculas de agua en la primera capa de hidrataci\'on. Cuando se
agraga los efectos del bulto los resultados son los mismos la 
optimizaci\'on SCS-MP2 proporciona distancias promedio m\'as largas.
