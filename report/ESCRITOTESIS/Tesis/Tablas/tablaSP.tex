\begin{table}[h!]
\centering
\caption{\footnotesize Diferencias en la energ\'ia $(\Delta E 
[\frac{KJ}{mol}])$ de los sistemas $E$[Ln$\cdot$
(H$_2$O)$_{n+0}$]$^{3+}$-$E$[Ln$\cdot$(H$_2$O)$_{(n-1)+1}$]$^{3+}$.
La energ\'ia calculada es a la geometr\'ia obtenida con el m\'etodo
SCS-MP2$^{\clubsuit,4}$}
\begin{tabular}{c|cccccc}\hline\hline
Ion & MP2$^{\clubsuit,1}$ & SCS-MP2$^{\clubsuit,1}$ & MP2$^{\clubsuit,4}$ &
SCS-MP2$^{\clubsuit,4}$ & MP2$^{\clubsuit,2}$ & SCS-MP2$^{\clubsuit,2}$ 
\\ \hline
La$^{3+}$(n=9) & 1,39152 & 0.84016 & -10.50882 & -13.33098 & -11.58022 & -12.47113  \\ 
Gd$^{3+}$(n=9) &18,60954 &18.16846 & -53.66076 & -49.66369  \\ 
Lu$^{3+}$(n=9) &32.95003 &32.63495 &   9.68284 &   9.06060  \\ 
Lu$^{3+}$(n=8) & 0.86642 & 0,91893 & -15.93679 & -16.85046  \\ 
\hline 
\multicolumn{7}{p{14.5cm}}{
{\footnotesize $^\clubsuit$Pseudopotencial Largo \citep{Dolg1989}. 
$^\spadesuit$ Pseudopotencial Corto \citep{Cao2001}}.
{\footnotesize $^1$ Utilizando la base de \cite{Dolg1993}.} 
{\footnotesize $^2$ Utilizando la base de \cite{Yang2005}.}
{\footnotesize $^3$ Utilizando la base de \cite{Cao2002}.}  
{\footnotesize $^4$ Nivel SCS-MP2 \citep{Grim2003} y utilizando MPC 
\citep{Toma2005}.}}
\end{tabular}\label{tSP}\end{table}
